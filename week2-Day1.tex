\part{Linear-Equations}
\lecture{Linear Equations}{Linear-Equations}


\title{Ordinary Differential Equations}
\subtitle{Math 232 - Week 2, Day 1}

\author{Kelly Black}
\institute{Clarkson University}
\date{5 Sep 2011}

\begin{frame}
  \titlepage
\end{frame}

\begin{frame}
  \frametitle{Outline}
  \tableofcontents[pausesection,hideallsubsections]
\end{frame}


\section{Linear Equations}


\begin{frame}
  \frametitle{Linear Equations}

  If we have variables $x_1$, $x_2$, \ldots, $x_n$ then an equation of
  the form
  \begin{eqnarray*}
    a_1 x_1 + a_2 x_2 + \cdots + a_n x_n & = & c,
  \end{eqnarray*}
  where $a_1$, $a_2$, \ldots, $a_n$, and $c$ are constants, is a
  \textbf{linear equation}.

  if $c$ is zero the equation is \textbf{homogeneous}.

\end{frame}


\begin{frame}
  \frametitle{Linear Equations}

  \begin{eqnarray*}
    3 x_1 + 4 x_2 = 5 & & \uncover<2->{\mathrm{Linear}} \\
    3 x_1 + 4 x^3_2 = 5 & & \uncover<3->{\mathrm{Not~Linear}} \\
    3 \sqrt{x_1} + 4 x_2 = 5 & & \uncover<4->{\mathrm{Not~Linear}} \\
  \end{eqnarray*}


\end{frame}


\begin{frame}
  \frametitle{Linear Differential Equations}

  In the context of DEs an equation is linear if it is in the form 
  \begin{eqnarray*}
    a_0(t) y(t) + a_1(t) \frac{d}{dt}y(t) + \cdots +
    a_n(t) \frac{d^n}{dt^n}y(t) & = & f(t).
  \end{eqnarray*}

  if $f(t)$ is zero the equation is homogeneous.

\end{frame}


\begin{frame}
  \frametitle{Example}

  \vspace*{-3em}
  \begin{eqnarray*}
    3t y + y' - \sin(t) y'' & = & \ln(t)
  \end{eqnarray*}
  \centerline{\uncover<2->{Linear!}}

  \begin{eqnarray*}
    3t \sqrt{y} + y' - \sin(t) y'' & = & \ln(t)
  \end{eqnarray*}
  \centerline{\uncover<3->{Not Linear!}}

  \begin{eqnarray*}
    4 y + 3 y' -  y'' & = & 0
  \end{eqnarray*}
  \centerline{\uncover<4->{Linear, \textbf{constant coefficient}, and homogeneous}}

  Recognizing the form is the hardest part!

\end{frame}

\section{Solutions to Differential Equations}

\begin{frame}
  \frametitle{Solutions to Differential Equations}

  If an equation is linear and homogeneous then:
  \begin{itemize}
  \item If $y_1(t)$ is a solution, and
  \item if $y_2(t)$ is a solution
  \end{itemize}

  then, 
  \begin{eqnarray*}
    y_1(t) + y_2(t)
  \end{eqnarray*}
  is a solution.

  Also if $C_1$ and $C_2$ are constants then
  \begin{eqnarray*}
    C_1 y_1(t) + C_2 y_2(t)
  \end{eqnarray*}
  is also a solution.

  Because:
  \begin{eqnarray*}
    \frac{d}{dt} \lp C_1 y_1(t) + C_2 y_2(t) \rp & = & 
    C_1 y_1'(t) + C_2 y_2'(t)
  \end{eqnarray*}


\end{frame}


\begin{frame}
  \frametitle{Example}

  \begin{eqnarray*}
    y'' + y & = & 0, \\
    y_1(t) & = & \cos(t) \\
    y_2(t) & = & \sin(t).
  \end{eqnarray*}

  Because:
  \begin{eqnarray*}
    y''_1(t) & = & -\cos(t), \\
    \Rightarrow y''_1(t) + y_1(t) & = & -\cos(t)+\cos(t), \\
    y''_2(t) & = & -\sin(t), \\
    \Rightarrow y''_2(t) + y_2(t) & = & -\sin(t)+\sin(t).
  \end{eqnarray*}

\end{frame}

\begin{frame}
  \frametitle{Example}

  Also:
  \begin{eqnarray*}
    y(t) & = & C_1 \cos(t) + C_2 \sin(t), \\
    y'(t) & = & -C_1 \sin(t) + C_2 \cos(t), \\
    y''(t) & = & -C_1 \cos(t) - C_2 \sin(t), \\
    y''(t) + y(t) & = & -C_1 \cos(t) - C_2 \sin(t) + C_1 \cos(t) + C_2 \sin(t) \\
    & = & 0
  \end{eqnarray*}


\end{frame}


\section{Nonhomogeneous Solutions and Homogeneous Solutions}


\begin{frame}
  \frametitle{It Gets Worse!}

  Suppose that an equation is linear but not homogeneous,
  \begin{eqnarray*}
    a_0(t) y_p(t) + a_1(t) \frac{d}{dt} y_p(t) + \cdot + a_n(t) \frac{d^n}{dt^n} y_p(t) & = & f(t)
  \end{eqnarray*}
  where $f(t)$ is not zero.

  The homogeneous version is 
  \begin{eqnarray*}
    a_0(t) y_h(t) + a_1(t) \frac{d}{dt} y_h(t) + \cdot + a_n(t) \frac{d^n}{dt^n} y_h(t) & = & 0.
  \end{eqnarray*}

  then $y_p(t)+y_h(t)$ is a solution to the original equation!


\end{frame}


\begin{frame}
  \frametitle{Example}

  \begin{eqnarray*}
    y' - y & = & t
  \end{eqnarray*}

  Particular: $y_p = -t-1$

  \begin{eqnarray*}
    y'_p & = & -1, \\
    -1 - (-t -1) & = & t.
  \end{eqnarray*}

  Homogeneous: $y_h = k e^t$

  \begin{eqnarray*}
    y'_h - y_h & = & 0, \\
    y'_h & = & k e^t, \\
    k e^t - k e^t & = & 0.
  \end{eqnarray*}

\end{frame}

\begin{frame}
  \frametitle{Example}

  \begin{eqnarray*}
    y' - y & = & t
  \end{eqnarray*}

  \begin{eqnarray*}
    y & = & -t - 1 + ke^t, \\
    y' & = & -1 + ke^t, \\
    -1-ke^t - (-t-1-ke^t) & = & t.
  \end{eqnarray*}

\end{frame}


\begin{frame}
  \frametitle{Example}

  \begin{eqnarray*}
    y'' + 4y & = & t.
  \end{eqnarray*}

  \begin{eqnarray*}
    y_p & = & \frac{1}{4} t, \\
    y'_p & = & \frac{1}{4}, \\
    y''_p & = & 0, \\
    0 + 4\frac{1}{4} t & = & t
  \end{eqnarray*}
\end{frame}

\begin{frame}
  \frametitle{Example - Homogeneous Part}

  \begin{eqnarray*}
    y''_h + 4y_h & = & 0, \\
    y_h & = & C_1 \cos(2t) + C_2 \sin(2t), \\
    y'_h & = & -2 C_1 \sin(2t) + 2 C_2 \cos(2t), \\
    y''_h & = & -4 C_1 \cos(2t) - 4 C_2 \sin(2t), 
  \end{eqnarray*}
  \begin{eqnarray*}
    -4 C_1 \cos(2t) - 4 C_2 \sin(2t) + 4\lp C_1 \cos(2t) + C_2 \sin(2t) \rp & = & 0.
  \end{eqnarray*}

\end{frame}


\begin{frame}
  \frametitle{Example - Bring them together}

  \begin{eqnarray*}
    y & = & y_p + y_h \\
    & = & \frac{1}{4} t + C_1 \cos(2t) + C_2 \sin(2t), \\
    y' & = & \frac{1}{4}  - 2 C_1 \sin(2t) + 2 C_2 \cos(2t), \\
    y'' & = & - 4 C_1 \cos(2t) - 4 C_2 \sin(2t),
  \end{eqnarray*}

  \begin{eqnarray*}
    - 4 C_1 \cos(2t) - 4 C_2 \sin(2t) + 4 \lp \frac{1}{4} t + C_1 \cos(2t) + C_2 \sin(2t) \rp & = & t
  \end{eqnarray*}

\end{frame}


\begin{frame}
  \frametitle{Example}

  \begin{eqnarray*}
    y' - 3y & = & 4
  \end{eqnarray*}

  \begin{eqnarray*}
    y_p & = & -\frac{4}{3}, \\
    y_h & = & k e^{3t}
  \end{eqnarray*}

  \begin{eqnarray*}
    y & = & -\frac{4}{3} + k e^{3t}, \\
    y' & = & 3 k e^{3t}, \\
    3 k e^{3t} - 3 \lp -\frac{4}{3} + k e^{3t} \rp & = & 4
  \end{eqnarray*}


\end{frame}


\begin{frame}
  \frametitle{Example}

  \begin{eqnarray*}
    y' - 3ty & = & t^3
  \end{eqnarray*}

  \begin{eqnarray*}
    y_p & = & -\frac{1}{3} t^2 - \frac{2}{9}, \\
    y_h & = & k e^{\frac{3}{2}t^2}
  \end{eqnarray*}

  \begin{eqnarray*}
    y & = & -\frac{1}{3} t^2 - \frac{2}{9} + k e^{\frac{3}{2}t^2} \\
    y' & = & -\frac{2}{3} t + 3t k e^{\frac{3}{2}t^2} \\
    -\frac{2}{3} t + 3t k e^{\frac{3}{2}t^2} - 3t \lp -\frac{1}{3} t^2 - \frac{2}{9} + k e^{\frac{3}{2}t^2} \rp & = & t^3
  \end{eqnarray*}

\end{frame}


% LocalWords:  Clarkson pausesection hideallsubsections Nonhomogeneous
