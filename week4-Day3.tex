\part{Complex-Numbers-I}
\lecture{Complex Numbers I}{Complex-Numbers-I}


\title{Ordinary Differential Equations}
\subtitle{Math 232 - Week 4, Day 3}

\author{Kelly Black}
\institute{Clarkson University}
\date{23 Sep 2011}

\begin{frame}
  \titlepage
\end{frame}

\begin{frame}
  \frametitle{Outline}
  \tableofcontents[pausesection,hideallsubsections]
\end{frame}


\section{The Letter i}


\begin{frame}
  \frametitle{The Letter i}

  Definition: 
  \begin{eqnarray*}
    i & = & \sqrt{-1} 
  \end{eqnarray*}

  Which means....
  \begin{eqnarray*}
    i^2 & = & -1
  \end{eqnarray*}

\end{frame}


\begin{frame}
  \frametitle{Complex Numbers}

  Definition: A complex number is a  number that can be expressed as
  \begin{eqnarray*}
    a + bi,
  \end{eqnarray*}
  where $a$ and $b$ are real numbers.

\end{frame}



\begin{frame}
  \frametitle{Example}

  \begin{eqnarray*}
    z_1 & = & 3 + 2i \\
    z_2 & = & 5 - 7i
  \end{eqnarray*}

\end{frame}

\section{Operations With Complex Numbers}

\begin{frame}
  \frametitle{Addition}

  \begin{eqnarray*}
    (3+2i) + (5-7i) & = & (3+5) + (2-7)i \\
    & = & 8 - 5i\\
    ~ \\
    (4-2i) + (6-3i) & = & (4+6) + (-2-3)i \\
    & = & 10 - 5i \\
  \end{eqnarray*}

\end{frame}

\begin{frame}
  \frametitle{Multiplication}

  \begin{eqnarray*}
    (3+2i)(5-7i) & = & 15 + 10i - 21i - 14 i^2 \\
    & = & 15 - 11i + 14 \\
    & = & 29 - 11 i
  \end{eqnarray*}

  \begin{eqnarray*}
    (4-2i)(6+3i) & = & 24 - 12i + 12 i - 6i^2 \\
    & = & 24 + 6 \\
    & = & 30
  \end{eqnarray*}

\end{frame}


\section{Complex Conjugate}

\begin{frame}
  \frametitle{The Complex Conjugate}

  Definition: The complex conjugate of $z$, denoted $\bar{z}$, is
  found by reversing the sign of the imaginary part.
  
  Example:
  For the numbers given below:
  \begin{eqnarray*}
    z_1 & = & 3 + 2i \\
    z_2 & = & 5 - 6i 
  \end{eqnarray*}

  Their complex conjugates are the following:
  \begin{eqnarray*}
    \bar{z_1} & = & 3 - 2i \\
    \bar{z_2} & = & 5 + 6i 
  \end{eqnarray*}


\end{frame}

\begin{frame}
  \frametitle{Special Property of the Complex Conjugate}

  \begin{eqnarray*}
    z\cdot\bar{z} & = & (a+bi)(a-bi) \\
    & = & a^2 +abi - abi - b^2 i^2 \\
    & = & a^2 + b^2
  \end{eqnarray*}

\end{frame}



\begin{frame}
  \frametitle{So What?}

  \begin{eqnarray*}
    \frac{5-3i}{2+3i} & = & \frac{(5-3i)(2-3i)}{(2+3i)(2-3i)} \\
    & = & \frac{10-6i-15i+9i^2}{4+6i-6i-9i^2} \\
    & = & \frac{1-21i}{13} \\
    & = & \frac{1}{13} - \frac{21}{13} i
  \end{eqnarray*}

\end{frame}

\begin{frame}
  \frametitle{Example}

  \begin{eqnarray*}
    \frac{2+3i}{5-3i} & = & \frac{(2+3i)(5+3i)}{(5-3i)(5+3i)} \\
    & = & \frac{10+15i+6i+9i^2}{25-15i+15i-9i^2} \\
    & = & \frac{1+21i}{34} \\
    & = & \frac{1}{34} + \frac{21}{34}i
  \end{eqnarray*}

\end{frame}

\section{Graphical View of Complex Numbers}

\begin{frame}
  \frametitle{Graphical View of Complex Numbers}
 
  $z=a+bi$.

  
  \begin{eqnarray*}
    a & = & r\cos(\theta) \\
    b & = & r\sin(\theta) \\
    r & = & \sqrt{a^2+b^2} \\
    \tan(\theta) & = & \frac{b}{a}
  \end{eqnarray*}

\end{frame}


\begin{frame}
  \frametitle{Example}

  \begin{eqnarray*}
    x & = &  2 + 2i
  \end{eqnarray*}

\end{frame}


\begin{frame}
  \frametitle{Example}

  \begin{eqnarray*}
    z & = & -2 + 2i
  \end{eqnarray*}
    
\end{frame}

\section{Euler's Formula}

\begin{frame}
  \frametitle{A Differential Equation}

  Note: $i$ is a constant.
  \begin{eqnarray*}
    \Rightarrow \frac{d}{dt} e^{it} & = & i e^{it}.
  \end{eqnarray*}

  The function
  \begin{eqnarray*}
    y(t) & = & e^{it}
  \end{eqnarray*}
  is a solution to the differential equation
  \begin{eqnarray*}
    y' & = & iy, \\
    y(0) & = & 1.
  \end{eqnarray*}

\end{frame}

\begin{frame}
  \frametitle{Another Solution}

  There is another solution,
  \begin{eqnarray*}
    y(t) & = & \cos(t) + i \sin(t).
  \end{eqnarray*}

  \begin{eqnarray*}
    \frac{d}{dt} y(t) & = & -\sin(t) + i \cos(t), \\
    y(0) & = & \cos(0) + i\sin(0) \\
    & = & 1.
  \end{eqnarray*}

\end{frame}


\begin{frame}
  \frametitle{Euler's Formula}

  \begin{eqnarray*}
    e^{it} & = & \cos(t) + i\sin(t)
  \end{eqnarray*}

  or

  \begin{eqnarray*}
    e^{i\theta} & = & \cos(\theta) + i\sin(\theta)
  \end{eqnarray*}
  
\end{frame}

\begin{frame}
  \frametitle{Another way to express complex numbers}

  Suppose we have
  \begin{eqnarray*}
    z & = & a + bi, \\
    & = & r\cos(\theta) + ir\sin(\theta) \\
    & = & r \lp \cos(\theta) + i\sin(\theta) \rp \\
    & = & r e^{i\theta}
  \end{eqnarray*}

  This is the ``Euler form'' for the complex number.

\end{frame}

\begin{frame}
  \frametitle{Example}
  
  \begin{eqnarray*}
    z & = & 2 + 2i \\
    & = & 2\sqrt{2} \cos(\pi/4) + i 2\sqrt{2} \sin(\pi/4) \\
    & = & 2\sqrt{2} \lp \cos(\pi/4) + i\sin(pi/4) \rp \\
    & = & 2\sqrt{2} e^{i\pi/4}
  \end{eqnarray*}
\end{frame}

\begin{frame}
  \frametitle{Example}
  \begin{eqnarray*}
    z & = & \sqrt{3} + i 
  \end{eqnarray*}
\end{frame}

\begin{frame}
  \frametitle{Example}
  \begin{eqnarray*}
    z & = & -1
  \end{eqnarray*}
\end{frame}


\section{Root Finding}

\begin{frame}
  \frametitle{So What?}

  What is the cube root of 1?
  \begin{eqnarray*}
    \uncover<2->{1,~e^{i2\pi/3},~e^{i4\pi/3}}.
  \end{eqnarray*}

  \uncover<3->
  {
    Cuz....
    \begin{eqnarray*}
      1^3 & = & 1. \\
      \lp e^{i2\pi/3} \rp^3 & = &  e^{i2\pi} \\
      \lp e^{i4\pi/3} \rp^3 & = & e^{i4\pi}
    \end{eqnarray*}

  }

  
\end{frame}

\begin{frame}
  \frametitle{lol wut?}

  Find the cube root of 3.
  \begin{eqnarray*}
    z^3 & = & 1 \\
    z & = & r e^{i\theta} \\
    z^3 & = & r^3 e^{i3\theta} \\
    z^3 & = & e^{i0} \\
    z^3 & = & e^{i2\pi} \\
    z^3 & = & e^{i4\pi}
  \end{eqnarray*}
  
\end{frame}

\begin{frame}
  \frametitle{Root Finding}
  To find the n\textsuperscript{th} root:
  \begin{itemize}
  \item n\textsuperscript{th} root of $a+bi$
  \item Convert $a+bi=re^{i\theta}$ 
  \item Find $z^n=re^{i\theta}$
    \begin{eqnarray*}
      z^n & = & re^{i\theta} \\
      z^n & = & re^{i(\theta+2\pi)} \\
      z^n & = & re^{i(\theta+4\pi)} \\
      z^n & = & re^{i(\theta+6\pi)} \\
      \vdots & & \vdots \\
      z^n & = & re^{i(\theta+2(n-1)\pi)} \\
    \end{eqnarray*}
  \end{itemize}
\end{frame}

\begin{frame}
  If 
  \begin{eqnarray*}
    z & = & R e^{i\alpha} \\
    z^n & = & R^n e^{i n\alpha} \\
    R & = & r^{1/n} \\
    \alpha & = & \frac{\theta}{n}, ~ \frac{\theta+2\pi}{n}, ~
    \frac{\theta+4\pi}{n}, \ldots, ~ \frac{\theta+2(n-1)\pi}{n}
  \end{eqnarray*}
\end{frame}

\begin{frame}
  \frametitle{Example}

  Find the fourth root of 16.
  \begin{eqnarray*}
    z^4 & = & 16 e^{i0}, ~ 16 e^{i2\pi}, ~ 16 e^{i4\pi}, ~ 16 e^{i6\pi} \\
    R^4 e^{i4\alpha} & = & 16 e^{i0}, ~ 16 e^{i2\pi}, ~ 16 e^{i4\pi}, ~ 16 e^{i6\pi} 
  \end{eqnarray*}

  Then
  \begin{eqnarray*}
    R & = & 2 \\
    \alpha & = & 0, ~ \frac{\pi}{2}, ~ \pi, ~ \frac{3\pi}{2}
  \end{eqnarray*}

\end{frame}


% LocalWords:  Clarkson pausesection hideallsubsections
