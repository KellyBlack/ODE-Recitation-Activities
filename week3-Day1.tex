
\title{Ordinary Differential Equations}
\subtitle{Math 232 - Week 3, Day 1}

\author{Kelly Black}
\institute{Clarkson University}
\date{12 Sep 2011}

\begin{frame}
  \titlepage
\end{frame}

\begin{frame}
  \frametitle{Outline}
  \tableofcontents[pausesection,hideallsubsections]
\end{frame}


\section{Mixing Problems}


\begin{frame}
  \frametitle{Example}

  Brine is pumped into a tank at 50 liters per hour and has .5 kg per
  liter of salt. The volume of the tank is 500 liters, and initially
  the tank contains pure water. The well mixed solution is pumped out
  at 50 liters per hour.

  What is the concentration of the tank at any time?

\end{frame}


\begin{frame}
  \frametitle{Conservation of Mass}

  Mass is conserved.

  \uncover<2->{Let $A(t)$ be the amount of ``stuff'' in the tank (kg).}

  \uncover<3->{

    \begin{eqnarray*}
      \Rightarrow ~ \frac{dA}{dt} & = & \mathrm{Rate ~ it ~ changes ~
        (kg/hr)} \\
      & = & \mathrm{rate ~ in ~ - ~ rate ~ out}.
    \end{eqnarray*}

  }

\end{frame}


\begin{frame}

  \begin{eqnarray*}
    \mathrm{rate~in~kg/hr} & = & \uncover<2->{.5 \frac{kg}{l} \cdot 50 \frac{l}{hour} \\
    & = & 25 \frac{kg}{hour}}
  \end{eqnarray*}

  \uncover<3->{

    \begin{eqnarray*}
      \mathrm{rate~out~kg/hr} & = & \uncover<4->{A~kg \cdot
        \frac{1}{500~l} \cdot 50 \frac{l}{hr} \\
        & = & \frac{1}{10}A}
    \end{eqnarray*}

  }

  \uncover<5->{

    \begin{eqnarray*}
      A' & = & 25 - \frac{1}{10} A, \\
      A(0) & = & 0
    \end{eqnarray*}

  }

\end{frame}


\begin{frame}

  \begin{eqnarray*}
    \frac{1}{25 - \frac{1}{10}A} A' & = & 1, \\
    \int \frac{1}{25 - \frac{1}{10}A} A' ~ dt & = & \int 1 ~ dt, \\
    -10 \ln\lp25 - \frac{1}{10}A\rp & = & t + c \\
     \ln\lp25 - \frac{1}{10}A\rp & = & -\frac{t + c}{10} \\
     25 - \frac{1}{10} A & = & ke^{-t/10}, \\
     A & = & 250 - 10ke^{-t/10}, \\
     A(0) & = & 0 \\
     & = & 250-10k, \\
     k & = & 25, \\
     A(t) & = & 250-250e^{-t/10}
  \end{eqnarray*}

\end{frame}

\begin{frame}

  Concentration:
  \begin{eqnarray*}
    C(t) & = & A(t)/500 \\
    & = & \frac{1}{2} - \frac{1}{2} e^{-t/10}.
  \end{eqnarray*}

\end{frame}

\begin{frame}
  \frametitle{Example}

  A 2000 liter tank initially contains water with a solution of 0.2 kg
  per liter of salt. Brine is pumped in at a rate of 100 liters per
  hour with a concentration of 0.3 kg per liter. The well mixed
  solution is pumped out at a rate of 100 liters per hour. Determine
  the amount of salt in the tank at any time.

\end{frame}


\begin{frame}

  Let $A(t)$ be the amount of salt in the tank at time $t$.

  \begin{eqnarray*}
    A' & = & \mathrm{rate~in~} - \mathrm{~rate~out}.
  \end{eqnarray*}

  \begin{eqnarray*}
    \mathrm{rate~in} & = & 100 \frac{l}{hr} \cdot .3 \frac{kg}{l} \\
    & = & 30 \frac{kg}{hr}
  \end{eqnarray*}

  \begin{eqnarray*}
    \mathrm{rate~out} & = & A ~ kg \cdot \frac{1}{2000 l} \cdot 100 \frac{l}{hr} \\
    & = & \frac{1}{20} A
  \end{eqnarray*}

\end{frame}

\begin{frame}
   
  \begin{eqnarray*}
    A' & = & 30 - \frac{1}{20} A, \\
    A(0) & = & 2000 l \cdot .2 \frac{kg}{l} \\
    & = & 400 ~ kg
  \end{eqnarray*}

\end{frame}


\begin{frame}

  \begin{eqnarray*}
    A' + \frac{1}{20} A & = & 30 \\
    \uncover<2->{
      A' e^{t/20} + \frac{1}{20} e^{t/20} A & = & 30 e^{t/20} \\
      \frac{d}{dt} \lp A e^{t/20} \rp  & = & 30 e^{t/20} \\
      A e^{t/20}  & = & 600 e^{t/20} + C \\
      A  & = & 600 + C e^{-t/20} \\
      A(0) & = & 400 \\
      & = & 600 + C \\
      \Rightarrow C & = & -200 \\
      A(t) & = & 600-200 e^{-t/20}
    }
  \end{eqnarray*}

\end{frame}


\begin{frame}
  \frametitle{Example}

  A 500 liter tank initially contains fresh water. Brine with a
  concentration of 0.1 kg per liter is pumped into the tank at a rate
  of 20 liters per hour. The well mixed solution is pumped out at a
  rate of 30 liters per hour. What will the concentration be at the
  moment the tank is half full?

\end{frame}


\begin{frame}

  \begin{eqnarray*}
    A' & = & \mathrm{rate~in~} - \mathrm{~rate~out}
  \end{eqnarray*}

  \begin{eqnarray*}
    \mathrm{Rate~in} & = & .1 \frac{kg}{l} \cdot 20 \frac{l}{hr} \\
    & = & 2 \frac{kg}{l}
  \end{eqnarray*}

  \begin{eqnarray*}
    v(t) & = & 500-10t
  \end{eqnarray*}

  \begin{eqnarray*}
    \mathrm{rate~out} & = & A kg \cdot \frac{1}{500-10t ~ l} \cdot 30 \frac{l}{hr} \\
    & = & A \frac{30}{500-10t}
  \end{eqnarray*}

  \begin{eqnarray*}
    A' & = & 2 - A \frac{30}{500-10t}
  \end{eqnarray*}

\end{frame}


\begin{frame}

  \begin{eqnarray*}
    A' + A \frac{30}{500-10t} & = & 2 \\
    (500-10t)^{-3} A' + 30(500-10t)^{-4} A & = & 2 (500-10t)^{-3} \\
    \frac{d}{dt} \lp (500-10t)^{-3} A \rp & = & 2 (500-10t)^{-3} \\
    (500-10t)^{-3} A  & = & \int 2 (500-10t)^{-3} ~ dt  \\
    (500-10t)^{-3} A  & = & \frac{1}{10} (500-10t)^{-2} + C  \\
    A(0) & = & 0 \\
    & = & \frac{1}{10} 500^{-2} + C \\
    C & = & -\frac{1}{10\cdot500^{-2}}
  \end{eqnarray*}

  \begin{eqnarray*}
    A(t) & = & \frac{1}{10} (500-10t) - \frac{1}{10\cdot500^{-2}} (500-10t)^3
  \end{eqnarray*}

\end{frame}


\begin{frame}

  The tank is empty at 50 hours:
  \begin{eqnarray*}
    v(t) & = & 0 \\
    & = & 500-10t \\
    \Rightarrow t & = & 50.
  \end{eqnarray*}

  \begin{eqnarray*}
    A(25) & = & \frac{1}{10} (500-10\cdot 25) - \frac{1}{10\cdot500^{-2}} (500-10\cdot 25)^3 \\
    & \approx & 18.75 ~ kg
  \end{eqnarray*}

  \begin{eqnarray*}
    \mathrm{Concentration} & = & A(25)/250 \\
    & \approx & 0.75 kg/l
  \end{eqnarray*}

\end{frame}

\section{Newton's Law of Cooling}

\begin{frame}
  \frametitle{Newton's Law of Cooling}

  the rate of change of the temperature of an object is proportional
  to the difference in the objects temperature and the difference in
  the object's surroundings.

  Let M be the ambient temperature

  Let $T(t)$ be the temperature

  \begin{eqnarray*}
    T'(t) & = & k (M-T)
  \end{eqnarray*}


\end{frame}

\section{Examples of Newton's Law of Cooling}

\begin{frame}
  \frametitle{Example}

  A brick is pulled out of a furnace and is 350 degrees Celsius. It is
  placed outside where the temperature is 20 degrees Celsius. After
  one hour the temperature of the brick is 280 degrees Celsius. What
  will the temperature of the brick be after 6 hours?

  \uncover<2->{
    
    \begin{eqnarray*}
      T' & = & k(M-T) \\
      T(0) & = & 350 \\
      M & = & 20 \\
      T(1) & = & 280 \\
      T(6) & = & ??
    \end{eqnarray*}

  }

\end{frame}


\begin{frame}

  \begin{eqnarray*}
    T' & = & k (20-T) \\
    \frac{T'}{20-T} & = & k \\
    -\ln(20-T) & = & kt + C \\
    T & = & 20 - A e^{-kt}
  \end{eqnarray*}

  \begin{eqnarray*}
    T(0) & = & 350 \\
    & = & 20 - A \\
    A & = & -330, \\
    T(t) & = & 20 + 330 e^{-kt}
  \end{eqnarray*}

\end{frame}


\begin{frame}

  \begin{eqnarray*}
    T(1) & = & 280 \\
    & = & 20 + 330 e^{-k} \\
    k & = & -\ln\lp\frac{26}{33}\rp \\
    T(t) & = & 20 + 330 e^{t\ln\lp\frac{26}{33}\rp}
  \end{eqnarray*}

  \begin{eqnarray*}
    T(6) & = & 20 + 330 e^{6\ln\lp\frac{26}{33}\rp} \\
    & \approx & 98.9 ~C
  \end{eqnarray*}

\end{frame}


\begin{frame}
  \frametitle{Example}

  A cup of coffee is found. The temperature when it is found is 70
  degrees Celsius. Thirty minutes later it's temperature is 65 degrees
  Celsius. The temperature in the room is 18 degrees Celsius. When the
  coffee was poured the temperature was 85 degrees Celsius. When was
  the coffee poured?

  \uncover<2->{
    
    \begin{eqnarray*}
      t_0 & = & \mathrm{time~when~the~coffee~was~found.} \\
      M & = & 18 \\
      T(t_0) & = & 70 \\
      T(t_0+30) & = & 65.
    \end{eqnarray*}

  }

\end{frame}


\begin{frame}

  \begin{eqnarray*}
    T' & = & k(18-T) \\
    \frac{T'}{18-T} & = & k \\
    -\ln(18-T) & = & kt + C \\
    T & = & 18 - A e^{-kt} 
  \end{eqnarray*}

\end{frame}

\begin{frame}

  At time $t_0$:
  \begin{eqnarray*}
    T(t_0) & = & 70 \\
    & = & 18 - Ae^{-kt_0} \\
    \Rightarrow Ae^{-kt_0} & = & -52
  \end{eqnarray*}

  At time $t_0+30$:
  \begin{eqnarray*}
    A(t_0+30) & = & 65 \\
    & = & 18 - A e^{-kt_0-30k} \\
    & = & 18 - A e^{-kt_0} e^{-30k} \\
    & = & 18 + 52 e^{30k} \\
    \Rightarrow k & = & -\frac{1}{30} \ln\lp\frac{47}{52}\rp.
  \end{eqnarray*}

\end{frame}


\begin{frame}

  \begin{eqnarray*}
    T(0) & = & 85 \\
    & = & 18 - A e^{0} \\
    \Rightarrow A & = & -67
  \end{eqnarray*}

  \begin{eqnarray*}
    T(t)  & = & 18 + 67 e^{\frac{t}{30} \ln\lp\frac{47}{52}\rp} \\
    T(t_0)& = & 70 \\
    & = & 18 + 67 e^{\frac{t_0}{30} \ln\lp\frac{47}{52}\rp} \\
    \Rightarrow t_0 & = & \frac{30\ln\lp\frac{52}{67}\rp}{\ln\lp\frac{47}{52}\rp}.
  \end{eqnarray*}

\end{frame}



% LocalWords:  Clarkson pausesection hideallsubsections
