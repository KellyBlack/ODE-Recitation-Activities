\part{Span-of-Vectors}
\lecture{Span of Vectors}{Span-of-Vectors}




\title{Ordinary Differential Equations}
\subtitle{Math 232 - Week 7, Day 1}

\author{Kelly Black}
\institute{Clarkson University}
\date{10 Oct 2011}

\begin{frame}
  \titlepage
\end{frame}

\begin{frame}
  \frametitle{Outline}
  \tableofcontents[pausesection,hideallsubsections]
\end{frame}


\section{Span of a Set of Vectors}


\begin{frame}
  \frametitle{Span of a Set of Vectors}

  Any vector in $\mathbb{R}^2$ can be written as 
  \begin{eqnarray*}
    \vec{x} & = & x \vec{i} + y \vec{j} \\
    \uncover<2->
    {
      & = & \vecTwo{x}{y} 
    } \\
    \uncover<3->
    {
      & = & \vecTwo{x}{0} + \vecTwo{0}{y} 
    } \\
    \uncover<4->
    {
      & = & x \vecTwo{1}{0} + y \vecTwo{0}{1}.
    }
  \end{eqnarray*}

  $x$ and $y$ can be \textbf{any} constants.

\end{frame}


\begin{frame}
  \frametitle{Span of a Set of Vectors}

  The set of all possible vectors that can be written in the form
  \begin{eqnarray*}
    \vec{v} & = & x \vecTwo{1}{0} + y \vecTwo{0}{1}
  \end{eqnarray*}
  for any real numbers $x$ and $y$ is called the \textit{\underline{span}} of the
  vectors $\vecTwo{1}{0}$ and $\vecTwo{0}{1}$. 

  \vfill

  \uncover<2->
  {
    Any vector that I can write in the form
    \begin{eqnarray*}
      x \vecTwo{1}{0} + y \vecTwo{0}{1}
    \end{eqnarray*}
    is in this \textbf{\underline{set}} of vectors.
  }

\end{frame}


\begin{frame}
  \frametitle{Span of a Set of Vectors}

  In general, given a \textbf{\underline{set}} of vectors, $\vec{v_1}$, $\vec{v_2}$, $\vec{v_3}$,
  $\ldots$, $\vec{v_n}$ then the \textbf{\underline{set}} of vectors
  that can be written in the form
  \begin{eqnarray*}
    c_1 \vec{v_1} + c_2 \vec{v_2} + c_3 \vec{v_3} + \cdots + c_n \vec{v_n}
  \end{eqnarray*}
  is called the \textit{\underline{span}} of the vectors.

  \uncover<2->
  {
    Notation:
    \begin{eqnarray*}
      \mathrm{span}\{\vec{v_1}, \vec{v_2}, \vec{v_3},\ldots, \vec{v_n} \}
    \end{eqnarray*}
  }

\end{frame}


\begin{frame}
  \frametitle{$\mathbb{R}^2$}

  Note: any vector in $\mathbb{R}^2$ can be written as 
  \begin{eqnarray*}
    \vec{v} & = & \vecTwo{x}{y}.
  \end{eqnarray*}

  We can break this out
  \begin{eqnarray*}
    \vec{v} & = & x \vecTwo{1}{0} + y \vecTwo{0}{1}.
  \end{eqnarray*}

  So... 
  \begin{eqnarray*}
    \mathbb{R}^2 & = & \mathrm{span}\left\{\vecTwo{1}{0},\vecTwo{0}{1}\right\}
  \end{eqnarray*}
 


\end{frame}


\section{Column Space of a Matrix}

\begin{frame}
  \frametitle{Column Space of a Matrix}

  Why should I care?

  \begin{eqnarray*}
    \arrayThree{1}{2}{4}{0}{1}{3}{2}{1}{4} \vecThree{x_1}{x_2}{x_3} &
    = & 
    \vecThree{x_1+2x_2+4x_3}{0x_1+1 x_2+3x_3}{2x_1+1 x_2+4 x_3} \\
    & = & x_1 \vecThree{1}{0}{2} + x_2 \vecThree{2}{1}{1} + x_3 \vecThree{4}{3}{4}
  \end{eqnarray*}

  The matrix vector multiplication is a linear combination of the
  columns of the matrix!

\end{frame}


\begin{frame}
  \frametitle{Column Space of a Matrix}

  Definition: The \textit{Column space} of a matrix is the span of its
  column vectors.

  \uncover<2->
  {
    The column span of 
    \begin{eqnarray*}
      \arrayThree{1}{2}{4}{0}{1}{3}{2}{1}{4}
    \end{eqnarray*}
    is 
    \begin{eqnarray*}
      \mathrm{span}\left\{
        \vecThree{1}{0}{2},\vecThree{2}{1}{1},\vecThree{4}{3}{4}\right\}.
    \end{eqnarray*}
  }

\end{frame}

\section{Linear Systems}

\begin{frame}
  \frametitle{Solving Linear Systems}

  but... I still do not care.

  We want to solve
  \begin{eqnarray*}
    A \vec{x} & = & \vec{b}.
  \end{eqnarray*}
  The matrix $A$ can be thought of as a bunch of column vectors
  \begin{eqnarray*}
    A & = & \left[ \vec{v}_1~\vec{v}_2~\vec{v}_3~\cdots~\vec{v}_n \right].
  \end{eqnarray*}

  We want to find constants so that
  \begin{eqnarray*}
    x_1 \vec{v}_1 + x_2 \vec{v}_2 + x_3 \vec{v}_3 + \cdots + x_n \vec{v}_n & = & \vec{b}
  \end{eqnarray*}


\end{frame}


\begin{frame}
  \frametitle{Another Way to Pose the Question}

  Given a vector, $\vec{b}$, can I find 
  \begin{eqnarray*}
    x_1,~x_2,~x_3,~\ldots~,x_n
  \end{eqnarray*}
  so that 
  \begin{eqnarray*}
    x_1 \vec{v}_1 + x_2 \vec{v}_2 + x_3 \vec{v}_3 + \cdots + x_n \vec{v}_n & = & \vec{b}?
  \end{eqnarray*}


\end{frame}


\begin{frame}
  \frametitle{Relationship to Linear Systems}

  Meh, I still do not care.

  But this is a big question!

  Solving this:
  \begin{eqnarray*}
    x_1 \vec{v}_1 + x_2 \vec{v}_2 + x_3 \vec{v}_3 + \cdots + x_n \vec{v}_n & = & \vec{b}?
  \end{eqnarray*}

  Is the same thing as solving a system in the augmented matrix:
  \begin{eqnarray*}
    \left[ \vec{v}_1 ~ \vec{v}_2 ~ \vec{v}_3 ~ \cdots ~ \vec{v}_n \bigg| \vec{b} \right]
  \end{eqnarray*}

  Put this system into RREF and solve for the unknowns.
  

\end{frame}


\begin{frame}
  \frametitle{Example}

  Find the solution to the system of equations
  \begin{eqnarray*}
    \arrayTwo{1}{3}{2}{4} \vec{x} & = & \vec{b}.
  \end{eqnarray*}

  \uncover<2-> { Another way to state the problem: Given any vector,
    $\vec{b}$, can I find constants, $x_1$ and $x_2$, so that 
    \begin{eqnarray*}
      x_1 \vecTwo{1}{2} + x_2 \vecTwo{3}{4}  & = & \vec{b}?
    \end{eqnarray*}
  }

  
\end{frame}


\begin{frame}

  \begin{eqnarray*}
    \startRowOpsTwo
    \oneRowOpsTwo{1}{3}{b_1}{~}
    \oneRowOpsTwo{2}{4}{b_2}{~}
    \stopRowOps \\
    \uncover<2->
    {
      \stateTwo{~}{R_2-2R_1}
      \startRowOpsTwo
      \oneRowOpsTwo{1}{3}{b_1}{~}
      \oneRowOpsTwo{0}{2}{b_2-2b_1}{~}
      \stopRowOps \\
    }
    \uncover<3->
    {
      \stateTwo{~}{1/2 R_2}
      \startRowOpsTwo
      \oneRowOpsTwo{1}{3}{b_1}{~}
      \oneRowOpsTwo{0}{1}{\half \lp b_2-2b_1 \rp}{~}
      \stopRowOps \\
    }
    \uncover<4->
    {
      \stateTwo{R_1-3R_2}{~}
      \startRowOpsTwo
      \oneRowOpsTwo{1}{0}{-2 b_1 + \frac{3}{2} b_2 }{~}
      \oneRowOpsTwo{0}{1}{\half \lp b_2-2b_1 \rp}{~}
      \stopRowOps \\
    }
  \end{eqnarray*}
  
  \uncover<5->{I can find the solution no matter what the value of $\vec{b}$ is!}
    
\end{frame}


\begin{frame}
  \frametitle{Example}

  Find the solution to the system of equations
  \begin{eqnarray*}
    \arrayTwo{1}{2}{2}{4} \vec{x} & = & \vec{b}.
  \end{eqnarray*}

  \uncover<2-> { Another way to state the problem: Given any vector,
    $\vec{b}$, can I find constants, $x_1$ and $x_2$, so that 
    \begin{eqnarray*}
      x_1 \vecTwo{1}{2} + x_2 \vecTwo{2}{4}  & = & \vec{b}?
    \end{eqnarray*}
  }


  

\end{frame}

\begin{frame}
  
  \begin{eqnarray*}
    \startRowOpsTwo
    \oneRowOpsTwo{1}{2}{b_1}{~}
    \oneRowOpsTwo{2}{4}{b_2}{~}
    \stopRowOps \\
    \uncover<2->
    {
      \stateTwo{~}{R_2-2R_1}
      \startRowOpsTwo
      \oneRowOpsTwo{1}{3}{b_1}{~}
      \oneRowOpsTwo{0}{0}{b_2-2b_1}{~}
      \stopRowOps \\
    }
  \end{eqnarray*}

  \uncover<3->{I cannot find the solution except in certain circumstances!}

\end{frame}

\begin{frame}
  \frametitle{Example}

  Given any vector, $\vec{b}$, can I find constants, $x_1$, $x_2$ and
  $x_3$, so that
  \begin{eqnarray*}
    x_1 \vecTwo{1}{2} + x_2 \vecTwo{1}{1}  + x_3 \vecTwo{4}{3} & = & \vec{b}?
  \end{eqnarray*}  

\end{frame}

\begin{frame}

  \begin{eqnarray*}
    \startRowOpsThree
    \oneRowOpsThree{1}{1}{4}{b_1}{}{}
    \oneRowOpsThree{2}{1}{3}{b_2}{}{}
    \stopRowOps \\
    \uncover<2->
    {
      \stateThree{~}{R_2-2R_1}{}
      \startRowOpsThree
      \oneRowOpsTwo{1}{1}{4}{\#}{}{}
      \oneRowOpsTwo{0}{-1}{-5}{\#}{}{}
      \stopRowOps \\
    }
    \uncover<3->
    {
      \stateThree{}{-R_2}{}
      \startRowOpsThree
      \oneRowOpsTwo{1}{1}{4}{\#}{}{}
      \oneRowOpsTwo{0}{1}{5}{\#}{}{}
      \stopRowOps \\
    }
    \uncover<4->
    {
      \stateThree{R_1-R_2}{}{}
      \startRowOpsThree
      \oneRowOpsTwo{1}{0}{-1}{\#}{}{}
      \oneRowOpsTwo{0}{1}{5}{\#}{}{}
      \stopRowOps \\
    }
  \end{eqnarray*}

  \uncover<5->{We have extra information! }
   
  
\end{frame}

\begin{frame}
  \frametitle{Linearly Dependent}


  Given any vector, $\vec{b}$, can I find constants, $x_1$, $x_2$ and
  $x_3$, so that
  \begin{eqnarray*}
    x_1 \vecTwo{1}{2} + x_2 \vecTwo{1}{1}  + x_3 \vecTwo{4}{3} & = & \vec{b}?
  \end{eqnarray*}  
  

  Note that
  \begin{eqnarray*}
    \vecTwo{4}{3} & = & -\vecTwo{1}{2} + 5 \vecTwo{1}{1}.
  \end{eqnarray*}

  So $\vecTwo{4}{3}$ \textit{\underline{depends}} on $\vecTwo{1}{2}$
  and $\vecTwo{1}{1}$. We say that the vectors are ``linearly
  dependent.''

\end{frame}

\section{The Wronksian}

\begin{frame}
  \frametitle{The Wronskian}

  Any quadratic can be written as 
  \begin{eqnarray*}
    f(t) & = & at^2 + bt + c.
  \end{eqnarray*}
  The span of $\{t^2,t,1\}$, is the set of all quadratic functions.

  What about $\{t^2-1,t^2+1,t^2-2\}$? Can I write any quadratic function with these?

\end{frame}

\begin{frame}
  \frametitle{The Wronskian}

  We define the Wronskian to be the following determinant:
  \begin{eqnarray*}
    W & = & 
    \mathrm{det}
    \arrayThree{t^2-1}{t^2+1}{t^2-2}{\frac{d}{dt}\lp t^2-1\rp}{\frac{d}{dt}\lp t^2+1\rp}{\frac{d}{dt}\lp t^2-2\rp}{\frac{d^2}{dt^2}\lp t^2-1\rp}{\frac{d^2}{dt^2}\lp t^2+1\rp}{\frac{d^2}{dt^2}\lp t^2-2\rp}  \\
    \uncover<2->
    {
    & = & 
    \mathrm{det}
    \arrayThree{t^2-1}{t^2+1}{t^2-2}{2t}{2t}{2t}{2}{2}{2}  \\
    }
    \uncover<3->
    {
      & = & 0
    }
  \end{eqnarray*}

  \uncover<4->
  {
    If the Wronskian is zero then the functions are linearly dependent.
  }

\end{frame}


\begin{frame}
  \frametitle{Definition of the Wronskian}

  The Wronskian is defined to be
  \begin{eqnarray*}
    W & = & \mathrm{det}
    \left[
      \begin{array}{rrcr}
        f_1 & f_2 & \cdot & f_n \\
        f_1' & f_2' & \cdot & f_n' \\
        f_1'' & f_2'' & \cdot & f_n'' \\
        \vdots & \vdots & & \vdots \\
        f_1^{(n-1)} & f_2^{(n-1)} & \cdots & f_n^{(n-1)}
      \end{array}
    \right]
  \end{eqnarray*}

\end{frame}

\begin{frame}
  \frametitle{Example}

  Are the functions $\{t^2,t,1\}$ linearly independent?

  \begin{eqnarray*}
    W & = & \mathrm{det}
    \arrayThree{t^2}{t}{1}{2t}{1}{0}{2}{0}{0} \\
    & = & -2 \\
    & \neq & 0
  \end{eqnarray*}

\end{frame}



% LocalWords:  Clarkson pausesection hideallsubsections Meh RREF Wronksian det
% LocalWords:  Wronskian
