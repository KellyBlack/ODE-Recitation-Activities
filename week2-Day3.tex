\part{Growth-and-Decay}
\lecture{Growth and Decay}{Growth-and-Decay}
\section{Growth and Decay}

\title{Ordinary Differential Equations}
\subtitle{Math 232 - Week 2, Day 3}
\date{9 Sep 2011}

\begin{frame}
  \titlepage
\end{frame}

\begin{frame}
  \frametitle{Outline}
  \tableofcontents[pausesection,hideothersubsections]
\end{frame}


\subsection{Growth and Decay}


\begin{frame}
  \frametitle{Growth and Decay}

  \begin{quote}
    Rate of growth is proportional to the amount that is present.
  \end{quote}

  \begin{tabular}{rrcl}
    & rate of growth & $=$ & $y'$ \\
    \uncover<2->{& amount present & $=$ & $y$ } \\
    \uncover<3->{$\Rightarrow$ & There is a constant where & & $y'=ky$.}
  \end{tabular}


\end{frame}


\begin{frame}
  \frametitle{Growth, $k>-$}

  \includegraphics<1>[height=6cm]{week2GrowthSlopeField}
  \includegraphics<2>[height=6cm]{week2GrowthSlopeFieldSolutions}


\end{frame}


\begin{frame}
  \frametitle{Decay, $k<0$}

  \includegraphics<1>[height=6cm]{week2DecaySlopeField}
  \includegraphics<2>[height=6cm]{week2DecaySlopeFieldSolutions}


\end{frame}


\begin{frame}
  \frametitle{Solution}

  \begin{eqnarray*}
    y' & = & k y \\
    \frac{y'}{y} & = & k \\
    \int \frac{y'}{y} ~ dt & = & \int k ~ dt \\
    \ln(y) & = & kt + C \\
    y & = & e^{kt+C} \\
    y & = & A e^{kt}
  \end{eqnarray*}

  where $A$ is a constant.

  Solution decays if $k<0$

  Solution grows if $k>0$

\end{frame}

\subsection{Examples}

\begin{frame}
  \frametitle{Example - Radioactive Decay}

  The rate of decay of the radioactive material is proportional to the
  amount present. A sample of wood is found at the site. It contains
  13\% of what modern wood contains of C-14. The half-life of C-14 is
  5600 years. How old is the wood?

\end{frame}


\begin{frame}

  Let $Q$ be the amount of C-14 in the sample.

  \begin{eqnarray*}
    Q' & = & kQ \\
    Q & = & A e^{kt}
  \end{eqnarray*}

  \uncover<2->{In 5600 years $Q$ is cut in half.}

  \uncover<3->{Let $t=0$ be the start time:}
    \begin{eqnarray*}
      \uncover<4->{Q(t) & = & A e^{kt} }\\
      \uncover<5->{Q(0) & = & A }\\
      \uncover<6->{Q(5600) & = & \half A } \\
      \uncover<7->{Q(5600) & = & A e^{k 5600} }\\
      \uncover<8->{\half & = & e^{k 5600} }\\
      \uncover<9->{\ln\lp\half\rp & = & k 5600 }\\
      \uncover<10->{k & = & \frac{\ln\lp\half\rp}{5600} }\\
  \end{eqnarray*}

\end{frame}


\begin{frame}

  \begin{eqnarray*}
    Q(t) & = & A e^{\frac{\ln\lp\half\rp}{5600}t} \\
    \uncover<2->{Q(T) & = & 0.15 A } \\
    \uncover<3->{.15A & = & A e^{\frac{\ln\lp\half\rp}{5600}T} }\\
    \uncover<4->{.15 & = & e^{\frac{\ln\lp\half\rp}{5600}T} }\\
    \uncover<4->{\ln(.15) & = & \frac{\ln\lp\half\rp}{5600}T }\\
    \uncover<4->{T & = & \frac{\ln(.15) 5600}{\ln\lp\half\rp} ~ \mathrm{years}}
  \end{eqnarray*}

\end{frame}


\begin{frame}
  \frametitle{Continual Interest}

  The rate of change in a bank account is proportional to the amount
  of money present.

  Let $A(t)$ be the amount of money.

  \begin{eqnarray*}
    A' & = & r A \\
    A(0) & = & A_0 \\
    A(t) & = & k e^{rt} \\
    A_0 & = & k e^0 \\
    k  & = & A_0 \\
    A(t) & = & A_0 e^{rt}
  \end{eqnarray*}

\end{frame}


\begin{frame}
  \frametitle{Example}
  A bank promises 5\% interest compounded annually. How long till it
  takes to double your money?

  \begin{eqnarray*}
    A(t) & = & A_0 e^{rt} 
  \end{eqnarray*}

  \uncover<2->{When does $A(t)=2A_0$?}

  \begin{eqnarray*}
    \uncover<3->{
      2A_0 & = & A_0 e^{0.05T}, \\
      2 & = & e^{0.05T} \\
      \ln(2) & = & .05T \\
      T & = & \frac{\ln(2)}{.05}
    }
  \end{eqnarray*}


\end{frame}


\begin{frame}
  \frametitle{Light Attenuation}

  Light shining in water is attenuated at a rate proportional to its
  intensity.

  $L$ is the intensity of light (in Lumens).

  \begin{eqnarray*}
    L' & = & k L \\
    L & = & A e^{kd}
  \end{eqnarray*}


\end{frame}

\begin{frame}
  \frametitle{Example}
  
  A diver measures the light intensity at $d=10m$ and finds 1800
  Lumens.

  A diver measures the light intensity at $d=20m$ and finds 1200
  Lumens.

  What is the rate of decay?

  \uncover<2->
  {
    We have two data points, one at 10m:
    \begin{eqnarray*}
      L(10) & = & 1800 \\
      & = & A e^{10k} 
    \end{eqnarray*}


    The other at 20m:
    \begin{eqnarray*}
      L(20) & = & 1200 \\
      & = & A e^{20k}
    \end{eqnarray*}

  }

\end{frame}

\begin{frame}
  \frametitle{Example}

  \begin{eqnarray*}
    1800 e^{-10k} & = & 1200 e^{-20k} \\
    e^{10k} & = & \frac{2}{3} \\
    10k & = & \ln(2/3) \\
    k & = & \frac{1}{10} \ln(2/3)
  \end{eqnarray*}

\end{frame}


% LocalWords:  Clarkson pausesection hideothersubsections
