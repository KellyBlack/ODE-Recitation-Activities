
\preClass{Linear Algebra}

\begin{problem} 
\item Multiply the following matrices ($A*B$ and $B*A$):
  \begin{eqnarray*}
    A & = & 
    \left[
      \begin{array}{rr}
        3 & 4 \\ 
        1 & 1
      \end{array}
    \right] \\
    B & = & 
    \left[
      \begin{array}{rr}
        -1 & 4 \\
         1 & -3
      \end{array}
    \right] \\
  \end{eqnarray*}

  \iftoggle{solutions}{%

    \begin{eqnarray*}
      \left[
        \begin{array}{rr}
          3 & 4 \\ 
          1 & 1
        \end{array}
      \right]*
    \left[
      \begin{array}{rr}
        -1 & 4 \\
         1 & -3
      \end{array}
    \right]  & = & 
    \left[
      \begin{array}{rr}
        1 & 0 \\
        0 & 1
      \end{array}
    \right] \\
    \left[
      \begin{array}{rr}
        -1 & 4 \\
         1 & -3
      \end{array}
    \right] *
    \left[
      \begin{array}{rr}
        3 & 4 \\ 
        1 & 1
      \end{array}
    \right]
    & = & 
    \left[
      \begin{array}{rr}
        1 & 0 \\
        0 & 1
      \end{array}
    \right] \\
    \end{eqnarray*}
    

  }{%
    \vspace{4em}
  }

  \item Solve the following system of equations using matrix techniques:
  \begin{eqnarray*}
    \left[
      \begin{array}{rr}
        3 & 4 \\ 
        1 & 1
      \end{array}
    \right] 
    \vec{x} & = & 
    \left[
      \begin{array}{r}
        1 \\
        0
      \end{array}
    \right]
  \end{eqnarray*}

  \iftoggle{solutions}{%

    \begin{eqnarray*}
      \left[
        \begin{array}{rr}
          -1 & 4 \\
          1 & -3
        \end{array}
      \right] *
      \left[
        \begin{array}{rr}
          3 & 4 \\ 
          1 & 1
        \end{array}
      \right] * \vec{x} & = & 
      \left[
        \begin{array}{rr}
          -1 & 4 \\
           1 & -3
         \end{array}
       \right] *
       \left[
         \begin{array}{r}
           0 \\
           1
         \end{array}
       \right] \\
       \vec{x} & = & 
       \left[
         \begin{array}{r}
            4 \\
           -3
         \end{array}
       \right]
     \end{eqnarray*}

  }

  \vfill

  \item Solve the following system of equations using matrix techniques:
  \begin{eqnarray*}
    \left[
      \begin{array}{rr}
        3 & 4 \\ 
        1 & 1
      \end{array}
    \right] 
    \vec{x} & = & 
    \left[
      \begin{array}{r}
        0 \\
        1
      \end{array}
    \right]
  \end{eqnarray*}

  \iftoggle{solutions}{%

    \begin{eqnarray*}
      \left[
        \begin{array}{rr}
          -1 & 4 \\
           1 & -3
         \end{array}
       \right] *
       \left[
         \begin{array}{rr}
           3 & 4 \\ 
           1 & 1
         \end{array}
       \right] * \vec{x} & = & 
       \left[
         \begin{array}{rr}
           -1 & 4 \\
           1 & -3
         \end{array}
       \right] *
       \left[
         \begin{array}{r}
           1 \\
           0
         \end{array}
       \right] \\
       \vec{x} & = & 
       \left[
         \begin{array}{r}
           -1 \\
           1
         \end{array}
       \right]
     \end{eqnarray*}

  }

  \vfill

\end{problem}


  \actTitle{Inverse of a matrix}
  \begin{problem}
  \item Given the following matrix
    \begin{eqnarray*}
      A & = & \arrayTwo{1}{2}{1}{-4}
    \end{eqnarray*}
    \begin{subproblem}
      \item Does the inverse exist? (Determine the answer without
        finding the inverse.)

        \iftoggle{solutions}{%

          The determinant of the matrix is
          \begin{eqnarray*}
            \mathrm{det}\arrayTwo{1}{2}{1}{-4} & = & -6
          \end{eqnarray*}
          so the matrix is invertible since it is not zero.

        }

        \vfill
      \item Is there a unique solution to the equation
        \begin{eqnarray*}
          A \vec{x} & = & \vec{b}
        \end{eqnarray*}
        for any vector $\vec{b}$?

        \iftoggle{solutions}{%
          Yes, since the inverse of the matrix exists.
        }

        \vfill
      \item Are the columns of the matrix linearly independent?

        \iftoggle{solutions}{%

          Yes, because the system can be written as
          \begin{eqnarray*}
            x_1 \vecTwo{1}{1} + x_2 \vecTwo{2}{-4} & = & \vecTwo{0}{0}, \\
            \arrayTwo{1}{2}{1}{-4} \vecTwo{x_1}{x_2} & = & \vecTwo{0}{0}.
          \end{eqnarray*}
          The only possible solution is $x_1=0$ and $x_2=0$ since the
          solution is unique.

        }

        \vfill
    \end{subproblem}

    \clearpage
  \item Find the inverse of the matrix.

    \iftoggle{solutions}{%

      \begin{eqnarray*}
        \startRowOpsTwo
        \oneRowOpsTwo{1}{2}{1}{0}
        \oneRowOpsTwo{1}{-4}{0}{1}
        \stopRowOps  \\
        \stateTwo{~}{-R_1+R_2}
        \startRowOpsTwo
        \oneRowOpsTwo{1}{2}{1}{0}
        \oneRowOpsTwo{0}{-6}{-1}{1}
        \stopRowOps  \\
        \stateTwo{~}{R_2/6}
        \startRowOpsTwo
        \oneRowOpsTwo{1}{2}{1}{0}
        \oneRowOpsTwo{0}{1}{1/6}{-1/6}
        \stopRowOps  \\
        \stateTwo{R_1-2R_1}{~}
        \startRowOpsTwo
        \oneRowOpsTwo{1}{0}{2/3}{1/3}
        \oneRowOpsTwo{0}{1}{1/6}{-1/6}
        \stopRowOps 
      \end{eqnarray*}
      \begin{eqnarray*}
        A^{-1} & = & \arrayTwo{2/3}{1/3}{1/6}{-1/6} 
      \end{eqnarray*}

    }

        \vfill
        
  \end{problem}


  \actTitle{Linear Dependence}
  \begin{problem}

    \item Determine whether or not the set of functions whose first
      derivative is zero at $x=0$ is a vector space.

      \iftoggle{solutions}{%

        Let $V$ be the set of functions that satisfy $f'(0)=0$.

        Addition: suppose that $f$ and $g$ are in $V$. Is $f+g$ also in V?
        \begin{eqnarray*}
          \deriv{~}{t} (f+g) & = & f'+g', \\
          f'(0)+g'(0) & = & 0,
        \end{eqnarray*}
        so $f+g$ are in $V$.

        Scalar multiplication: for any real number, $c$, is $c\cdot f(t)$ in $V$?
        \begin{eqnarray*}
          \deriv{~}{t} (c\cdot f(t)) & = & c \cdot f'(t), \\
          c \cdot f(0) & = & 0,
        \end{eqnarray*}
        so $c \cdot f(t)$ is in $V$.

        The derivative of the function $f(t)=0$ is zero so it is in $V$.

        For any function that satisfies $f'(0)=0$ then $-f'(0)=0$ so $-f(t)$ is in $V$.

        It is a vector space.

      }

      \vfill

    \item Determine whether or not the set of functions that are equal
      to one at $x=0$ is a vector space.

      \iftoggle{solutions}{%

        Let $V$ be the set of functions that satisfy $f(0)=1$. It is
        not a vector space. If $f(t)$ and $g(t)$ are in the space then
        $f(0)+g(0)=2$ so $f+g$ is not in the set.

      }

      \vfill

      \clearpage

  \item Are the following vectors linearly independent? If not find a
    way to express one of the vectors in terms of the others

    \begin{eqnarray*}
      \vecFour{1}{2}{-1}{-2},~\vecFour{3}{6}{-3}{-6},~\vecFour{-4}{-10}{5}{11},
      ~\vecFour{2}{7}{-3}{-9}
    \end{eqnarray*}

    \iftoggle{solutions}{%
      
      First put it all in a matrix,
      \begin{eqnarray*}
        \startRowFour
        \oneRowFour{1}{3}{-4}{2}
        \oneRowFour{2}{6}{-10}{7}
        \oneRowFour{-1}{-3}{5}{-3}
        \oneRowFour{-2}{-6}{11}{-9}
        \stopRowOps .
      \end{eqnarray*}
      When you put this in RREF you get
      \begin{eqnarray*}
        \startRowFour
        \oneRowFour{1}{3}{0}{0}
        \oneRowFour{0}{0}{1}{0}
        \oneRowFour{0}{0}{0}{1}
        \oneRowFour{0}{0}{0}{0}
        \stopRowOps .
      \end{eqnarray*}
      This means that the second vector is three times the first vector.

    }

  \vfill

\end{problem}
