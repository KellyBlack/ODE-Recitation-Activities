
\preClass{Linear Algebra}

\begin{problem}
\item  Find the eigen values and eigen vectors for the following matrices:

  \begin{subproblem}
  \item
    \begin{eqnarray}
      \left[
        \begin{array}{rr}
          12 & -1 \\
          4 & 7
        \end{array}
      \right].
    \end{eqnarray}
    \vfill

  \item 
    \begin{eqnarray}
      \left[
        \begin{array}{rr}
          4 & 6 \\
          2 & 3
        \end{array}
      \right].
    \end{eqnarray}
    \vfill
      
  \end{subproblem}
\end{problem}

\actTitle{Linear Algebra}
\begin{problem}
\item Express the following second order differential equation as a
  system of two equations:
  \begin{eqnarray}
    x'' - 3x & = & 0, \\
    x(0) & = & 1.0, \\
    x'(0) & = & 0.0
  \end{eqnarray}
  Find the eigen vectors and eigen values for the system. Compare the
  eigen values to the roots of the characteristic equation of the
  original equation. Compare the solution to the system to the
  solution to the original equation.

  \vfill



  \clearpage

\item Express the following second order differential equation as a
  system of two equations:
  \begin{eqnarray}
    x'' - 2x' - 3x & = & 0, \\
    x(0) & = & -2.0, \\
    x'(0) & = & 1.0
  \end{eqnarray}
    Find the eigen vectors and eigen values for the system. Compare
    the eigen values to the roots of the characteristic equation of
    the original equation. Compare the solution to the system to the
    solution to the original equation.

    \vfill


  \end{problem}


  \actTitle{Linear Algebra}
  \begin{problem}

  \item Two tanks are arranged so that they exchange their
    contents. The first tank initially contains 300 liters of fresh
    water. Water that contains 5 g/liter of a chemical is pumped in at
    a rate of 4 liters/minute. The well mixed solution is pumped into
    the second tank at a rate of 6 liters per minute.

    The second tank initially contains 200 liters of fresh water. The
    well mixed solution is pumped into the first tank at a rate of 6
    liters per minute. Additionally, the solution is pumped out at a
    rate of 4 liters/minute.


    \begin{subproblem}
      \item Draw a picture. Then determine and organize the given
        information. Define and label the variables that represent the
        amount of chemical in each tank.
        \vfill

      \item Determine the differential equation for the first tank.
        \vfill

      \item Determine the differential equation for the second tank.
        \vfill

      \item Write the equations as a system of differential equations.
        \vfill

    \end{subproblem}

    \clearpage

  \item For each of the following systems of differential equations
    sketch the solution trajectories in the phase plane for each
    equation.
    
    \begin{subproblem}
      
    \item 
        \begin{eqnarray}
          \frac{d}{dt} \vec{x} & = & 
          \left[ \begin{array}{rr}
               8 &  6 \\
              -9 & -7
            \end{array} \right] \vec{x}.
        \end{eqnarray}

        \vfill
        \clearpage


      \item 
        \begin{eqnarray}
          \frac{d}{dt} \vec{x} & = & 
          \left[ \begin{array}{rr}
              10 &  -6 \\
              18 & -11
            \end{array} \right] \vec{x}.
        \end{eqnarray}

        \vfill
        \clearpage

    \item 
        \begin{eqnarray}
          \frac{d}{dt} \vec{x} & = & 
          \left[ \begin{array}{rr}
               -6 & -6 \\
                1 & -1
            \end{array} \right] \vec{x}.
        \end{eqnarray}

        \vfill
        \clearpage


      \item 
        \begin{eqnarray}
          \frac{d}{dt} \vec{x} & = & 
          \left[ \begin{array}{rr}
              -5 &  -12 \\
               4 &    9
            \end{array} \right] \vec{x}.
        \end{eqnarray}

        \vfill


    \end{subproblem}


\end{problem}

