

\preClass{Linear Theory}

\begin{problem}
\item Find the solution to the following differential equation
  \begin{eqnarray}
    \label{eqn:nonhomogeneousPreClass}
    y' + ty & = & e^{-t}.
  \end{eqnarray}

  \begin{subproblem}
  \item First find a solution to the homogeneous equation
  \begin{eqnarray*}
    y_h' + ty_h & = & 0.
  \end{eqnarray*}
    \vfill

  \item Assume that the solution to equation
    \ref{eqn:nonhomogeneousPreClass} is in the form $y(t) = u(t)
    y_h(t)$ where $y_h(t)$ is your solution to the homogenous
    equation. Take the derivative of $y(t)$. (Use the product rule!)
    \vspace{8em}

    (See the next page.)
    \clearpage

  \item Substitute the function $y(t)$ into the original differential
    equation and solve for $u(t)$.
    \vfill

  \end{subproblem}
\end{problem}


  \actTitle{Growth and Decay}
  \begin{problem}
  \item Find the solution to the following differential equation:
    \begin{eqnarray*}
      A' & = & -0.1 A, \\
      A(0) & = & 1000.
    \end{eqnarray*}
    What is the long term solution?
    \vfill
    \clearpage
  \item The number of bacteria in a colony at 10:00am is approximately
    five million. At noon the number is approximately seven
    million. How many bacteria were in the colony at 9:00am?
    \vfill
  \end{problem}


  \actTitle{Mixing}
  \begin{problem}

  \item A tank initially contains 2,000 litres of water with a
    concentration of mercury of $6.0\times 10^{-5}$ grams per
    litre. Water that contains $3.0\times 10^{-5}$ grams per litre is
    pumped into the tank at 100 litres per hour. The well mixed
    solution is pumped out of the tank at 100 litres per hour.  How
    long will it take for the concentration in the tank to reach
    $4.5\times 10^{-5}$ grams per litre?

    \begin{subproblem}
      \item Draw a picture. Label and define the important quantities.
        \vfill

      \item Define the quantity to use. Determine the rate that
        mercury moves into the tank, and determine the rate that
        mercury moves out of the tank. Define the initial condition.
        \vfill

        \clearpage

      \item Write out the differential equation and solve it to find
        the amount of mercury in the tank at any time. Once you find a
        formula for the amount of mercury determine the formula for
        the concentration at any time.

        \vfill
        

    \end{subproblem}



\end{problem}
