
\preClass{Second order, constant coefficient equations.}

\begin{problem}
\item The following questions refer to the  differential
  equation
  \begin{eqnarray}
    y'' - y' - 12y = 0.
  \end{eqnarray}

  \begin{subproblem}
  \item Show that $y=e^{4t}$ is a solution.

      \iftoggle{solutions}{%

        \begin{eqnarray*}
          y & = & e^{4t}, \\
          y' & = & 4 e^{4t}, \\
          y'' & = & 16 e^{4t}, \\
          \Rightarrow y'' - y' - 12y & = & 16 e^{4t} - 4 e^{4t} - 12 e^{4t}, \\
          & = & 0.
        \end{eqnarray*}
        
      }{
        \vspace{4cm}
      }
  \item Show that $y=e^{-3t}$ is a solution.

      \iftoggle{solutions}{%

        \begin{eqnarray*}
          y & = & e^{-3t}, \\
          y' & = & -3 e^{-3t}, \\
          y'' & = & 9 e^{-3t}, \\
          \Rightarrow y'' - y' - 12y & = & 9 e^{-3t} + 3 e^{-3t} - 12 e^{-3t}, \\
          & = & 0.
        \end{eqnarray*}
        
      }{
        \vspace{4cm}
      }

  \item Show that $y=C_1 e^{4t} + C_2 e^{-3t}$ is a solution for
    any constants $C_1$ and $C_2$. Will the solution grow or
    decay?

      \iftoggle{solutions}{%

        \begin{eqnarray*}
          y & = & C_1 e^{4t} + C_2 e^{-3t}, \\
          y' & = & C_1 4 e^{4t} - C_2 3 e^{-3t}, \\
          y'' & = & C_1 16 e^{4t} + C_2 9 e^{-3t}, \\
          \Rightarrow y'' - y' - 12y & = & C_1 16 e^{4t} + C_2 9 e^{-3t} 
                                         - C_1 4 e^{4t} + C_2 3 e^{-3t} 
                                         - C_1 12 e^{4t} + 12 C_2 e^{-3t}, \\
          & = & 0.
        \end{eqnarray*}
        
      }{
        \vfill
      }

  \end{subproblem}

\end{problem}


  \actTitle{Second Order Equations}
  \begin{problem}

  \item The following questions refer to the  differential
    equation
    \begin{eqnarray}
      y'' + 3 y' + 2 y = 0.
    \end{eqnarray}
    \begin{subproblem}
    \item Assume that the solution is in the form $y=Ae^{rt}$. Find
      the first and second derivatives of this function.

      \iftoggle{solutions}{%

        \begin{eqnarray*}
          y & = & A e^{rt}, \\
          y' & = & A r e^{rt}, \\
          y'' & = & A r^2 e^{rt}, \\
        \end{eqnarray*}
        
      }{
        \vfill
      }


    \item Substitute the previous results into the differential equation.
      
      \iftoggle{solutions}{%

        \begin{eqnarray*}
          y'' + 3 y' + 2y & = & A r^2 e^{rt} + 3 A r e^{rt} + 2 A e^{rt}.
        \end{eqnarray*}
        
      }{
        \vfill
      }

      
    \item Factor the $Ae^{rt}$ and solve for $r$.

      \iftoggle{solutions}{%

        \begin{eqnarray*}
          A r^2 e^{rt} + 3 A r e^{rt} + 2 A e^{rt} 
          & = & A e^{rt} \lp r^2 + 3r + 2 \rp.
        \end{eqnarray*}
        
      }{
        \vfill
      }

    \item Determine the general solution to the differential equation.

      \iftoggle{solutions}{%

        \begin{eqnarray*}
          r^2 + 3r + 2 & = & 0, \\
          r & = & \frac{-3\pm\sqrt{9-8}}{2}, \\
          & = & -2, -1, \\
          \Rightarrow y & = & C_1 e^{-t} + C_2 e^{-2t}.
        \end{eqnarray*}
        
      }{
        \vfill
      }

    
    \end{subproblem}


 
    \clearpage


  \item Find the solutions to the following differential equation:
    \begin{eqnarray}
      y'' + 3y' - 10y = 0.
    \end{eqnarray}
    Determine if the solution will exhibit growth or decay. Can you
    determine the answer to the question of stability by only looking
    at the roots of the characteristic equation? If so how?

      \iftoggle{solutions}{%
        Assume that $y=Ae^{rt}$, then substitute back into the equation:
        \begin{eqnarray*}
          y'' + 3y' - 10y & = & 0, \\
          A e^{rt} \lp r^2 + 3r - 10 \rp & = & 0, \\
          \Rightarrow \lp r^2 + 3r - 10 \rp & = & 0, \\
          r & = & \frac{-3 \pm \sqrt{9+40}}{2}, \\
          r & = & 2,~ -5, \\
          \Rightarrow y & = & C_1 e^{2t} + C_2 e^{-5t}.
        \end{eqnarray*}

        The solution will grow like $e^{2t}$.

        The roots of the characteristic equation indicate the
        coefficient in the exponential term. If both roots are
        negative the solution will decay. Otherwise at least one root
        will be positive, and the solution will grow
        exponentially. Here the roots are 2 and -5, so the solution
        will grow like $e^{2t}$.
        
      }

      \vfill

  \clearpage


  \item Find the solutions to the following differential equation:
    \begin{eqnarray}
      y'' + 4y' + 2y = 0.
    \end{eqnarray}
    Determine if the solution will exhibit growth or decay. Can you
    determine the answer to the question of stability by only looking
    at the roots of the characteristic equation? If so how?

      \iftoggle{solutions}{%

        Assume that $y=Ae^{rt}$, then substitute back into the equation:
        \begin{eqnarray*}
          y'' + 4y' + 2y & = & 0, \\
          A e^{rt} \lp r^2 + 4r + 2 \rp & = & 0, \\
          \Rightarrow \lp r^2 + 4r + 2 \rp & = & 0, \\
          r & = & \frac{-4 \pm \sqrt{16-8}}{2}, \\
          r & = & -2+\sqrt{2}, -2-\sqrt{2}, \\
          \Rightarrow y & = & C_1 e^{\lp -2+\sqrt{2}\rp t} + C_2 e^{\lp -2-\sqrt{2}\rp t}.
        \end{eqnarray*}

        The solution will decay like $e^{\lp -2+\sqrt{2}\rp t}$.

        The roots of the characteristic equation indicate the
        coefficient in the exponential term. If both roots are
        negative the solution will decay. Otherwise at least one root
        will be positive, and the solution will grow exponentially. In
        this case both roots are negative so the solution decays.

        
      }{
        \vfill
      }

  \end{problem}


  \actTitle{Second Order Equations}
  \begin{problem}

  \item The questions refer to the following differential equation:
    \begin{eqnarray}
      y'' + 4y & = & 0.
    \end{eqnarray}

    \begin{subproblem}
      \item Show that $y=C_1 e^{i2t} + C_2 e^{-i2t}$ is a solution to
        the differential equation.

      \iftoggle{solutions}{%

        Assume that $y=C_1 e^{i2t} + C_2 e^{-i2t}$, and the derivatives are
        \begin{eqnarray*}
          y  & = & C_1 e^{i2t} + C_2 e^{-i2t}, \\
          y' & = & 2i C_1 e^{i2t} - 2i C_2 e^{-i2t}, \\
          y''& = & -4 C_1 e^{i2t} - 4 C_2 e^{-i2t}.
        \end{eqnarray*}
        Substitute this back into the original equation to get
        \begin{eqnarray*}
          y'' + 4y & = & -4 C_1 e^{i2t} - 4 C_2 e^{-i2t} + 4 C_1 e^{i2t} + 4 C_2 e^{-i2t}, \\
          & = & 0.
        \end{eqnarray*}

        
      }{
        \vfill
      }


      \item Use Euler's formula to write the solution, $y=C_1 e^{i2t}
        + C_2 e^{-i2t}$, in terms of sines and cosines.

        \iftoggle{solutions}{%
        
          Assume that $y=C_1 e^{i2t} + C_2 e^{-i2t}$ and substitute the
          exponential terms using Euler's formula:
          \begin{eqnarray*}
            y  & = & C_1 e^{i2t} + C_2 e^{-i2t}, \\
            & = & C_1 \lp \cos(2t) + i \sin(2t) \rp +
                  C_2 \lp \cos(-2t) + i \sin(-2t) \rp.
          \end{eqnarray*}

        }{
          \vfill
        }


        \clearpage

      \item Simplify the terms so that the sine and cosine terms are
        only expressed in terms of $\sin(2t)$ and $\cos(2t)$.

        \iftoggle{solutions}{%

          From the previous page we have that
          \begin{eqnarray*}
            y  & = & C_1 e^{i2t} + C_2 e^{-i2t}, \\
            & = & C_1 \lp \cos(2t) + i \sin(2t) \rp +
                  C_2 \lp \cos(-2t) + i \sin(-2t) \rp, \\
            & = & C_1 \lp \cos(2t) + i \sin(2t) \rp +
                  C_2 \lp \cos(2t) - i \sin(2t) \rp. \\
          \end{eqnarray*}

        }{
          \vfill
        }

      \item Collect all the sine terms and the cosine terms so that
        the solution is in the form
        \begin{eqnarray}
          y & = & \#_1 \cos(2t) + \#_2 \sin(2t).
        \end{eqnarray}

        \iftoggle{solutions}{%

          From the previous step we have that
          \begin{eqnarray*}
            C_1 \lp \cos(2t) + i \sin(2t) \rp +
            C_2 \lp \cos(2t) - i \sin(2t) \rp 
            & = & \lp C_1 + C_2 \rp \cos(2t) + i (C_1 - C_2) \sin(2t).
          \end{eqnarray*}

        }{
          \vfill
        }

      \item Show that $y=k_1 \cos(2t) + k_2 \sin(2t)$ is a solution to
        the original differential equation.
        \iftoggle{solutions}{%

          Assume that $y=k_1 \cos(2t) + k_2 \sin(2t)$ and the derivatives are
          \begin{eqnarray*}
            y   & = & k_1 \cos(2t) + k_2 \sin(2t), \\
            y'  & = & -2 k_1 \sin(2t) + 2 k_2 \cos(2t), \\
            y'' & = & -4 k_1 \cos(2t) - 4 k_2 \sin(2t). \\
          \end{eqnarray*}

          Substitute this back into the original equation and
          simplify. The result is the following:
          \begin{eqnarray*}
            y'' + 4y & = & k_1 \cos(2t) + k_2 \sin(2t)
            -4 k_1 \cos(2t) - 4 k_2 \sin(2t), \\
            & = & 0.
          \end{eqnarray*}

        }{
          \vfill
        }

        

    \end{subproblem}


\end{problem}
