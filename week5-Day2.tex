\part{Linear-Algebra}
\lecture{Linear Algebra}{Linear-Algebra}
\section{Linear Algebra}

\title{Ordinary Differential Equations}
\subtitle{Math 232 - Week 5, Day 2}
\date{28 Sep 2011}

\begin{frame}
  \titlepage
\end{frame}

\begin{frame}
  \frametitle{Outline}
  \tableofcontents[pausesection,hideothersubsections]
\end{frame}


\subsection{Linear Algebra}


\begin{frame}
  \frametitle{Linear Algebra}


  \begin{eqnarray*}
    3x + 2y & = & 7 \\
    4x - y & = & 1
  \end{eqnarray*}

  \uncover<2->
  {
    \begin{eqnarray*}
      \arrayTwo{3}{2}{4}{-1} \vecTwo{x}{y} & = & \vecTwo{7}{1}
    \end{eqnarray*}

    A matrix is an array of numbers arranged in rows and columns.

    A column vector is a matrix with one column.

  }


\end{frame}


\begin{frame}
  \frametitle{Matrix Addition/Subtraction}

  If two matrices have the same number of rows and columns you add the
  two matrices by adding/subtracting them entry by entry.

  \begin{eqnarray*}
    \arrayTwo{3}{-1}{2}{4} + \arrayTwo{7}{6}{2}{1} & = & \arrayTwo{10}{5}{4}{5}
  \end{eqnarray*}

\end{frame}


\begin{frame}
  \frametitle{Scalar Multiplication}

  Multiply every entry in the matrix by the same number.

  \begin{eqnarray*}
    4 \arrayTwo{3}{-1}{2}{4} & = & \arrayTwo{12}{-4}{8}{16}
  \end{eqnarray*}


\end{frame}


\begin{frame}
  \frametitle{Transpose}

  Switch the rows and the columns:
  \begin{eqnarray*}
    \arrayTwo{2}{7}{4}{6}^T & = & \arrayTwo{2}{4}{7}{6} \\
    \left[
    \begin{array}{r}
      1 \\ 7 \\ 3 \\ 2
    \end{array}
    \right]^T & = & 
    \left[
    \begin{array}{rrrr}
      1 & 7 & 3 & 2
    \end{array}
    \right]
  \end{eqnarray*}

\end{frame}


\begin{frame}
  \frametitle{Matrix Multiplication}

  \begin{eqnarray*}
    \left[
    \begin{array}{rrrr}
      a_{11} & a_{12} & \cdots & a_{1m} \\
      a_{21} & a_{22} & \cdots & a_{2m} \\
      \vdots &       & \ddots & \vdots \\ \hline
      a_{i1} & a_{i2} & \cdots & a_{im} \\ \hline
      \vdots &       & \ddots & \vdots \\
      a_{n1} & a_{n2} & \cdots & a_{nm}
    \end{array}
  \right] \cdot
    \left[
    \begin{array}{rr|r|rr}
      b_{11} & \cdots & b_{1j} & \cdots  & b_{1k} \\
      b_{21} & \cdots & b_{2j} & \cdots  & b_{2k} \\
      \vdots &        & \vdots & \vdots & \vdots \\
      b_{n1} & \cdots & b_{mj}  & \cdots & b_{mk}
    \end{array}
  \right]  =  \\
    \left[
    \begin{array}{rrrrr}
      * & \cdots & * & \cdots  &  * \\
      *  & \cdots & *  & \cdots  &  * \\
      \vdots &        & c_{ij} & \vdots & \vdots \\
      * & \cdots & *  & \cdots & *
    \end{array}
  \right]
  \end{eqnarray*}

  Entry in row i and column j is 
  \begin{eqnarray*}
    c_{ij} & = & a_{i1}b_{1j} + a_{i2}b_{2j} + \cdots + a_{im}b_{mj}
  \end{eqnarray*}

\end{frame}


\begin{frame}
  \frametitle{Example}

  \begin{eqnarray*}
    \arrayTwo{2}{7}{4}{6} \cdot \arrayTwo{2}{1}{-2}{1} & = & 
    \arrayTwo{-10}{9}{-4}{10} \\
    \left[
      \begin{array}{rrr}
        2 & 3 & 1 \\ -1 & 2 & 1
      \end{array}
    \right] \cdot
    \left[
      \begin{array}{r}
        1 \\ 2 \\ 3
      \end{array}
    \right]
    & = & 
    \vecTwo{11}{6}
  \end{eqnarray*}

\end{frame}

\subsection{Vector Operations}

\begin{frame}
  \frametitle{Vector Operations}

  If
  \begin{eqnarray*}
    \vec{x} & = & 
    \left[
    \begin{array}{r}
      x_1 \\ x_2 \\ \vdots \\ x_n
    \end{array}
    \right]
  \end{eqnarray*}
  then
  \begin{eqnarray*}
    \left[
      \begin{array}{rrrr}
        x_1 & x_2 & \cdot & x_n
      \end{array}
    \right] \cdot
    \left[
      \begin{array}{r}
        x_1 \\ x_2 \\ \vdots \\ x_n
      \end{array}
    \right] & = & 
    x_1^2 + x_2^2 + \cdots x_n^2
  \end{eqnarray*}

\end{frame}


\begin{frame}
  \frametitle{The Dot Product}

  Definition:
  \begin{eqnarray*}
    \vec{x} \cdot \vec{y} & = & \vec{x}^T \vec{y} 
  \end{eqnarray*}

  Definition
  \begin{eqnarray*}
    \| \vec{x} \| & = & \sqrt{\vec{x}^T \vec{x}}
  \end{eqnarray*}

\end{frame}


\begin{frame}
  \frametitle{Example}

  \begin{eqnarray*}
    \vec{x} & = & 
    \left[
      \begin{array}{r}
        2 \\ 4 \\ -1
      \end{array} \right] \\
      \| \vec{x} \| & = & \sqrt{2^2 + 4^2 + (-1)^2} \\
      & = & \sqrt{21}
  \end{eqnarray*}

\end{frame}

\subsection{Matrices to Know}

\begin{frame}
  \frametitle{Matrices to Know}
  
  A diagonal matrix is in the form
  \begin{eqnarray*}
    \left[
      \begin{array}{rrrr}
        a_{11} & 0 & \\
        0 & a_{22} & 0 \\
        & & \ddots & \\
        & & 0 & a_{nn}
      \end{array}
    \right]
  \end{eqnarray*}

  Special case, the identity matrix:
  \begin{eqnarray*}
    I_n & = & 
    \left[
      \begin{array}{rrrr}
        1 & 0 & \\
        0 & 1 & 0 \\
        & & \ddots & \\
        & & 0 & 1
      \end{array}
    \right]
  \end{eqnarray*}

\end{frame}


\begin{frame}
  \frametitle{Example}
  
  Special case, the identity matrix:
  \begin{eqnarray*}
    I_3 & = & 
    \left[
      \begin{array}{rrr}
        1 & 0 & 0\\
        0 & 1 & 0 \\
        0 & 0 & 1
      \end{array}
    \right]
  \end{eqnarray*}

  Note:
  \begin{eqnarray*}
    \left[
      \begin{array}{rrr}
        1 & 0 & 0 \\
        0 & 1 & 0 \\
        0 & 0 & 1
      \end{array}
    \right] \cdot
       \left[
      \begin{array}{rrr}
        3 & 4 & 2 \\
        -5 & 6 & 7 \\
        8 & 7 & 3
      \end{array}
    \right] & = & 
       \left[
      \begin{array}{rrr}
        3 & 4 & 2 \\
        -5 & 6 & 7 \\
        8 & 7 & 3
      \end{array}
    \right] 
  \end{eqnarray*}

  In general $I_n\cdot A = A$.

\end{frame}

\subsection{Orthogonality}

\begin{frame}
  \frametitle{Orthogonality}

  \begin{eqnarray*}
    \vec{x} & = & \vecTwo{1}{0} \\
    \vec{y} & = & \vecTwo{0}{1} \\
    \vec{x}\cdot\vec{y} & = & 0
  \end{eqnarray*}

  In general,
  \begin{eqnarray*}
    \vec{u}\cdot\vec{v} & = & \| \vec{u} \| \| \vec{v} \| \cos(\theta)
  \end{eqnarray*}

  Definition, if $\vec{u}\cdot\vec{v}=0$ then the vectors are \textbf{orthogonal}.
  
\end{frame}

\subsection{Differentiation}

\begin{frame}
  \frametitle{Differentiation}

  The derivative of a matrix is the derivative of each of its elements.
  \begin{eqnarray*}
    \frac{d}{dt} \arrayTwo{3t^{-1}}{\sin(t)}{\sqrt{t+1}}{\tan(t)} & = & 
    \arrayTwo{-3t^{-2}}{\cos(t)}{\half\lp t+1\rp^{-\half}}{\sec^2(t)}
  \end{eqnarray*}

\end{frame}

\begin{frame}
  \frametitle{Why bother?}

  Suppose that
  \begin{eqnarray*}
    \frac{d}{dt} x & = & 3x + 2y \\
    \frac{d}{dt} y & = & -2 x + 4 y
  \end{eqnarray*}

  Another way to express the system:
  \begin{eqnarray*}
    \frac{d}{dt} \vecTwo{x}{y} & = & \arrayTwo{3}{2}{-2}{4} \vecTwo{x}{y}
  \end{eqnarray*}

\end{frame}


% LocalWords:  Clarkson pausesection hideothersubsections
