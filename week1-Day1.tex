\part{Introduction}
\lecture{Introduction}{Introduction}
\section{Introduction}

\title{Ordinary Differential Equations}
\subtitle{Introduction}

\date{28 Aug 2011}

\begin{frame}
  \titlepage
\end{frame}

\begin{frame}
  \frametitle{Outline}
  \tableofcontents[pausesection,hideothersubsections,sectionstyle=show/hide]
\end{frame}


\subsection{What is a Differential Equation}


\begin{frame}
  \frametitle{What is a DE?}

  Given
  \begin{eqnarray*}
    y' & = & y,
  \end{eqnarray*}
  what is $y(x)=?$

  \begin{eqnarray*}
    \deriv{y(x)}{x} & = & y(x)
  \end{eqnarray*}

\end{frame}


\begin{frame}
  Given
  \begin{eqnarray*}
    y'' + 3y' +2y & = & 0
  \end{eqnarray*}

  what is $y(x)$?

\end{frame}

\begin{frame}
  \frametitle{Notation}
  \begin{eqnarray*}
    \dot{y} & = & \deriv{y}{t} \\
    \ddot{y} & = & \derivTwo{y}{t} \\
    y' & = & \mathrm{depends.... usually ~~} \deriv{y(x)}{x} \\
    y'' & = & \derivTwo{y(x)}{x}
  \end{eqnarray*}
\end{frame}

\begin{frame}
  \frametitle{Nomenclature}
  
  \vfill

  $\deriv{y}{x}$ - then ``ordinary differential equation.''

  \vfill

  $\frac{\partial y}{\partial x}$ - then ``partial differential
  equation.''

  \vfill

  Order is the highest number of derivatives:
  \begin{eqnarray*}
    y'' - 3 y' + 2y & = & 0, \mathrm{second~order} \\
    y'  & = & 4y, \mathrm{first~order} 
  \end{eqnarray*}

  \vfill


\end{frame}

\subsection{Modeling}


\begin{frame}
  \frametitle{Modeling}

  Why bother?

  Many ``mathematical models'' provide a relationship between rates.

  ex: Newton's Second Law, ``$\vec{F} = m \vec{a}$'' In 1 dimension:
  \begin{eqnarray*}
    m\mathrm{~(acceleration)} & = & \sum_i \lp F_x \rp_i, \\
    m \ddot{x} & = & \sum_i \lp f_x \rp_i
  \end{eqnarray*}

  
\end{frame}


\begin{frame}
  \frametitle{Circuit}
  
  The voltage across a resistor is proportional to the current flowing
  through it.

  There is some number, R, where
  \begin{eqnarray*}
    V & = & IR
  \end{eqnarray*}
\end{frame}

\begin{frame}
  \frametitle{Proportionality}
  
  If $a$ is proportional to $b$ then there is a constant, $k$, where 
  \begin{eqnarray*}
    a  & = & k \cdot b
  \end{eqnarray*}

  if $a$ is inversely proportional to $b$ then there is a constant $c$
  where 
  \begin{eqnarray*}
    a & = & c \frac{1}{b}
  \end{eqnarray*}
\end{frame}


% LocalWords:  Clarkson pausesection hideallsubsections
