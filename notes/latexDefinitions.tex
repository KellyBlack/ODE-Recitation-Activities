
% %%%%%%%%%%%%%%%%%%%%%%%%%%%%%%%%%%%%%%%%%%%%%%%%%%%%%%%%%%%%%%%%%%%%%%%
% List of definitions that are used in the different pages for the
% notes

% %%%%%%%%%%%%%%%%%%%%%%%%%%%%%%%%%%%%%%%%%%%%%%%%%%%%%%%%%%%%%%%%%%%%%%%
% Basic definitions used throughout the notes

\newcommand{\half}{\mbox{$\frac{1}{2}$}}
\newcommand{\deltat}{\mbox{$\triangle t$}}
\newcommand{\deltax}{\mbox{$\triangle x$}}
\newcommand{\deltay}{\mbox{$\triangle y$}}

\newcommand{\deriv}[2]{\frac{d}{d#2}#1}
\newcommand{\derivTwo}[2]{\frac{d^2}{d#2^2}#1}

\newcommand{\lp}{\left(}
\newcommand{\rp}{\right)}


% %%%%%%%%%%%%%%%%%%%%%%%%%%%%%%%%%%%%%%%%%%%%%%%%%%%%%%%%%%%%%%%%%%%%%%
% Basic color additions
\definecolor{fuchsia}{RGB}{255,0.0,255}

\newcommand{\redText}[1]{{\color{red}#1}}
\newcommand{\blueText}[1]{{\color{blue}#1}}
\newcommand{\greenText}[1]{{\color{green}#1}}
\newcommand{\fuchsiaText}[1]{{\color{fuchsia}#1}}

% %%%%%%%%%%%%%%%%%%%%%%%%%%%%%%%%%%%%%%%%%%%%%%%%%%%%%%%%%%%%%%%%%%%%%%
% Basic linear algebra commands

\newcommand{\arrayTwo}[4]{
  \left[
  \begin{array}{rr}
    #1 & #2 \\
    #3 & #4
  \end{array}
  \right]
}

\newcommand{\vecTwo}[2]{
  \left[
  \begin{array}{r}
    #1 \\ #2
  \end{array}
  \right]
}


\newcommand{\stateTwo}[2]{
  \begin{array}{rr}
    \mbox{\fontsize{6}{6}\selectfont $#1$} \\ \mbox{\fontsize{6}{6}\selectfont $#2$}
  \end{array}
}


\newcommand{\arrayThree}[9]{
  \left[
    \begin{array}{rrr}
      #1 & #2 & #3 \\
      #4 & #5 & #6 \\
      #7 & #8 & #9
    \end{array}
  \right]
}

\newcommand{\startRowOps}{
  \left[
    \begin{array}{rrr|r}
}

\newcommand{\oneRowOps}[4] {
      #1 & #2 & #3 & #4 \\
}

\newcommand{\stopRowOps}{
    \end{array}
  \right]
}


\newcommand{\vecThree}[3]{
  \left[
  \begin{array}{r}
    #1 \\ #2 \\ #3
  \end{array}
  \right]
}


\newcommand{\stateThree}[3]{
  \begin{array}{r}
    \mbox{\fontsize{6}{6}\selectfont $#1$} \\
    \mbox{\fontsize{6}{6}\selectfont $#2$} \\
    \mbox{\fontsize{6}{6}\selectfont $#3$}
  \end{array}
}





\newcommand{\detTwo}[4]{
  \left|
  \begin{array}{rr}
    #1 & #2 \\
    #3 & #4
  \end{array}
  \right|
}



\newcommand{\detThree}[9]{
  \left|
    \begin{array}{rrr}
      #1 & #2 & #3 \\
      #4 & #5 & #6 \\
      #7 & #8 & #9
    \end{array}
  \right|
}




\newcommand{\startRowFour}{
  \left[
    \begin{array}{rrrr}
}

\newcommand{\oneRowFour}[4] {
      #1 & #2 & #3 & #4 \\
}




\newcommand{\startRowOpsTwo}{
  \left[
    \begin{array}{rr|rr}
}

\newcommand{\oneRowOpsTwo}[4] {
      #1 & #2 & #3 & #4 \\
}


\newcommand{\startRowOpsThree}{
  \left[
    \begin{array}{rrr|rrr}
}

\newcommand{\oneRowOpsThree}[6] {
      #1 & #2 & #3 & #4 & #5 & #6 \\
}



% %%%%%%%%%%%%%%%%%%%%%%%%%%%%%%%%%%%%%%%%%%%%%%%%%%%%%%%%%%%%
% Laplace Transforms

\newcommand{\laplace}[1]{\makebox{$ {\cal L} \{ #1 \}$}}
\newcommand{\invlaplace}[1]{\makebox{$ {\cal L}^{-1} \left\{ #1 \right\}$}}




%%% Local Variables:
%%% mode: latex
%%% TeX-master: t
%%% End: 
