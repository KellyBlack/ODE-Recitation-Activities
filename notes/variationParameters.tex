\part{Variation-of-Parameters}
\lecture{Variation of Parameters}{Variation-of-Parameters}
\section{Variation of Parameters}

\title{Ordinary Differential Equations}
\subtitle{Math 232 - Week 9, Day 1}
\date{24 October 2012}

\begin{frame}
  \titlepage
\end{frame}

\begin{frame}
  \frametitle{Outline}
  \tableofcontents[pausesection,hideothersubsections]
\end{frame}


\subsection{Variation of Parameters}


\begin{frame}
  \frametitle{Variation of Parameters}

  \begin{eqnarray*}
    y'' + p(t) y' + q(t) y & = & f(t)
  \end{eqnarray*}

  \begin{itemize}
  \item Find the homogeneous solutions,
    \begin{eqnarray*}
      y_h & = & C_1 y_1(t) + C_2 y_2(t).
    \end{eqnarray*}
  \item Determine the form of a particular solution,
    \begin{eqnarray*}
      y_p & = & v_1(t) y_1(t) + v_2(t) y_2(t).
    \end{eqnarray*}
  \item Substitute this into the original equation and solve.
  \end{itemize}

\end{frame}


\begin{frame}
  \frametitle{Note}

  The form of this equation,
  \begin{eqnarray*}
    y_p & = & v_1(t) y_1(t) + v_2(t) y_2(t),
  \end{eqnarray*}
  has two unknown functions, $v_1(t)$ and $v_2(t)$.
  

  When we substitute this back into the original differential equation
  we only have one equation.

  We can pick \textbf{any} other equation that we want as long as the
  original differential equation is satisfied.

\end{frame}


\begin{frame}
  \frametitle{The other equation}

  \begin{eqnarray*}
    y_p & = & v_1 y_1 + v_2 y_2, \\
    y_p' & = & v_1' y_1 + v_1 y_1' + v_2' y_2 + v_2 y_2'. 
  \end{eqnarray*}

  Yuck, that is too messy. We want a first order equation in the end
  so we make up a new equation
  \begin{eqnarray}
    \label{eqn:firstConstraint}
    v_1' y_1 + v_2' y_2 & = & 0.
  \end{eqnarray}

  This results in the following equations:
  \begin{eqnarray*}
    y_p & = & v_1 y_1 + v_2 y_2, \\
    y_p' & = & v_1 y_1' + v_2 y_2', \\
    y_p' & = & v_1' y_1' + v_1 y_1'' + v_2' y_2' + v_2 y_2''. 
  \end{eqnarray*}


\end{frame}


\begin{frame}
  \frametitle{Substitute Back Into the Original DE}

  \begin{eqnarray*}
    v_1' y_1' + v_1 y_1'' + v_2' y_2' + v_2 y_2'' & & \\
    + p(t) \lp v_1 y_1' + v_2 y_2' \rp & & \\
    + q(t) \lp v_1 y_1 + v_2 y_2 \rp & = & f(t)
  \end{eqnarray*}

  Collect the like terms:
  \begin{eqnarray*}
    v_1' y_1' + v_2' y_2'  & & \\
    + v_1 y_1'' + v_2 y_2'' & & \\
    p(t) \lp v_1 y_1' + v_2 y_2' \rp & & \\
    q(t) \lp v_1 y_1 + v_2 y_2 \rp & = & f(t)
  \end{eqnarray*}


\end{frame}


\begin{frame}
  \frametitle{Factor the Terms}

  \begin{eqnarray*}
    v_1' y_1' + v_2' y_2'  & & \\
    + v_1 y_1'' + p(t) v_1 y_1' + q(t) v_1 y_1 & & \\
    + v_2 y_2'' + p(t) v_2 y_2' + q(t) v_2 y_2 & = & f(t)
  \end{eqnarray*}

  or 

  \begin{eqnarray*}
    v_1' y_1' + v_2' y_2'  & & \\
    + v_1 \lp y_1'' + p(t) y_1' + q(t) y_1 \rp & & \\
    + v_2 \lp y_2'' + p(t) y_2' + q(t) y_2 \rp & = & f(t)
  \end{eqnarray*}

  Since $y_1$ and $y_2$ are homogeneous solutions this reduces to 
  \begin{eqnarray}
    \label{eqn:secondConstraint}
    v_1' y_1' + v_2' y_2'  & = & f(t).
  \end{eqnarray}


\end{frame}


\begin{frame}
  \frametitle{We Have Two Equations and Two Unknowns}

  \begin{eqnarray*}
    v_1' y_1 + v_2' y_2 & = & 0 \\
    v_1' y_1' + v_2' y_2'  & = & f(t).
  \end{eqnarray*}


\end{frame}


\begin{frame}
  \frametitle{Solving the System of Equations}

  Solve the first equation for $v_1'$
  \begin{eqnarray*}
    v_1' & = & -v_2' \frac{y_2}{y_1}.
  \end{eqnarray*}

  Substitute into the second equation
  \begin{eqnarray*}
    -v_2' \frac{y_2}{y_1} y_1' + v_2' y_2'  & = & f(t), \\
    v_2' \lp -\frac{y_2}{y_1} y_1' + y_2' \rp  & = & f(t), \\
    v_2' & = & \frac{f(t)}{-\frac{y_2}{y_1} y_1' + y_2'}, \\
    v_2' & = & \frac{y_1 f(t)}{-y_2 y_1' + y_1 y_2'}. \\
    v_2' & = & \frac{y_1 f(t)}{y_1 y_2'-y_2 y_1'}. \\
  \end{eqnarray*}

\end{frame}


\begin{frame}
  \frametitle{Isolate $v_1$}

  The first function has been solved in terms of the second
  \begin{eqnarray*}
    v_1' & = & -v_2' \frac{y_2}{y_1}, \\
    v_1' & = & - \frac{y_1 f(t)}{y_1 y_2'-y_2 y_1'} \cdot \frac{y_2}{y_1}, \\
    v_1' & = & \frac{ -y_2 f(t)}{y_1 y_2'-y_2 y_1'}. \\
  \end{eqnarray*}
  


\end{frame}


\begin{frame}
  \frametitle{One More Time}

  We know the functions $y_1$ and $y_2$ and have the following:
  \begin{eqnarray*}
    v_1' & = & \frac{ -y_2 f(t)}{y_1 y_2'-y_2 y_1'} \\
    v_2' & = & \frac{y_1 f(t)}{y_1 y_2'-y_2 y_1'}
  \end{eqnarray*}

\end{frame}

\subsection{Examples}

\begin{frame}
  \frametitle{Example}

  \begin{eqnarray*}
    y'' + y & = & \tan(t).
  \end{eqnarray*}

  \uncover<2->
  {
    The homogeneous solutions are 
    \begin{eqnarray*}
      y_h & = & C_1 \cos(t) + C_2 \sin(t).
    \end{eqnarray*}
  }

  \uncover<3->
  {
    Define the particular solution to be in the following form:
    \begin{eqnarray*}
      y & = & v_1 \cos(t) + v_2 \sin(t), \\
      y_1 & = & \cos(t), \\
      y_2 & = & \sin(t).
    \end{eqnarray*}
  }

\end{frame}


\begin{frame}
  \frametitle{Substitute Into the Original Equation}

  \begin{eqnarray*}
    y' & = & v_1' \cos(t) - v_1 \sin(t) + v_2' \sin(t) + v_2 \cos(t).
  \end{eqnarray*}

  Make the following constraint:
  \begin{eqnarray*}
    v_1' \cos(t) + v_2' \sin(t)  & = & 0.
  \end{eqnarray*}

  The derivative reduces to 
  \begin{eqnarray*}
    y' & = & - v_1 \sin(t) + v_2 \cos(t), \\
    y'' & = & - v_1' \sin(t) - v_1 \cos(t) + v_2' \cos(t) - v_2 \sin(t).
  \end{eqnarray*}


\end{frame}

\begin{frame}
  \frametitle{Substitute Into the Original DE}

  \begin{eqnarray*}
    - v_1' \sin(t) - v_1 \cos(t) + v_2' \cos(t) - v_2 \sin(t) & & \\
    + v_1 \cos(t) + v_2 \sin(t) & = & \tan(t), \\
    - v_1' \sin(t) + v_2' \cos(t) & = & \tan(t).
  \end{eqnarray*}

  The two equations are 
  \begin{eqnarray*}
    v_1' \cos(t) + v_2' \sin(t)  & = & 0, \\
    - v_1' \sin(t) + v_2' \cos(t) & = & \tan(t).
  \end{eqnarray*}

\end{frame}

\begin{frame}
  \frametitle{Solve For the Two Functions}

  When you solve the system for the two functions you get 
  \begin{eqnarray*}
    v_1' & = & \frac{-\sin(t)\tan(t)}{\sin(t)\sin(t)+\cos(t)\cos(t)}, \\
    & = & -\sin(t)\tan(t), \\
    & = & \frac{-\sin^2(t)}{\cos(t)}, \\
    v_2' & = & \cos(t)\tan(t), \\
    & = & \sin(t).
  \end{eqnarray*}

\end{frame}


\begin{frame}
  \frametitle{Integrate the Relationships}

  \begin{eqnarray*}
    v_1 & = & \int \frac{-\sin^2(t)}{\cos(t)} ~ dt \\
    & = & -\ln\lp\sec(t)+\tan(t)\rp + \sin(t) + C_1. \\
    v_2 & = & \int \sin(t) ~ dt \\ 
    & = & -\cos(t) + C_2.
  \end{eqnarray*}

\end{frame}


\begin{frame}
  \frametitle{Form the Solution}

  \begin{eqnarray*}
    y & = & v_1 y_1 + v_2 y_2 \\
    & = & \lp -\ln\lp\sec(t)+\tan(t)\rp + \sin(t) + C_1 \rp \cos(t) \\
    & & 
    + \lp -\cos(t) + C_2 \rp \sin(t) \\
    & = & - \ln\lp\sec(t)+\tan(t)\rp \cos(t) + C_1 \cos(t) + C_2 \sin(t)
  \end{eqnarray*}

\end{frame}

\begin{frame}
  \frametitle{Example}
  
  \begin{eqnarray*}
    y'' + 3y' + 2y & = & \frac{1}{1+e^{2t}} 
  \end{eqnarray*}

  \uncover<2->
  {
    Find the homogeneous solution:
    \begin{eqnarray*}
      y_h & = & C_1 e^{-2t} + C_2 e^{-t}.
    \end{eqnarray*}
  }

\end{frame}


\begin{frame}
  \frametitle{Variation of Parameters}

  Assume that the solution is in the form
  \begin{eqnarray*}
    y & = & v_1 e^{-2t} + v_2 e^{-t}.
  \end{eqnarray*}

  With this definition
  \begin{eqnarray*}
    y_1 & = & e^{-2t}, \\
    y_2 & = & e^{-t}.
  \end{eqnarray*}

\end{frame}

\begin{frame}
  \frametitle{Take the Derivatives}
  \begin{eqnarray*}
    y' & = & v_1'e^{-2t} -2 v_1 e^{-2t} + v_2' e^{-t} - v_2 e^{-t}.
  \end{eqnarray*}

  Make a completely crazy and insane assumption
  \begin{eqnarray*}
    v_1'e^{-2t} + v_2' e^{-t}& = & 0.
  \end{eqnarray*}

  The second derivative is
  \begin{eqnarray*}
    y'' & = & -2 v_1' e^{-2t} + 4 v_1 e^{-2t} - v_2' e^{-t} + v_2 e^{-t}.
  \end{eqnarray*}
\end{frame}

\begin{frame}
  \frametitle{Plug This Back Into the Original DE}

  \begin{eqnarray*}
    -2 v_1' e^{-2t} + 4 v_1 e^{-2t} - v_2' e^{-t} + v_2 e^{-t} & & \\
    + 3 \lp -2 v_1 e^{-2t} - v_2 e^{-t} \rp & & \\
    2 \lp v_1 e^{-2t} + v_2 e^{-t} \rp & = & \frac{1}{1+e^{2t}}.
  \end{eqnarray*}

  Simplify to get
  \begin{eqnarray*}
    -2 v_1' e^{-2t}  - v_2' e^{-t} & = & \frac{1}{1+e^{2t}}.
  \end{eqnarray*}

\end{frame}

\begin{frame}
  \frametitle{The Two Equations}

  We now have two equations and two unknowns
  \begin{eqnarray*}
    v_1'e^{-2t} + v_2' e^{-t}& = & 0 \\
    -2 v_1' e^{-2t}  - v_2' e^{-t} & = & \frac{1}{1+e^{2t}}. 
  \end{eqnarray*}

\end{frame}

\begin{frame}
  \frametitle{Solve for the unknowns}
  \begin{eqnarray*}
    v_1' & = & \frac{-e^{-t} \frac{1}{1+e^{2t}}}{
      -e^{-t}\lp -2 e^{-2t} \rp + \lp - e^{-t}\rp \lp e^{-2t} \rp} \\
    & = & \frac{-e^{2t}}{1+e^{2t}}, \\
    v_2 & = & \frac{-e^{-2t}\frac{1}{1+e^{2t}}}{e^{-3t}}, \\
    & = & \frac{e^t}{1+e^{2t}}.
  \end{eqnarray*}
\end{frame}

\begin{frame}
  \frametitle{Integrate}
  
  \begin{eqnarray*}
    v_1 & = & \int \frac{-e^{2t}}{1+e^{2t}} ~ dt, \\
    & = & -\half \ln\lp 1 + e^{2t} \rp + C_1 \\
    v_2 & = & \int \frac{e^t}{1+e^{2t}} ~ dt, \\
    & = & \arctan\lp e^t \rp + C_2.
  \end{eqnarray*}

  Form the solution:
  \begin{eqnarray*}
    y & = & \lp -\half \ln\lp 1 + e^{2t} \rp + C_1 \rp e^{-2t} + \lp \arctan\lp e^t \rp + C_2 \rp e^{-t} \\
    & = & -\half e^{-2t} \ln \lp 1 + e^{2t} \rp + C_1 e^{-2t} + e^{-t} \arctan\lp e^t \rp + C_2 e^{-t}.
  \end{eqnarray*}

\end{frame}


\begin{frame}
  \frametitle{Example}
  \begin{eqnarray*}
    t^2 y'' - 2t y' - 4y & = & \ln(t).
  \end{eqnarray*}

  \uncover<2->
  {

    The homogeneous solution is 
    \begin{eqnarray*}
      y & = & C_1 t^4 + C_2 t^{-1}, \\
      y_1 & = & t^4, \\
      y_2 & = & t^{-1}.
    \end{eqnarray*}

    Rewrite it in the proper form:
    \begin{eqnarray*}
      y'' - \frac{2}{t} y' - \frac{4}{t^2} & = & \frac{1}{t^2} \ln(t), \\
      f(t) & = & \frac{1}{t^2} \ln(t).
    \end{eqnarray*}

  }
\end{frame}

\begin{frame}
  \frametitle{Variation of Parameters}

  \begin{eqnarray*}
    v_1' & = & \frac{-t^{-1}\frac{1}{t^2}\ln(t)}{t^4 \lp -t^{-2} \rp - t^{-1} \lp 4 t^3 \rp}, \\
    & = & \frac{-t^{-3}\ln(t)}{-5t^2}, \\
    & = & \frac{1}{5} t^{-5} \ln(t), \\
    v_2' & = & \frac{t^4 \frac{1}{t^2} \ln(t)}{-5t^2}, \\
    & = & \frac{-1}{5} \ln(t).
  \end{eqnarray*}
  

\end{frame}

\begin{frame}
  \frametitle{Integrate}

  \begin{eqnarray*}
    v_1 & = & \int \frac{1}{5} t^{-5} \ln(t) ~ dt  \\
    & = & -\frac{1}{20} t^{-4} \ln(t) - \frac{1}{80} t^{-4} + C_1 \\
    v_2 & = & \int \frac{-1}{5} \ln(t) ~ dt, \\
    & = & -\frac{1}{5} t \ln(t) + \frac{1}{5} t + C_2.
  \end{eqnarray*}

  Form the solution
  \begin{eqnarray*}
    y & = & t^4 \lp -\frac{1}{20} t^{-4} \ln(t) - \frac{1}{80} t^{-4} + C_1 \rp \\
    & & + t^{-1} \lp -\frac{1}{5} t \ln(t) + \frac{1}{5} t + C_2 \rp, \\
    & = & -\frac{1}{4} \ln(t) + \frac{3}{16} + C_1 t^4 + C_2 t^{-1}.
  \end{eqnarray*}

\end{frame}


% LocalWords:  Clarkson pausesection hideothersubsections
