\part{Variation-of-Parameters}
\lecture{Variation of Parameters}{Variation-of-Parameters}
\section{Variation of Parameters}

\title{Ordinary Differential Equations}
\subtitle{Math 232 - Variation of Parameters}
\date{19 October 2012}

\begin{frame}
  \titlepage
\end{frame}

\begin{frame}
  \frametitle{Outline}
  \tableofcontents[pausesection,hideothersubsections]
\end{frame}

\subsection{Review: Undetermined Coefficients and Cauchy-Euler}

\begin{frame}
  \frametitle{Undetermined Coefficients Method: general steps}

  Given non-homogeneous, constant coefficients DE:
  \begin{eqnarray*}
    a y'' + by' + cy & = & f(t):
  \end{eqnarray*}
  \vspace{-0.5cm}
  \begin{enumerate}
  \item Solve the homogeneous problem,
    {\color{blue}\begin{eqnarray*}
      a y_h'' + by_h' + cy_h & = & 0\\
      a r^2 + br + c & = & 0.
    \end{eqnarray*}}
  \vspace{-0.5cm}
  \item Solve for a particular solution,
    {\color{blue}\begin{eqnarray*}
      a y_p'' + by_p' + cy_p & = & f(t).
    \end{eqnarray*}}
  \item Form the general solution
   {\color{red} \begin{eqnarray*}
      y(t) & = & y_h(t) + y_p(t).
    \end{eqnarray*}}
  \item Determine the constants if initial values are given. \\
    (Do this last!)
  \end{enumerate}


\end{frame}

\begin{frame}
  \frametitle{Undetermined Coefficients Method: guessing particular solutions}
   Guessing a particular solution to
     \begin{eqnarray*}
      a y'' + b y' + cy  = f(t)
    \end{eqnarray*}

  \begin{enumerate}
  \item If {\color{blue}$f(t) = a_nt^n+\cdots a_1t+ a_0$}, then
      guess {\color{red}$$y_p(t) = b_nt^n+\cdots b_1t+b_0.$$}\\

  \item If {\color{blue} $f(t)= A e^{rt}$}, then guess
       {\color{green} $y_p(t) = Be^{rt}, Bte^{rt}, Bt^2e^{rt}$.}\\
  \vspace{.5cm}
  \item If {\color{blue}$f(t) = A\sin(rt)$}, then guess
      {\color{red}$y_p(t) = C_1\sin(rt) +C_2\cos(rt)$.}\\
  \vspace{.5cm}

  \item If {\color{blue} $f(t) = B\cos(rt)$}, then guess
      {\color{red}$y_p(t) = C_1\sin(rt) +C_2\cos(rt)$.}\\
  \vspace{.5cm}

  \item If {\color{blue}$f(t) = A\sin(rt)+B\cos(rt)$}, then guess
     {\color{red} $$y_p(t) = C_1\sin(rt) +C_2\cos(rt).$$}
  \end{enumerate}

\end{frame}


\begin{frame}
  \frametitle{Cauchy-Euler Equations}

  Find the general solution to
  {\color{red}\begin{eqnarray*}
    a t^2 y''+ bt y' + c y  & = & t^d, \\
  \end{eqnarray*}}

  \begin{enumerate}
    \item[Step 1] Assume {\color{blue}$y_h=t^r$} is the homogeneous solution. Then 
    \begin{eqnarray*}
      ar(r-1) +br + c & = & 0, 
    \end{eqnarray*}
    \vspace{-0.5cm}
    \begin{enumerate}
    \item[Case 1] Two distinct roots:  
           {\color{orange}$y_h=C_1t^{r_1}+C_2t^{r_2}$} 
    \item[Case 2] Repeated root:  
            {\color{orange}$y_h=C_1t^{r}+C_2t^{r}\ln(t)$ }
    \end{enumerate}

    \item[Step 2] Assume {\color{blue}$y_p=At^d$}. Find $d$. 
    \item[Step 3] The general solution {\color{red}$y=y_h+y_p$}. 
   \end{enumerate}
 


\end{frame}


\subsection{Another Method: Variation of Parameters(Sec 4.5)}

\begin{frame}
  \frametitle{Second order, nonhomogeneous DE}
  \vspace{-1cm}
  \begin{eqnarray}\label{eqn1}
    y'' + p(t) y' + q(t) y & = & f(t)
  \end{eqnarray}
  \begin{itemize}
  \item[Step 1] Find the homogeneous solutions,
    \begin{eqnarray*}
      \redText{y_h} & \redText{=} & \redText{C_1 y_1(t) + C_2 y_2(t)}.
    \end{eqnarray*}
  \item[Step 2] Determine the form of a particular solution,
    \begin{eqnarray*}
      \blueText{y_p} & \blueText{=} & \blueText{v_1(t) y_1(t) + v_2(t) y_2(t)}.
    \end{eqnarray*}
  \item[Step 3] Substitute $y_p$ into (\ref{eqn1}) and solve  for $v_1(t), v_2(t)$.
  \item[Step 4] The general solution is 
    \begin{eqnarray*}
      y & = & \redText{y_h}+\blueText{y_p}, \\
        & = & \redText{C_1 y_1(t) + C_2 y_2(t)} + \blueText{v_1(t) y_1(t) + v_2(t) y_2(t)}.
    \end{eqnarray*}

  \end{itemize}

\end{frame}


\begin{frame}
  \frametitle{Note}

  The form of this equation,
  \begin{eqnarray*}
    y_p & = & v_1(t) y_1(t) + v_2(t) y_2(t),
  \end{eqnarray*}
  has two unknown functions, $v_1(t)$ and $v_2(t)$.
  
  \vfill

  \uncover<2->{%
    When we substitute this back into the original differential equation
    \redText{we only have one equation}.
  }

  \vfill

  \uncover<3->{%
    We can pick \textbf{any} other equation that we want as long as
    \blueText{the original differential equation is satisfied}.
  }

  \vfill

\end{frame}


\iftoggle{clicker}{%
\begin{frame}
  \frametitle{Clicker Quiz}

   \ifnum\value{clickerQuiz}=1{%
    Find the homogeneous solution to the following differential equation:
        \begin{eqnarray*}
          y'' + y & = & \tan(t)?
        \end{eqnarray*}

        \begin{tabular}{ll}
          A: & $y_h=C_1\cos(t)+C_2\sin(t)$ \\ [12pt]
          B: & $y_h=e^t(C_1\cos(t)+C_2\sin(t))$ \\ [12pt]
          C: & $y_h=C_1e^{0t}+C_2e^{1t}$ \\ [12pt]
          D: & $y_h=C_1e^t+C_2e^{-t}$
        \end{tabular}

        \vfill

 
 }\fi

 \ifnum\value{clickerQuiz}=2{%
    Find the homogeneous solution to the following differential equation:
        \begin{eqnarray*}
          y'' + y & = & \tan(t)?
        \end{eqnarray*}

        \begin{tabular}{ll}
          A: & $y_h=e^t(C_1\cos(t)+C_2\sin(t))$ \\ [12pt]
          B: & $y_h=C_1\cos(t)+C_2\sin(t)$ \\ [12pt]
          C: & $y_h=C_1e^t+C_2e^{-t}$  \\ [12pt]
          D: & $y_h=C_1e^{0t}+C_2e^{1t}$
        \end{tabular}

        \vfill


 }\fi

 \ifnum\value{clickerQuiz}=3{%
    Find the homogeneous solution to the following differential equation:
        \begin{eqnarray*}
          y'' + y & = & \tan(t)?
        \end{eqnarray*}

        \begin{tabular}{ll}
          A: & $y_h=C_1e^{0t}+C_2e^{1t}$ \\
          B: & $y_h=C_1e^t+C_2e^{-t}$ \\
          C: & $y_h=C_1\cos(t)+C_2\sin(t)$ \\
          D: & $y_h=e^t(C_1\cos(t)+C_2\sin(t))$ \\
        \end{tabular}

        \vfill
 }\fi
\end{frame}
}



\begin{frame}
  \frametitle{Example 1}
  Find the general solution to the following DE:
  \begin{eqnarray*}
    y'' + y & = & \tan(t).
  \end{eqnarray*}

  \uncover<2->
  {
    The homogeneous solutions are
    \begin{eqnarray*}
      y_h & = & C_1 \cos(t) + C_2 \sin(t).
    \end{eqnarray*}
  }

  \uncover<3->
  {
    Define the particular solution to be in the following form:
    \begin{eqnarray*}
      y & = & v_1 \cos(t) + v_2 \sin(t), \\
      y_1 & = & \cos(t), \\
      y_2 & = & \sin(t).
    \end{eqnarray*}
  }

\end{frame}

\begin{frame}
  \frametitle{Substitute Into the Original Equation}

  \begin{eqnarray*}
    y' & = & v_1' \cos(t) - v_1 \sin(t) + v_2' \sin(t) + v_2 \cos(t).
  \end{eqnarray*}

  \uncover<2->{%
    Make the following constraint:
    \begin{eqnarray*}
      \redText{v_1' \cos(t) + v_2' \sin(t)}  & \redText{=} & \redText{0}.
    \end{eqnarray*}
  }

  \uncover<3->{%
    The derivative reduces to
    \begin{eqnarray*}
      y' & = & - v_1 \sin(t) + v_2 \cos(t), \\
      y'' & = & - v_1' \sin(t) - v_1 \cos(t) + v_2' \cos(t) - v_2 \sin(t).
    \end{eqnarray*}
  }


\end{frame}

\begin{frame}
  \frametitle{Substitute Into the Original DE}

  \begin{eqnarray*}
    - v_1' \sin(t) - v_1 \cos(t) + v_2' \cos(t) - v_2 \sin(t) & & \\
    + v_1 \cos(t) + v_2 \sin(t) & = & \tan(t).
  \end{eqnarray*}

  \uncover<2->{%
    The two equations are
    \begin{eqnarray*}
      \redText{v_1' \cos(t) + v_2' \sin(t)}    & \redText{=}  & \redText{0}, \\
      \blueText{- v_1' \sin(t) + v_2' \cos(t)} & \blueText{=} & \blueText{\tan(t)}.
    \end{eqnarray*}
  }

\end{frame}

\begin{frame}
  \frametitle{Solve For the Two Functions}

  When you solve the system for the two functions you get
  \begin{eqnarray*}
    v_1' & = & \frac{-\sin(t)\tan(t)}{\sin(t)\sin(t)+\cos(t)\cos(t)}, \\
    & = & -\sin(t)\tan(t), \\
    & = & \frac{-\sin^2(t)}{\cos(t)}, \\
    v_2' & = & \cos(t)\tan(t), \\
    & = & \sin(t).
  \end{eqnarray*}

\end{frame}


\begin{frame}
  \frametitle{Integrate the Relationships}

  \begin{eqnarray*}
    v_1 & = & \int \frac{-\sin^2(t)}{\cos(t)} ~ dt \\
    & = & -\ln\lp\sec(t)+\tan(t)\rp + \sin(t) + C_1. \\
    v_2 & = & \int \sin(t) ~ dt \\ 
    & = & -\cos(t) + C_2.
  \end{eqnarray*}

\end{frame}


\begin{frame}
  \frametitle{Form the Solution}

  \begin{eqnarray*}
    y & = & v_1 y_1 + v_2 y_2 \\
    & = & \lp -\ln\lp\sec(t)+\tan(t)\rp + \sin(t) + C_1 \rp \cos(t) \\
    & & 
    + \lp -\cos(t) + C_2 \rp \sin(t) \\
    & = & - \ln\lp\sec(t)+\tan(t)\rp \cos(t) + C_1 \cos(t) + C_2 \sin(t)
  \end{eqnarray*}

\end{frame}
%...........................................

\subsection{General Derivation}

\begin{frame}
General derivation of particular solution to $$y''+p(t)y'+q(t)=f(t):$$  
  \frametitle{The other equation}

  We find the homogeneous solution 
  \begin{eqnarray*}
    y_h & = & C_1  y_1 + C_1 y_2. 
  \end{eqnarray*}
  Assume we have a particular solution
  \begin{eqnarray*}
    y_p & = & v_1 y_1 + v_2 y_2, \\
    y_p' & = & v_1' y_1 + v_1 y_1' + v_2' y_2 + v_2 y_2'. 
  \end{eqnarray*}

  Yuck, that is too messy. We want a first order equation in the end
  so {\color{red}we make up a new equation
  \begin{eqnarray}
    \label{eqn:firstConstraint}
    v_1' y_1 + v_2' y_2 & = & 0.
  \end{eqnarray}}

\end{frame}


\begin{frame}
General derivation of particular solution to $$y''+p(t)y'+q(t)=f(t):$$  
  \frametitle{The other equation}

  We find the homogeneous solution 
  \begin{eqnarray*}
    y_h & = & C_1  y_1 + C_1 y_2. 
  \end{eqnarray*}
  Assume  a particular solution and assume an extra condition
  \begin{eqnarray*}
    y_p & = & v_1 y_1 + v_2 y_2, \\
    v_1' y_1 + v_2' y_2 & = & 0.
  \end{eqnarray*}


  This results in the following equations:
  \begin{eqnarray*}
<<<<<<< HEAD
    y_p & = & v_1 y_1 + v_2 y_2, \\
    y_p' & = & v_1 y_1' + v_2 y_2', \\
=======
    y_p   & = & v_1 y_1 + v_2 y_2, \\
    y_p'  & = & v_1 y_1' + v_2 y_2', \\
>>>>>>> 5cd16194628e0d5d1b2f613f38a1e09f502d32aa
    y_p'' & = & v_1' y_1' + v_1 y_1'' + v_2' y_2' + v_2 y_2''. 
  \end{eqnarray*}


\end{frame}




\begin{frame}
  \frametitle{Substitute Back Into the Original DE}

  \begin{eqnarray*}
    v_1' y_1' + v_1 y_1'' + v_2' y_2' + v_2 y_2'' & & \\
    + p(t) \lp v_1 y_1' + v_2 y_2' \rp & & \\
    + q(t) \lp v_1 y_1 + v_2 y_2 \rp & = & f(t)
  \end{eqnarray*}

  Collect the like terms:
  \begin{eqnarray*}
    v_1' y_1' + v_2' y_2'  & & \\
    + v_1 y_1'' + v_2 y_2'' & & \\
    p(t) \lp v_1 y_1' + v_2 y_2' \rp & & \\
    q(t) \lp v_1 y_1 + v_2 y_2 \rp & = & f(t)
  \end{eqnarray*}


\end{frame}


\begin{frame}
  \frametitle{Factor the Terms}

  \begin{eqnarray*}
    v_1' y_1' + v_2' y_2'  & & \\
    + v_1 y_1'' + p(t) v_1 y_1' + q(t) v_1 y_1 & & \\
    + v_2 y_2'' + p(t) v_2 y_2' + q(t) v_2 y_2 & = & f(t)
  \end{eqnarray*}

  or 

  \begin{eqnarray*}
    v_1' y_1' + v_2' y_2'  & & \\
    + v_1 \lp  \redText{y_1'' + p(t) y_1' + q(t) y_1} \rp & & \\
    + v_2 \lp \blueText{y_2'' + p(t) y_2' + q(t) y_2} \rp & = & f(t)
  \end{eqnarray*}

  Since $y_1$ and $y_2$ are homogeneous solutions this reduces to 
  \begin{eqnarray}
    \label{eqn:secondConstraint}
    v_1' y_1' + v_2' y_2'  & = & f(t).
  \end{eqnarray}


\end{frame}


\begin{frame}
  \frametitle{We Have Two Equations and Two Unknowns}

  \begin{eqnarray*}
    v_1' y_1 + v_2' y_2 & = & 0 \\
    v_1' y_1' + v_2' y_2'  & = & f(t).
  \end{eqnarray*}


\end{frame}


\begin{frame}
  \frametitle{Solving the System of Equations}

  Solve the first equation for $v_1'$
  \begin{eqnarray*}
    v_1' & = & -v_2' \frac{y_2}{y_1}.
  \end{eqnarray*}

  Substitute into the second equation
  \begin{eqnarray*}
    -v_2' \frac{y_2}{y_1} y_1' + v_2' y_2'  & = & f(t), \\
    v_2' \lp -\frac{y_2}{y_1} y_1' + y_2' \rp  & = & f(t), \\
    v_2' & = & \frac{f(t)}{-\frac{y_2}{y_1} y_1' + y_2'}, \\
    v_2' & = & \frac{y_1 f(t)}{-y_2 y_1' + y_1 y_2'}. \\
    v_2' & = & \frac{y_1 f(t)}{y_1 y_2'-y_2 y_1'}. \\
  \end{eqnarray*}

\end{frame}


\begin{frame}
  \frametitle{Isolate $v_1$}

  The first function has been solved in terms of the second
  \begin{eqnarray*}
    v_1' & = & -v_2' \frac{y_2}{y_1}, \\
    v_1' & = & - \frac{y_1 f(t)}{y_1 y_2'-y_2 y_1'} \cdot \frac{y_2}{y_1}, \\
    v_1' & = & \frac{ -y_2 f(t)}{y_1 y_2'-y_2 y_1'}. \\
  \end{eqnarray*}
  


\end{frame}


\begin{frame}
  \frametitle{One More Time}

  We know the functions $y_1$ and $y_2$ and have the following:
  \begin{eqnarray*}
    v_1' & = & \frac{ -y_2 f(t)}{y_1 y_2'-y_2 y_1'} \\
    v_2' & = & \frac{y_1 f(t)}{y_1 y_2'-y_2 y_1'}
  \end{eqnarray*}
  Integrate the right-hand side, we can find $v_1(t), v_2(t)$ such that
  \begin{eqnarray*}
      y_p & = & v_1(t) y_1(t) + v_2(t) y_2(t)
    \end{eqnarray*}
  form a particular solution.   
\end{frame}

\subsection{Examples}


\iftoggle{clicker}{%
\begin{frame}
  \frametitle{Clicker Quiz}

   \ifnum\value{clickerQuiz}=1{%
    Find the homogeneous solution to the following differential equation:
        \begin{eqnarray*}
          y'' + 3y' + 2y & = & \frac{1}{1+e^{2t}} 
        \end{eqnarray*}

        \begin{tabular}{ll}
          A: & $y_h  =  C_1 e^{2t} + C_2 e^{t}$\\ [12pt]
          B: & $y_h  =  C_1 e^{-2t} + C_2 e^{-t}$ \\ [12pt]
          C: & $y_h  =  C_1 e^{2t} + C_2 e^{-t}$\\ [12pt]
          D: & $y_h  =  C_1 e^{-2t} + C_2 e^{t}$ 
        \end{tabular}
        \vfill


 }\fi

 \ifnum\value{clickerQuiz}=2{%
    Find the homogeneous solution to the following differential equation:
        \begin{eqnarray*}
          y'' + 3y' + 2y & = & \frac{1}{1+e^{2t}} 
        \end{eqnarray*}

        \begin{tabular}{ll}
          A: & $y_h  =  C_1 e^{-2t} + C_2 e^{-t}$ \\ [12pt]
          B: & $y_h  =  C_1 e^{2t} + C_2 e^{t}$  \\ [12pt]
          C: & $y_h  =  C_1 e^{-2t} + C_2 e^{t}$  \\ [12pt] 
          D: & $y_h  =  C_1 e^{2t} + C_2 e^{-t}$
        \end{tabular}

        \vfill


 }\fi

 \ifnum\value{clickerQuiz}=3{%
    Find the homogeneous solution to the following differential equation:
        \begin{eqnarray*}
          y'' + 3y' + 2y & = & \frac{1}{1+e^{2t}} 
        \end{eqnarray*}

        \begin{tabular}{ll}
          A: & $y_h  =  C_1 e^{2t} + C_2 e^{-t}$\\
          B: & $y_h  =  C_1 e^{-2t} + C_2 e^{t}$\\
          C: & $y_h  =  C_1 e^{2t} + C_2 e^{t}$\\
          D: & $y_h  =  C_1 e^{-2t} + C_2 e^{-t}$
        \end{tabular}

        \vfill
 }\fi
\end{frame}
}


\begin{frame}
  \frametitle{Example 2}
  Find the general solution to the following DE: 
  \begin{eqnarray*}
          y'' + 3y' + 2y & = & \frac{1}{1+e^{2t}}.
  \end{eqnarray*}
 
  \uncover<2->
  {
    Find the homogeneous solution:
    \begin{eqnarray*}
      y & = & C_1 e^{-2t} + C_2 e^{-t}.
    \end{eqnarray*}
  }

\end{frame}


\begin{frame}
  \frametitle{Variation of Parameters}

  Assume that the solution is in the form
  \begin{eqnarray*}
    y & = & v_1 e^{-2t} + v_2 e^{-t}.
  \end{eqnarray*}

  With this definition
  \begin{eqnarray*}
    y_1 & = & e^{-2t}, \\
    y_2 & = & e^{-t}.
  \end{eqnarray*}

\end{frame}

\begin{frame}
  \frametitle{Take the Derivatives}
  \begin{eqnarray*}
    y' & = & v_1'e^{-2t} -2 v_1 e^{-2t} + v_2' e^{-t} - v_2 e^{-t}.
  \end{eqnarray*}

  Make a completely crazy and insane assumption
  \begin{eqnarray*}
    \redText{v_1'e^{-2t} + v_2' e^{-t}} & \redText{=} & \redText{0}.
  \end{eqnarray*}

  The second derivative is
  \begin{eqnarray*}
    y'' & = & -2 v_1' e^{-2t} + 4 v_1 e^{-2t} - v_2' e^{-t} + v_2 e^{-t}.
  \end{eqnarray*}
\end{frame}

\begin{frame}
  \frametitle{Plug This Back Into the Original DE}

  \begin{eqnarray*}
    -2 v_1' e^{-2t} + 4 v_1 e^{-2t} - v_2' e^{-t} + v_2 e^{-t} & & \\
    + 3 \lp -2 v_1 e^{-2t} - v_2 e^{-t} \rp & & \\
    2 \lp v_1 e^{-2t} + v_2 e^{-t} \rp & = & \frac{1}{1+e^{2t}}.
  \end{eqnarray*}

  Simplify to get
  \begin{eqnarray*}
    \blueText{-2 v_1' e^{-2t}  - v_2' e^{-t}} & \blueText{=} & \blueText{\frac{1}{1+e^{2t}}}.
  \end{eqnarray*}

\end{frame}

\begin{frame}
  \frametitle{The Two Equations}

  We now have two equations and two unknowns
  \begin{eqnarray*}
    v_1'e^{-2t} + v_2' e^{-t}& = & 0 \\
    -2 v_1' e^{-2t}  - v_2' e^{-t} & = & \frac{1}{1+e^{2t}}. 
  \end{eqnarray*}

\end{frame}

\begin{frame}
  \frametitle{Solve for the unknowns}
  \begin{eqnarray*}
    v_1' & = & \frac{-e^{-t} \frac{1}{1+e^{2t}}}{
      -e^{-t}\lp -2 e^{-2t} \rp + \lp - e^{-t}\rp \lp e^{-2t} \rp} \\
    & = & \frac{-e^{2t}}{1+e^{2t}}, \\
    v_2' & = & \frac{-e^{-2t}\frac{1}{1+e^{2t}}}{e^{-3t}}, \\
    & = & \frac{e^t}{1+e^{2t}}.
  \end{eqnarray*}
\end{frame}

\begin{frame}
  \frametitle{Integrate}
  
  \begin{eqnarray*}
    v_1 & = & \int \frac{-e^{2t}}{1+e^{2t}} ~ dt, \\
    & = & -\half \ln\lp 1 + e^{2t} \rp + C_1 \\
    v_2 & = & \int \frac{e^t}{1+e^{2t}} ~ dt, \\
    & = & \arctan\lp e^t \rp + C_2.
  \end{eqnarray*}

  Form the solution:
  \begin{eqnarray*}
    y & = & \lp -\half \ln\lp 1 + e^{2t} \rp + C_1 \rp e^{-2t} + \lp \arctan\lp e^t \rp + C_2 \rp e^{-t} \\
    & = & -\half e^{-2t} \ln \lp 1 + e^{2t} \rp + C_1 e^{-2t} + e^{-t} \arctan\lp e^t \rp + C_2 e^{-t}.
  \end{eqnarray*}

\end{frame}


\iftoggle{clicker}{%
\begin{frame}
  \frametitle{Clicker Quiz}

   \ifnum\value{clickerQuiz}=1{%
    Find the homogeneous solution to the following differential equation:
        \begin{eqnarray*}
          t^2 y'' - 2t y' - 4y & = & \ln(t).
        \end{eqnarray*}

        \begin{tabular}{ll}
          A: & $y_h  =  C_1 t^4 + C_2 t^{-1}$ \\ [12pt]
          B: & $y_h  =  C_1 \ln(4t) + C_2 \ln(-t)$\\ [12pt]
          C: & $y_h  =  C_1 e^{(1-\sqrt{5})t} + C_2 e^{(1+\sqrt{5})t}$ 
        \end{tabular}
    \vfill


 }\fi

 \ifnum\value{clickerQuiz}=2{%
    Find the homogeneous solution to the following differential equation:
        \begin{eqnarray*}
          t^2 y'' - 2t y' - 4y & = & \ln(t).
        \end{eqnarray*}

        \begin{tabular}{ll}
          A: & $y_h  =  C_1 \ln(4t) + C_2 \ln(-t)$\\ [12pt]
          B: & $y_h  =  C_1 t^4 + C_2 t^{-1}$ \\ [12pt]
          C: & $y_h  =  C_1 e^{(1-\sqrt{5})t} + C_2 e^{(1+\sqrt{5})t}$ 
        \end{tabular}
    \vfill

 \vfill


 }\fi

 \ifnum\value{clickerQuiz}=3{%
    Find the homogeneous solution to the following differential equation:
        \begin{eqnarray*}
          t^2 y'' - 2t y' - 4y & = & \ln(t).
        \end{eqnarray*}

        \begin{tabular}{ll}
          A: & $y_h  =  C_1 t^4 + C_2 t^{-1}$ \\
          B: & $y_h  =  C_1 e^{(1-\sqrt{5})t} + C_2 e^{(1+\sqrt{5})t}$\\
          C: & $y_h  =  C_1 \ln(4t) + C_2 \ln(-t)$\\
        \end{tabular}

        \vfill
 }\fi
\end{frame}
}


\begin{frame}
  \frametitle{Example 3}
  Find the general solution to the following DE:
  \begin{eqnarray*}
    t^2 y'' - 2t y' - 4y & = & \ln(t).
  \end{eqnarray*}

  \uncover<2->
  {

    The homogeneous solution is 
    \begin{eqnarray*}
      y & = & C_1 t^4 + C_2 t^{-1}, \\
      y_1 & = & t^4, \\
      y_2 & = & t^{-1}.
    \end{eqnarray*}

    Rewrite it in the proper form:
    \begin{eqnarray*}
      y'' - \frac{2}{t} y' - \frac{4}{t^2} & = & \frac{1}{t^2} \ln(t), \\
      f(t) & = & \frac{1}{t^2} \ln(t).
    \end{eqnarray*}

  }
\end{frame}

\begin{frame}
  \frametitle{Variation of Parameters}

  \begin{eqnarray*}
    v_1' & = & \frac{-t^{-1}\frac{1}{t^2}\ln(t)}{t^4 \lp -t^{-2} \rp - t^{-1} \lp 4 t^3 \rp}, \\
    & = & \frac{-t^{-3}\ln(t)}{-5t^2}, \\
    & = & \frac{1}{5} t^{-5} \ln(t), \\
    v_2' & = & \frac{t^4 \frac{1}{t^2} \ln(t)}{-5t^2}, \\
    & = & \frac{-1}{5} \ln(t).
  \end{eqnarray*}
  

\end{frame}

\begin{frame}
  \frametitle{Integrate}

  \begin{eqnarray*}
    v_1 & = & \int \frac{1}{5} t^{-5} \ln(t) ~ dt  \\
    & = & -\frac{1}{20} t^{-4} \ln(t) - \frac{1}{80} t^{-4} + C_1 \\
    v_2 & = & \int \frac{-1}{5} \ln(t) ~ dt, \\
    & = & -\frac{1}{5} t \ln(t) + \frac{1}{5} t + C_2.
  \end{eqnarray*}

  Form the solution
  \begin{eqnarray*}
    y & = & t^4 \lp -\frac{1}{20} t^{-4} \ln(t) - \frac{1}{80} t^{-4} + C_1 \rp \\
    & & + t^{-1} \lp -\frac{1}{5} t \ln(t) + \frac{1}{5} t + C_2 \rp, \\
    & = & -\frac{1}{4} \ln(t) + \frac{3}{16} + C_1 t^4 + C_2 t^{-1}.
  \end{eqnarray*}

\end{frame}


% LocalWords:  Clarkson pausesection hideothersubsections
