\part{Laplace-Transforms}
\lecture{Laplace Transforms}{Laplace-Transforms}
\section{Laplace Transforms}


\title{Ordinary Differential Equations}
\subtitle{Math 232 - Week 12, Day 3}
\date{18 November 2012}

\begin{frame}
  \titlepage
\end{frame}

\begin{frame}
  \frametitle{Outline}
  \tableofcontents[pausesection,hideothersubsections]
\end{frame}


\subsection{Laplace Transform}


\begin{frame}
  \frametitle{Example}

  \begin{eqnarray*}
    \int^\infty_0 e^{-3t} ~ dt & = & \lim_{M\rightarrow\infty} -\frac{1}{3} e^{-3t} \bigg|^M_0, \\
    & = & \frac{1}{3}, \\
    \uncover<2->
    {
    \int^\infty_0 e^{-st} ~ dt & = & \lim_{M\rightarrow\infty} -\frac{1}{s} e^{-st} \bigg|^M_0, \\
    & = & \frac{1}{s}
    }
  \end{eqnarray*}
  

\end{frame}


\begin{frame}
  \frametitle{Laplace Transform}


  \begin{definition}
    The Laplace Transform of a function, $f(t)$, is 
    \begin{eqnarray*}
      \laplace{f} & = & F(s), \\
      & = & \int^\infty_0 f(t) e^{-st} ~ dt.
    \end{eqnarray*}
    \textit{(If it exists.)}
  \end{definition}

\end{frame}


\begin{frame}
  \frametitle{Properties of the Laplace Transform}

  The Laplace transform of the sum:
  \begin{eqnarray*}
    \laplace{f+g} & = & \int^\infty_0 \lp f(t) + g(t) \rp e^{-st} ~ dt, \\
    & = & \int^\infty_0 f(t) e^{-st} ~ dt + \int^\infty_0 g(t) e^{-st} ~ dt, \\
    & = & \laplace{f} + \laplace{g}.
  \end{eqnarray*}

  
  The Laplace transform of a constant times the function. Suppose $a$ is a constant, then
  \begin{eqnarray*}
    \laplace{a f} & = & \int^\infty_0 a f(t) e^{-st} ~ dt, \\
    & = & a \int^\infty_0 f(t) e^{-st} ~ dt, \\
    & = & a \laplace{f}.
  \end{eqnarray*}


\end{frame}


\begin{frame}
  \frametitle{So What?}

  If 
  \begin{eqnarray*}
    \laplace{f} & = & F(s)
  \end{eqnarray*}
  then given $F(s)$ we can find $f(t)$. (We will do that later.) Also,
  integrals can ``get rid of'' derivatives.

\end{frame}

\subsection{Laplace Transforms of Polynomials}

\begin{frame}
  \frametitle{Laplace Transforms of Polynomials}

  \begin{eqnarray*}
    \begin{array}{rcl@{\hspace{1em}}c@{\hspace{1em}}l}
    \laplace{1} & = & \int^\infty_0 1 \cdot e^{-st} ~ dt
    & = & \frac{1}{s}, \\ [10pt]
    \laplace{t} & = & \int^\infty_0 t \cdot e^{-st} ~ dt
    & = & \frac{1}{s^2}, \\ [10pt]
    \laplace{t^2} & = & \int^\infty_0 t^2 \cdot e^{-st} ~ dt
    & = & \frac{2}{s^3}.      
    \end{array}
  \end{eqnarray*}

\end{frame}


\begin{frame}
  \frametitle{Example}

  Find the Laplace transform of 
  \begin{eqnarray*}
    f(t) & = & 3 + 5t - 8t^2.
  \end{eqnarray*}

  The Laplace transform is
  \begin{eqnarray*}
    \laplace{f} & = & \frac{3}{s} + \frac{5}{s^2} - \frac{16}{s^3}.
  \end{eqnarray*}

\end{frame}


\begin{frame}
  \frametitle{Example}

  Find the Laplace transform of 
  \begin{eqnarray*}
    f(t) & = & 4 - 3t + 7t^2.
  \end{eqnarray*}

  The Laplace transform is
  \begin{eqnarray*}
    \laplace{f} & = & \frac{4}{s} - \frac{3}{s^2} + \frac{14}{s^3}.
  \end{eqnarray*}

\end{frame}


\begin{frame}
  \frametitle{Laplace Transform of Higher Order Polynomials}

  In general we have
  \begin{eqnarray*}
    \laplace{t^n} & = & \frac{n!}{s^{n+1}}.
  \end{eqnarray*}

\end{frame}


\begin{frame}
  \frametitle{Example}

  Find the Laplace transform of 
  \begin{eqnarray*}
    f(t) & = & 5 t^3 - 17 t^5.
  \end{eqnarray*}

  The Laplace transform is
  \begin{eqnarray*}
    \laplace{f} & = & 5 \frac{3!}{s^4} - 17 \frac{5!}{s^6}.
  \end{eqnarray*}

\end{frame}


\subsection{Laplace Transform of Exponentials}

\begin{frame}
  \frametitle{Laplace Transform of Exponentials}

  \begin{eqnarray*}
    \laplace{e^{at}} & = & \int^\infty_0 e^{at} e^{-st} ~ dt, \\
    & = & \int^\infty_0 e^{(a-s)t} ~ dt, \\
    \Rightarrow \laplace{e^{at}} & = & \frac{1}{s-a}.
  \end{eqnarray*}

\end{frame}


\begin{frame}
  \frametitle{Example}

  Find the Laplace transform of 
  \begin{eqnarray*}
    f(t) & = & e^{3t}.
  \end{eqnarray*}

  The Laplace transform is
  \begin{eqnarray*}
    \laplace{f} & = & \frac{1}{s-3}.
  \end{eqnarray*}

  \uncover<2->
  {
    Find the Laplace transform of 
    \begin{eqnarray*}
      g(t) & = & e^{-5t}.
    \end{eqnarray*}

    The Laplace transform is
    \begin{eqnarray*}
      \laplace{g} & = & \frac{1}{s+5}.
    \end{eqnarray*}
  }


\end{frame}


\begin{frame}
  \frametitle{Example}

  Find the Laplace transform of 
  \begin{eqnarray*}
    f(t) & = & 4t^2 - 18 e^{-6t}.
  \end{eqnarray*}

  The Laplace transform is
  \begin{eqnarray*}
    \laplace{f} & = & 4 \frac{2!}{s^3} - 18 \frac{1}{s+6}.
  \end{eqnarray*}

\end{frame}


\subsection{Laplace Transform of Sines and Cosines}

\begin{frame}
  \frametitle{Sines and Cosines}

  Note that
  \begin{eqnarray*}
    \laplace{e^{iat}} & = & \frac{1}{s-ai}, \\
    & = & \frac{s+ai}{(s-ai)(s+ai)}, \\
    & = & \frac{s+ai}{s^2+a^2}, \\
    & = & \frac{s}{s^2+a^2} + i \frac{a}{s^2+a^2}.
  \end{eqnarray*}

  At the same time, though,
  \begin{eqnarray*}
    \laplace{e^{iat}} & = & \laplace{\cos(at) + i \sin(at)}, \\
    & = & \laplace{\cos(at)} + i \laplace{\sin(at)}.
  \end{eqnarray*}


\end{frame}


\begin{frame}
  \frametitle{Sines and Cosines}

  The Laplace transform of the sine and cosine is the following:
  \begin{eqnarray*}
    \laplace{\cos(at)} & = & \frac{s}{s^2+a^2}, \\
    \laplace{\sin(at)} & = & \frac{a}{s^2+a^2}.
  \end{eqnarray*}


\end{frame}


\begin{frame}
  \frametitle{Example}

  Find the Laplace transform of 
  \begin{eqnarray*}
    f(t) & = & 5t^4 + 2\sin(4t) - 3 e^{8t} + 7 \cos(5t).
  \end{eqnarray*}

  The Laplace transform is
  \begin{eqnarray*}
    \laplace{f} & = & 5 \frac{4!}{s^5} + 2 \frac{4}{s^2+16}  - 3\frac{1}{s-8} + 7 \frac{s}{s^2+25}.
  \end{eqnarray*}

\end{frame}

\subsection{More Identities}

\begin{frame}
  \frametitle{More Identities}

  \begin{eqnarray*}
    \laplace{t f(t)} & = & \int^\infty_0 t f(t) e^{-st} ~ dt, \\
    & = & \int^\infty_0 f(t) \lp -\deriv{~}{s} e^{-st} \rp ~ dt, \\
    & = & -\deriv{~}{s} \int^\infty_0 f(t) e^{-st}  ~ dt, \\
    & = & -\deriv{~}{s} \laplace{f}(s).
  \end{eqnarray*}

\end{frame}

\begin{frame}
  \frametitle{Moar Moar Identities!}
  \begin{columns}
    \column{.5\textwidth} 
    \includegraphics[height=4cm]{img/cagemoar}


    \column{.5\textwidth}
    \begin{eqnarray*}
      \laplace{t^2 f(t)} & = & \frac{d^2}{ds^2} \laplace{f}, \\
      \uncover<2->
      {
        \laplace{t^3 f(t)} & = & -\frac{d^3}{ds^3} \laplace{f}, \\
      }
      \uncover<3->
      {
        \vdots             &   & \vdots \\
        \laplace{t^n f(t)} & = & (-1)^n \frac{d^n}{ds^n} \laplace{f}.
      }
    \end{eqnarray*}
  \end{columns}
  
\end{frame}


\begin{frame}
  \frametitle{Example}
    \begin{eqnarray*}
      f(t) & = & 7t \sin(4t) + 2t^3e^{-4t}, \\
      \uncover<2->
      {
        \laplace{f} & = & \deriv{~}{s} \frac{28}{s^2+16} - \frac{d^3}{ds^3} \frac{2}{(s+4)}, \\
         & = & \frac{56s}{\lp s^2+16\rp^2} + 12 \frac{1}{(s+4)^4}.
      }
    \end{eqnarray*}
\end{frame}

\begin{frame}
  \frametitle{Moar Moar Moar Identities!}
  \begin{columns}
    \column{.1\textwidth} 
    \includegraphics[height=4cm]{img/moar_18}


    \column{.9\textwidth}
    \begin{eqnarray*}
      \laplace{e^{at} f(t)} & = & \int^\infty_0 e^{at} f(t) e^{-st} ~ dt, \\
      \uncover<2->
      {
        & = & \int^\infty_0 f(t) e^{-(s-a)t} ~ dt, \\
      }
      \uncover<3->
      {
        & = & \laplace{f}(s-a).
      }
    \end{eqnarray*}
  \end{columns}
  
\end{frame}


\begin{frame}
  \frametitle{Example}
    \begin{eqnarray*}
      f(t) & = & 3t^4 e^{2t} - 4 e^{-5t}\sin(3t), \\
      \uncover<2->
      {
        \laplace{f} & = & 3 \laplace{t^4}(s-2) - 4 \laplace{\sin(3t)}(s+5), \\
      }
      \uncover<3->
      {
        & = & 3 \frac{4!}{(s-2)^5} - 4 \frac{3}{(s+5)^2+9}.
      }
    \end{eqnarray*}
\end{frame}


\begin{frame}
  \frametitle{Example}
    \begin{eqnarray*}
      f(t) & = & 2 e^{-4t}\cos(3t) - 5 t^2 \sin(2t), \\
      \uncover<2->
      {
        \laplace{f} & = & 3 \laplace{\cos(3t)}(s+4) - 5 (-1)^2 \frac{d^2}{ds^2} \laplace{\sin(3t)}, \\
      }
      \uncover<3->
      {
        & = & 2 \frac{s+4}{(s+4)^2+9} + 20 \lp -2(s^2+4)^{-3}(2s) + (s^2+4)^{-2} \rp.
      }
    \end{eqnarray*}
\end{frame}


\begin{frame}
  \frametitle{Oh By The Way....}

  \begin{definition}
    The hyperbolic sine and hyperbolic cosine:
    \begin{eqnarray*}
      \sinh(t) & = & \frac{e^t-e^{-t}}{2}, \\
      \cosh(t) & = & \frac{e^t+e^{-t}}{2}.
    \end{eqnarray*}
  \end{definition}

  \uncover<2->
  {
    Also,
    \begin{eqnarray*}
      \laplace{\sinh(t)} & = & \frac{a}{s^2-a^2}, \\
      \laplace{\cosh(t)} & = & \frac{s}{s^2-a^2}.
    \end{eqnarray*}
  }

  
\end{frame}



% LocalWords:  Clarkson pausesection hideothersubsections
