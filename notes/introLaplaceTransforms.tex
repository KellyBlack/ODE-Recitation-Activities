\part{Laplace-Transforms}
\lecture{Laplace Transforms}{Laplace-Transforms}
\section{Laplace Transforms}


\title{Ordinary Differential Equations}
\subtitle{Introduction to the Laplace Transform}
\date{12 November 2012}

\begin{frame}
  \titlepage
\end{frame}

\begin{frame}
  \frametitle{Outline}
  \tableofcontents[pausesection,hideothersubsections]
\end{frame}


\subsection{Laplace Transform}


\begin{frame}
  \frametitle{Example}

  \begin{eqnarray*}
    \int^\infty_0 e^{-3t} ~ dt & = & 
    \uncover<2->
    {%
      \lim_{M\rightarrow\infty} -\frac{1}{3} e^{-3t} \bigg|^M_0, \\
      & = & \frac{1}{3}.
    }
  \end{eqnarray*}

\end{frame}


\iftoggle{clicker}{%
\begin{frame}
  \frametitle{Clicker Quiz}

   \ifnum\value{clickerQuiz}=1{%
     Is the integral 
     \begin{eqnarray*}
       \int^\infty_0 e^{-7t} ~ dt
     \end{eqnarray*}
     a function of time?

     \vspace{2em}
     \begin{tabular}{ll}
       A: & Yes \\ [12pt]
       B: & No
     \end{tabular}

     \vfill
   }\fi

   \ifnum\value{clickerQuiz}=2{%
     Is the integral 
     \begin{eqnarray*}
       \int^\infty_0 e^{-2t} ~ dt
     \end{eqnarray*}
     a function of time?

     \vspace{2em}
     \begin{tabular}{ll}
       A: & Yes \\ [12pt]
       B: & No
     \end{tabular}

   \vfill
   }\fi

  \ifnum\value{clickerQuiz}=3{%


  \vfill
 }\fi
\end{frame}
}


\begin{frame}
  \frametitle{Example}

  \begin{eqnarray*}
    \int^\infty_0 e^{-st} ~ dt & = & 
    \uncover<2->
    {%
      \lim_{M\rightarrow\infty} -\frac{1}{s} e^{-st} \bigg|^M_0, \\
      & = & \frac{1}{s}
    }
  \end{eqnarray*}
  (Assume that $s>0$.)

  \uncover<3->
  {%
    Note that this is not a function of $t$. 
  }
  

\end{frame}


\begin{frame}
  \frametitle{Laplace Transform}


  \begin{definition}
    The Laplace Transform of a function, $f(t)$, is 
    \begin{eqnarray*}
      \laplace{f} & = & F(s), \\
      & = & \int^\infty_0 f(t) e^{-st} ~ dt.
    \end{eqnarray*}
    \textit{(If it exists.)}
  \end{definition}

  \uncover<2->
  {%
    It will turn out that this is useful for solving differential
    equations because integrals ``undo'' derivatives. The goal is to
    turn a calculus problem into an algebra problem.
  }

\end{frame}


\iftoggle{clicker}{%
\begin{frame}
  \frametitle{Clicker Quiz}

   \ifnum\value{clickerQuiz}=1{%
     Is the the Laplace Transform 
     \begin{eqnarray*}
       F(s) & = & \int^\infty_0 f(t) e^{-st} ~ dt
     \end{eqnarray*}
     a function of time?

     \vspace{2em}
     \begin{tabular}{ll}
       A: & Yes \\ [12pt]
       B: & No
     \end{tabular}

     \vfill
   }\fi

   \ifnum\value{clickerQuiz}=2{%
     Is the the Laplace Transform 
     \begin{eqnarray*}
       F(s) & = & \int^\infty_0 f(t) e^{-st} ~ dt
     \end{eqnarray*}
     a function of time?

     \vspace{2em}
     \begin{tabular}{ll}
       A: & Yes \\ [12pt]
       B: & No
     \end{tabular}

   \vfill
   }\fi

  \ifnum\value{clickerQuiz}=3{%


  \vfill
 }\fi
\end{frame}
}


\begin{frame}
  \frametitle{Properties of the Laplace Transform}

  The Laplace transform of the sum:
  \begin{eqnarray*}
    \laplace{f+g} & = & \int^\infty_0 \lp f(t) + g(t) \rp e^{-st} ~ dt, \\
    & = & \int^\infty_0 f(t) e^{-st} ~ dt + \int^\infty_0 g(t) e^{-st} ~ dt, \\
    & = & \laplace{f} + \laplace{g}.
  \end{eqnarray*}

  
  The Laplace transform of a constant times the function. Suppose $a$ is a constant, then
  \begin{eqnarray*}
    \laplace{a f} & = & \int^\infty_0 a f(t) e^{-st} ~ dt, \\
    & = & a \int^\infty_0 f(t) e^{-st} ~ dt, \\
    & = & a \laplace{f}.
  \end{eqnarray*}


\end{frame}


\begin{frame}
  \frametitle{So What?}

  If we have the Laplace transform of a function, 
  \begin{eqnarray*}
    \laplace{f} & = & F(s),
  \end{eqnarray*}
  then given $F(s)$ we can find $f(t)$. (We will do that later.) More
  importantly, integrals can ``get rid of'' derivatives.

\end{frame}

\subsection{Laplace Transforms of Polynomials}

\begin{frame}
  \frametitle{Laplace Transforms of Polynomials}

  \begin{eqnarray*}
    \begin{array}{rcl@{\hspace{1em}}c@{\hspace{1em}}l}
    \laplace{1} & = & \int^\infty_0 1 \cdot e^{-st} ~ dt
    & = & \frac{1}{s}, \\ [10pt]
    \laplace{t} & = & \int^\infty_0 t \cdot e^{-st} ~ dt
    & = & \frac{1}{s^2}, \\ [10pt]
    \laplace{t^2} & = & \int^\infty_0 t^2 \cdot e^{-st} ~ dt
    & = & \frac{2}{s^3}.      
    \end{array}
  \end{eqnarray*}

  \uncover<2->
  {%
    We can build up tables of these identities. It is then an issue of
    finding the right function in the table.
  }

\end{frame}


\begin{frame}
  \frametitle{Example}

  Find the Laplace transform of 
  \begin{eqnarray*}
    f(t) & = & \redText{3} + \blueText{5t} - \fuchsiaText{8t^2}.
  \end{eqnarray*}

  \uncover<2->
  {
    
    The Laplace transform is
    \begin{eqnarray*}
      \laplace{f} & = & \redText{\frac{3}{s}} + \blueText{\frac{5}{s^2}} - \fuchsiaText{\frac{16}{s^3}}.
    \end{eqnarray*}

  }

\end{frame}


\begin{frame}
  \frametitle{Example}

  Find the Laplace transform of 
  \begin{eqnarray*}
    f(t) & = & \redText{4} - \blueText{3t} + \fuchsiaText{7t^2}.
  \end{eqnarray*}

  \uncover<2->
  {

    The Laplace transform is
    \begin{eqnarray*}
      \laplace{f} & = & \redText{\frac{4}{s}} - \blueText{\frac{3}{s^2}} + \fuchsiaText{\frac{14}{s^3}}.
    \end{eqnarray*}

  }

\end{frame}


\begin{frame}
  \frametitle{Laplace Transform of Higher Order Polynomials}

  In general we have
  \begin{eqnarray*}
    \laplace{t^n} & = & \frac{n!}{s^{n+1}}.
  \end{eqnarray*}

\end{frame}


\begin{frame}
  \frametitle{Example}

  Find the Laplace transform of 
  \begin{eqnarray*}
    f(t) & = & \redText{5 t^3} - \blueText{17 t^5}.
  \end{eqnarray*}

  \uncover<2->
  {

    The Laplace transform is
    \begin{eqnarray*}
      \laplace{f} & = & \redText{5 \frac{3!}{s^4}} - \blueText{17 \frac{5!}{s^6}}.
    \end{eqnarray*}

  }

\end{frame}


\iftoggle{clicker}{%
\begin{frame}
  \frametitle{Clicker Quiz}

   \ifnum\value{clickerQuiz}=1{%

     What is the Laplace transform of $f(t)=3t-4$?

     \begin{tabular}{ll}
       A: & $\frac{1}{s^2}-\frac{1}{s}$ \\ [12pt]
       B: & $\frac{6}{s^2}-\frac{4}{s}$ \\ [12pt]
       C: & $\frac{3}{s^2}-\frac{4}{s}$ 
     \end{tabular}

     \vfill
   }\fi

   \ifnum\value{clickerQuiz}=2{%

     What is the Laplace transform of $f(t)=2t-6$?

     \begin{tabular}{ll}
       A: & $\frac{1}{s^2}-\frac{1}{s}$ \\ [12pt]
       B: & $\frac{2}{s^2}-\frac{6}{s}$ \\ [12pt]
       C: & $\frac{4}{s^2}-\frac{6}{s}$ 
     \end{tabular}

   \vfill
   }\fi

  \ifnum\value{clickerQuiz}=3{%

  \vfill
 }\fi
\end{frame}
}

\subsection{Laplace Transform of Exponentials}

\begin{frame}
  \frametitle{Laplace Transform of Exponentials}

  \begin{eqnarray*}
    \laplace{e^{at}} & = & \int^\infty_0 e^{at} e^{-st} ~ dt, \\
    & = & \int^\infty_0 e^{(a-s)t} ~ dt, \\
    \Rightarrow \laplace{e^{at}} & = & \frac{1}{s-a}.
  \end{eqnarray*}

\end{frame}


\begin{frame}
  \frametitle{Example}

  Find the Laplace transform of 
  \begin{eqnarray*}
    f(t) & = & e^{3t}.
  \end{eqnarray*}

  \uncover<2->
  {
    
    The Laplace transform is
    \begin{eqnarray*}
      \laplace{f} & = & \frac{1}{s-3}.
    \end{eqnarray*}

  }

  \uncover<3->
  {
    Find the Laplace transform of 
    \begin{eqnarray*}
      g(t) & = & e^{-5t}.
    \end{eqnarray*}

  }

  \uncover<4->
  {
    The Laplace transform is
    \begin{eqnarray*}
      \laplace{g} & = & \frac{1}{s+5}.
    \end{eqnarray*}
  }


\end{frame}


\begin{frame}
  \frametitle{Example}

  Find the Laplace transform of 
  \begin{eqnarray*}
    f(t) & = & \redText{4t^2} - \blueText{18 e^{-6t}}.
  \end{eqnarray*}

  \uncover<2->
  {

    The Laplace transform is
    \begin{eqnarray*}
      \laplace{f} & = & \redText{4 \frac{2!}{s^3}} - \blueText{18 \frac{1}{s+6}}.
    \end{eqnarray*}

  }

\end{frame}


\iftoggle{clicker}{%
\begin{frame}
  \frametitle{Clicker Quiz}

   \ifnum\value{clickerQuiz}=1{%

     What is the Laplace transform of $f(t)=i$?

     \begin{tabular}{ll}
       A: & $\frac{1}{s}$ \\ [12pt]
       B: & $\frac{i}{s}$ \\ [12pt]
       C: & $\frac{i}{s^2}$ \\ [12pt]
       D: & Not possible
     \end{tabular}

     \vfill
   }\fi

   \ifnum\value{clickerQuiz}=2{%

     What is the Laplace transform of $f(t)=i$?

     \begin{tabular}{ll}
       A: & $\frac{1}{s}$ \\ [12pt]
       B: & $\frac{i}{s}$ \\ [12pt]
       C: & $\frac{i}{s^2}$ \\ [12pt]
       D: & Not possible
     \end{tabular}

   \vfill
   }\fi

  \ifnum\value{clickerQuiz}=3{%

  \vfill
 }\fi
\end{frame}
}


\subsection{Laplace Transform of Sines and Cosines}

\begin{frame}
  \frametitle{Sines and Cosines}

  Note that
  \begin{eqnarray*}
    \laplace{e^{iat}} & = & \frac{1}{s-ai}, \\
    & = & \frac{s+ai}{(s-ai)(s+ai)}, \\
    & = & \frac{s+ai}{s^2+a^2}, \\
    \only<2>{& = & \frac{s}{s^2+a^2} + i \frac{a}{s^2+a^2}. \\}
    \uncover<3->{& = & \redText{\frac{s}{s^2+a^2}} + i \blueText{\frac{a}{s^2+a^2}}.}
  \end{eqnarray*}


  \uncover<4->
  {

    At the same time, though,
    \begin{eqnarray*}
      \laplace{e^{iat}} & = & \laplace{\cos(at) + i \sin(at)}, \\
      & = & \redText{\laplace{\cos(at)}} + i \blueText{\laplace{\sin(at)}}.
    \end{eqnarray*}

  }


\end{frame}


\begin{frame}
  \frametitle{Sines and Cosines}

  The Laplace transform of the sine and cosine is the following:
  \begin{eqnarray*}
    \laplace{\cos(at)} & = & \frac{s}{s^2+a^2}, \\
    \laplace{\sin(at)} & = & \frac{a}{s^2+a^2}.
  \end{eqnarray*}


\end{frame}


\begin{frame}
  \frametitle{Example}

  Find the Laplace transform of 
  \begin{eqnarray*}
    f(t) & = & \redText{5t^4} + \blueText{2\sin(4t)} - \fuchsiaText{3 e^{8t}} + 7 \cos(5t).
  \end{eqnarray*}

  \uncover<2->
  {

    The Laplace transform is
    \begin{eqnarray*}
      \laplace{f} & = & \redText{5 \frac{4!}{s^5}} + \blueText{2 \frac{4}{s^2+16}}  - \fuchsiaText{3\frac{1}{s-8}} + 7 \frac{s}{s^2+25}.
    \end{eqnarray*}

  }

\end{frame}

\subsection{More Identities}

\begin{frame}
  \frametitle{More Identities}

  \begin{eqnarray*}
    \laplace{t f(t)} & = & \int^\infty_0 t f(t) e^{-st} ~ dt, \\
    & = & \int^\infty_0 f(t) \lp -\deriv{~}{s} e^{-st} \rp ~ dt, \\
    & = & -\deriv{~}{s} \int^\infty_0 f(t) e^{-st}  ~ dt, \\
    & = & -\deriv{~}{s} \laplace{f}(s).
  \end{eqnarray*}

\end{frame}

\begin{frame}
  \frametitle{Moar Moar Identities!}
  \begin{columns}
    \column{.5\textwidth} 
    \includegraphics[height=4cm]{img/cagemoar}


    \column{.5\textwidth}
    \begin{eqnarray*}
      \laplace{t^2 f(t)} & = & \frac{d^2}{ds^2} \laplace{f}, \\
      \uncover<2->
      {
        \laplace{t^3 f(t)} & = & -\frac{d^3}{ds^3} \laplace{f}, \\
      }
      \uncover<3->
      {
        \vdots             &   & \vdots \\
        \laplace{t^n f(t)} & = & (-1)^n \frac{d^n}{ds^n} \laplace{f}.
      }
    \end{eqnarray*}
  \end{columns}
  
\end{frame}


\begin{frame}
  \frametitle{Example}
    \begin{eqnarray*}
      f(t) & = & \redText{7t \sin(4t)} + \blueText{2t^3e^{-4t}}, \\
      \uncover<2->
      {
        \laplace{f} & = &- \redText{\deriv{~}{s} \frac{28}{s^2+16}} \blueText{- \frac{d^3}{ds^3} \frac{2}{(s+4)}}, \\
         & = & \redText{\frac{56s}{\lp s^2+16\rp^2}} + \blueText{12 \frac{1}{(s+4)^4}}.
      }
    \end{eqnarray*}
\end{frame}

\begin{frame}
  \frametitle{Moar Moar Moar Identities!}
  \begin{columns}
    \column{.1\textwidth} 
    \includegraphics[height=4cm]{img/moar_18}


    \column{.9\textwidth}
    \begin{eqnarray*}
      \laplace{e^{at} f(t)} & = & \int^\infty_0 e^{at} f(t) e^{-st} ~ dt, \\
      \uncover<2->
      {
        & = & \int^\infty_0 f(t) e^{-(s-a)t} ~ dt, \\
      }
      \uncover<3->
      {
        & = & \laplace{f}(s-a).
      }
    \end{eqnarray*}
  \end{columns}
  
\end{frame}


\iftoggle{clicker}{%
\begin{frame}
  \frametitle{Clicker Quiz}

   \ifnum\value{clickerQuiz}=1{%

     Was that Billy Idol?

     \begin{tabular}{ll}
       A: & Yes \\ [12pt]
       B: & Who? \\ [12pt]
       C: & No, that would be lame.
     \end{tabular}

     \vfill
   }\fi

   \ifnum\value{clickerQuiz}=2{%

     Was that Billy Idol?

     \begin{tabular}{ll}
       A: & Yes \\ [12pt]
       B: & Who? \\ [12pt]
       C: & No, that would be lame.
     \end{tabular}


   \vfill
   }\fi

  \ifnum\value{clickerQuiz}=3{%

  \vfill
 }\fi
\end{frame}
}


\begin{frame}
  \frametitle{Example}
    \begin{eqnarray*}
      f(t) & = & \redText{3t^4 e^{2t}} - \blueText{4 e^{-5t}\sin(3t)}, \\
      \uncover<2->
      {
        \laplace{f} & = & \redText{3 \laplace{t^4}(s-2)} - \blueText{4 \laplace{\sin(3t)}(s+5)}, \\
      }
      \uncover<3->
      {
        & = & \redText{3 \frac{4!}{(s-2)^5}} - \blueText{4 \frac{3}{(s+5)^2+9}}.
      }
    \end{eqnarray*}
\end{frame}


\begin{frame}
  \frametitle{Example}
    \begin{eqnarray*}
      f(t) & = & \redText{2 e^{-4t}\cos(3t)} - \blueText{5 t^2 \sin(2t)}, \\
      \uncover<2->
      {
        \laplace{f} & = & \redText{3 \laplace{\cos(3t)}(s+4)} - \blueText{5 (-1)^2 \frac{d^2}{ds^2} \laplace{\sin(2t)}}, \\
      }
      \uncover<3->
      {
        & = & \redText{2 \frac{s+4}{(s+4)^2+9}} + \blueText{20 \lp -8(s^2+4)^{-3}s^2 + (s^2+4)^{-2} \rp}.
      }
    \end{eqnarray*}
\end{frame}


\begin{frame}
  \frametitle{Oh By The Way....}

  \begin{definition}[The hyperbolic sine and hyperbolic cosine:]
    \begin{eqnarray*}
      \sinh(t) & = & \frac{e^t-e^{-t}}{2}, \\
      \cosh(t) & = & \frac{e^t+e^{-t}}{2}.
    \end{eqnarray*}
  \end{definition}

  \uncover<2->
  {
    Why? Because it makes other things easier.
    \begin{eqnarray*}
      \deriv{~}{t} \cosh(t) & = & \sinh(t), \\
      \deriv{~}{t} \sinh(t) & = & \cosh(t).
    \end{eqnarray*}
    (Think about solutions to DEs!)
  }

  
\end{frame}


\begin{frame}
  \frametitle{Oh By The Way....}

  It makes dealing with the inverse Laplace Transform much easier in some cases:
  \begin{eqnarray*}
    \laplace{\sinh(at)} & = & \frac{a}{s^2-a^2}, \\
    \laplace{\cosh(at)} & = & \frac{s}{s^2-a^2}.
  \end{eqnarray*}

  (We will see why soon.)

  But mostly it is because \textbf{some engineers like to use it}. You
  have to be prepared to come across this in the future.
  
\end{frame}



% LocalWords:  Clarkson pausesection hideothersubsections
