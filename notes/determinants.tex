\part{Determinants}
\lecture{Determinants}{Determinants}
\section{Determinants}

\title{Ordinary Differential Equations}
\subtitle{Math 232 - Determinants}
\date{21 Oct 2013}

\begin{frame}
  \titlepage
\end{frame}

\begin{frame}
  \frametitle{Outline}
  \tableofcontents[currentsection]
\end{frame}


\subsection{Determinants}
\iftoggle{clicker}{%
\begin{frame}
  \frametitle{Clicker Quiz}

       \ifnum\value{clickerQuiz}=1{%
         Which matrix is the inverse of 
         \begin{eqnarray*}
           \arrayTwo{0}{1}{2}{1}?
         \end{eqnarray*}

         \vfill
 
         \begin{eqnarray*}
           \begin{array}{llcr}
             A: &  \arrayTwo{1/2}{1/2}{1}{0}  \\ [12pt]
             B: &  \arrayTwo{-1/2}{1/2}{1}{0}   \\ [12pt] 
             C: &  \arrayTwo{0}{1/2}{1}{1/2} \\  [12pt]
             D: &  \arrayTwo{0}{1}{1/2}{1}
           \end{array}
         \end{eqnarray*}

 
         \vfill
 
       }\fi

         \ifnum\value{clickerQuiz}=2{%
         Which matrix is the inverse of 
         \begin{eqnarray*}
           \arrayTwo{1}{1}{2}{0}?
         \end{eqnarray*}

         \vfill
 
         \begin{eqnarray*}
           \begin{array}{llcr}
             A: &  \arrayTwo{1/2}{1/2}{1}{0}  \\ [12pt]
             B: &  \arrayTwo{-1}{1}{1/2}{0}   \\ [12pt] 
             C: &  \arrayTwo{0}{1/2}{1}{-1/2} \\  [12pt]
             D: &  \arrayTwo{1/2}{0}{1}{1}
           \end{array}
         \end{eqnarray*}


         
         \vfill
        
       }\fi
 

      \ifnum\value{clickerQuiz}=3{%
       Calculate the inverse of 
        \begin{eqnarray*}
         \arrayTwo{2}{0}{1}{1}.
        \end{eqnarray*}

        \vfill
 
         \begin{eqnarray*}
          \begin{array}{llcr}
            A: &  \arrayTwo{1/2}{0}{-1/2}{1}\\ 
            B: &  \arrayTwo{1/2}{0}{1}{1}\\ 
            C: &  \arrayTwo{1/2}{0}{1/2}{1}\\ 
            D: &  \arrayTwo{-1/2}{0}{-1}{1}\\ 
          \end{array}
        \end{eqnarray*}

        \vfill

     }\fi
    \vfill
    \vfill
    \vfill

\end{frame}

}


\begin{frame}
  \frametitle{Definition of the Determinant}

  The determinant of a $2\times 2$ matrix is defined to be
  \begin{eqnarray*}
    |A| & = & \detTwo{a}{b}{c}{d} \\
    & = & ad - bc
  \end{eqnarray*}

\end{frame}


\begin{frame}
  \frametitle{Why?}

  Suppose I want to solve the following system of equations:
  \begin{eqnarray*}
    \arrayTwo{a}{b}{c}{d} \vecTwo{x}{y} & = & \vecTwo{\#}{\#} \\
    \uncover<1->{%
      \startRowOpsTwo
      \oneRowOpsTwo{a}{b}{1}{0}
      \oneRowOpsTwo{c}{d}{0}{1}
      \stopRowOps} \\
    \uncover<1->{%
      \stateTwo{}{-cR_1+aR_2}
      \startRowOpsTwo
      \oneRowOpsTwo{a}{b}{1}{0}
      \oneRowOpsTwo{0}{ad-bc}{-c}{a}
      \stopRowOps
    }
  \end{eqnarray*}

  \uncover<1->
  {
    If $ad-bc$ is zero then there is no inverse. The linear system
    does not have a unique solution.
  }

  \uncover<1->
  {
    In general if the determinant of a matrix is zero its inverse does
    not exist. 
  }

\end{frame}

\begin{frame}
  \frametitle{Example}

  \begin{eqnarray*}
    A & = & \arrayTwo{2}{-1}{3}{1}, \\
    |A| & = & 2(1)-3(-1), \\
    & = & 5.
  \end{eqnarray*}

\end{frame}

\iftoggle{clicker}{%
\begin{frame}
  \frametitle{Clicker Quiz}
    
      \ifnum\value{clickerQuiz}=1{%

        \vfill

        What is determinant of the following matrix
        \begin{eqnarray*}
          M & = & \arrayTwo{3}{-1}{4}{1}?
        \end{eqnarray*}
        
        \begin{eqnarray*}
          \begin{array}{llcr}
            A: & |M| &  = & 11 \\ 
            B: & |M| &  = &  1 \\ 
            C: & |M| &  = & -1 \\
            D: & |M| &  = &  7
          \end{array}
        \end{eqnarray*}

          \vfill

     }\fi

     \ifnum\value{clickerQuiz}=2{%

        \vfill

        What is determinant of the following matrix
        \begin{eqnarray*}
          M & = & \arrayTwo{2}{-1}{5}{1}?
        \end{eqnarray*}
        
        \begin{eqnarray*}
          \begin{array}{llcr}
            A: & |M| &  = &  7 \\ 
            B: & |M| &  = & -3 \\ 
            C: & |M| &  = &  3 \\
            D: & |M| &  = & 10
          \end{array}
        \end{eqnarray*}

          \vfill


     }\fi

      \ifnum\value{clickerQuiz}=3{%
        \vfill
      What is determinant of the following matrix
        \begin{eqnarray*}
          M & = & \arrayTwo{-2}{1}{5}{2}?
        \end{eqnarray*}

        \begin{eqnarray*}
          \begin{array}{llcr}
            A: & |M| &  = &  1 \\ 
            B: & |M| &  = & -9 \\ 
            C: & |M| &  = &  -1 \\
            D: & |M| &  = & 9
          \end{array}
        \end{eqnarray*}

          \vfill


     }\fi

    \vfill
    \vfill
    \vfill

\end{frame}

}



\subsection{Higher Dimensions}

\begin{frame}
  \frametitle{Higher Dimensions}

  Definition: Given a \textbf{square} matrix:
  \begin{eqnarray*}
    A & = & 
    \left[
      \begin{array}{rrr|r|rr}
        a_{11} & a_{12} & \cdots & {\color{red}a_{1j}}  & \cdots & a_{1n} \\
        \vdots &       &       & {\color{red}\vdots} &        & \vdots \\ \hline
        {\color{green}a_{i1}} & {\color{green}a_{i2}}  & {\color{green}\cdots} & {\color{blue}a_{ij}} & {\color{green}\cdots} & {\color{green}a_{in}} \\ \hline
        \vdots &       &       & {\color{red}\vdots} &        & \vdots \\
        a_{n1} & a_{n2} & \cdots & {\color{red}a_{nj}}  & \cdots & a_{nn} \\
      \end{array}
    \right]
  \end{eqnarray*}

  The \textbf{minor}, $M_{ij}$ of a matrix is found by removing row
  $i$ and column $j$.

\end{frame}


\begin{frame}
  \frametitle{The Cofactor of a matrix}

  The \textbf{cofactor}, $c_{ij}$, of a matrix is defined to be 
  \begin{eqnarray*}
    c_{ij} & = & (-1)^{i+j}\left| M_{ij} \right|.
  \end{eqnarray*}

  Note:
  \begin{eqnarray*}
    \left| M_{ij} \right|
  \end{eqnarray*}
  is the determinant of $M_{ij}$.

\end{frame}


\begin{frame}
  \frametitle{Determinants in Higher Dimensions}

  The determinant of a square matrix with more than two rows is
  defined to be
  \begin{eqnarray*}
    \left| A \right| & = & \sum^n_{j=1} a_{ij} c_{ij}.
  \end{eqnarray*}
  This is the determinant by expansion along the $i^{\mathrm{th}}$
  row. It is the same for any $i$ where $1\leq i \leq n$.

  An equivalent definition:
  \begin{eqnarray*}
    \left| A \right| & = & \sum^n_{i=1} a_{ij} c_{ij}.
  \end{eqnarray*}
  This is the determinant by expansion along the $j^{\mathrm{th}}$
  column.

  

\end{frame}


\subsection{Examples of Determinants}

\begin{frame}
  \frametitle{Example}

  Find the determinant of the following matrix:

  \only<1>{%
    \begin{eqnarray*}
      \mathrm{det}
      \arrayThree{3}{2}{-1}{4}{6}{2}{1}{3}{1}
    \end{eqnarray*}
  }

  \only<1>{%
    \begin{eqnarray*}
      \mathrm{det}
      \arrayThree{{\color{red}3}}{\color{red}{2}}{\color{red}{-1}}{4}{6}{2}{1}{3}{1}
    \end{eqnarray*}
  }  

  \only<1>{%
    \begin{eqnarray*}
      \mathrm{det}
      \arrayThree{{\color{blue}(+)}{\color{red}3}}{{\color{blue}(-)}{\color{red}2}}{{\color{blue}(+)}{\color{red}-1}}{4}{6}{2}{1}{3}{1}
    \end{eqnarray*}
  }  


\end{frame}

\begin{frame}
  \frametitle{Example}

  Find the determinant of the following matrix:

  \begin{eqnarray*}
    \mathrm{det}
    \arrayThree{{\color{red}3}}{{\color{red}2}}{{\color{red}-1}}{4}{6}{2}{1}{3}{1}
    & = & 
    {\color{red}3} \detTwo{6}{2}{3}{1} - {\color{red}2} \detTwo{4}{2}{1}{1}
    + {\color{red}(-1)} \detTwo{4}{6}{1}{3} \\
    & = & 3 (6-6) - 2(4-2) + (-1) (12-6) \\
    & = & -10
  \end{eqnarray*}

\end{frame}




\begin{frame}
  \frametitle{Example}
  Find the determinant of the following matrix:

  \only<1>{%
    \begin{eqnarray*}
      \mathrm{det}
      \startRowFour
      \oneRowFour{4}{3}{-1}{0}
      \oneRowFour{1}{0}{1}{7}
      \oneRowFour{0}{1}{2}{6}
      \oneRowFour{2}{0}{3}{2}
      \stopRowOps
    \end{eqnarray*}
  }


  \only<1>{%
    \begin{eqnarray*}
      \mathrm{det}
      \startRowFour
      \oneRowFour{4}{{\color{red}3}}{-1}{0}
      \oneRowFour{1}{{\color{red}0}}{1}{7}
      \oneRowFour{0}{{\color{red}1}}{2}{6}
      \oneRowFour{2}{{\color{red}0}}{3}{2}
      \stopRowOps
    \end{eqnarray*}
  }

  \only<1>{%
    \begin{eqnarray*}
      \mathrm{det}
      \startRowFour
      \oneRowFour{4}{{\color{green}(-)}{\color{red}3}}{-1}{0}
      \oneRowFour{1}{{\color{green}(+)}{\color{red}0}}{1}{7}
      \oneRowFour{0}{{\color{green}(-)}{\color{red}1}}{2}{6}
      \oneRowFour{2}{{\color{green}(+)}{\color{red}0}}{3}{2}
      \stopRowOps
    \end{eqnarray*}
  }


\end{frame}


\begin{frame}
  \frametitle{Example}

  Find the determinant of the following matrix:

  \begin{eqnarray*}
    \mathrm{det}
    \startRowFour
    \oneRowFour{4}{{\color{red}3}}{-1}{0}
    \oneRowFour{1}{{\color{red}0}}{1}{7}
    \oneRowFour{0}{{\color{red}1}}{2}{6}
    \oneRowFour{2}{{\color{red}0}}{3}{2}
    \stopRowOps
    & = & 
    {\color{red}-3} \color{green}{\detThree{1}{1}{7}{0}{2}{6}{2}{3}{2}}
    + {\color{red}0} \color{blue}{\detThree{4}{-1}{0}{0}{2}{6}{2}{3}{2}} \\
    & & 
    - {\color{red}1} \color{orange}{\detThree{4}{-1}{0}{1}{1}{7}{2}{3}{2}}
    + {\color{red}0} \color{cyan}{\detThree{4}{-1}{0}{1}{1}{7}{0}{2}{6}} \\
    & = & 178
  \end{eqnarray*}


\end{frame}

\subsection{Properties of Determinants}

\begin{frame}
  \frametitle{Properties of Determinants}

  \begin{eqnarray*}
    \mathrm{det}(AB) & = & \mathrm{det}(A) \cdot \mathrm{det}(B) \\
    \mathrm{det}(I_n) & = & 1
  \end{eqnarray*}

  \uncover<1->{%
    Note:
    \begin{eqnarray*}
      A A^{-1} & = & I_n \\
      \mathrm{det}(A A^{-1}) & = & \mathrm{det}(I_n) \\
      \mathrm{det}(A) \mathrm{det}(A^{-1}) & = & 1 \\
      \mathrm{det}(A^{-1}) & = & \frac{1}{\mathrm{det}(A)}
    \end{eqnarray*}
  }

  \uncover<1->{%
    Read Cramer's rule pp. 160-162.
  }

\end{frame}


\subsection{Vector Spaces}

\begin{frame}
  \frametitle{Vector Spaces}

  Definition of a vector space, V:
  \begin{itemize}
  \item Elements in V are called ``vectors.''
  \item If $\vec{x}$, $\vec{y}$ are members of V then
    $\vec{x}+\vec{y}$ is in V.
  \item If $c$ is a real number and $\vec{x}$ is in V then so is
    $c\vec{x}$.
  \item There is a ``zero vector'' where $\vec{x}+\vec{0}=\vec{x}$ for
    every member of V.
  \item For any $\vec{x}$ in V there is another $\vec{y}$ in V where
    $\vec{x}+\vec{y}=\vec{0}$. ($\vec{y}$ is called ``$-\vec{x}$.'')
  \end{itemize}

  See p. 168 for definition and properties.

\end{frame}

\begin{frame}
  \frametitle{Example}

  $\mathbb{R}^2$ is a vector space.
  \only<1->{Suppose that $x$,  $y$, $u$, $v$, and $c$ are real numbers.}
  \begin{eqnarray*}
    \only<1->{\vecTwo{x}{y} & \in & \mathbb{R}^2} \\
    \only<1->{\vecTwo{x}{y} + \vecTwo{u}{v} & = & \vecTwo{x+u}{y+v}} \\
    \only<1->{c \vecTwo{x}{y} & = & \vecTwo{cx}{cy}} \\
    \only<1->{\vec{0} & = & \vecTwo{0}{0}} \\
    \only<1->{\vecTwo{x}{y} + \vecTwo{-x}{-y} & = & \vec{0}}
  \end{eqnarray*}

\end{frame}


\begin{frame}
  \frametitle{Example - Continuous Functions}

  The set of continuous functions ($C_0$) is a vector space.
  Suppose $f(t),~g(t)~\in~C_o$:
  \begin{itemize}
  \item<1-> $f(t)+g(t)$ is continuous.
  \item<1-> $c f(t)$ is continuous if $c$ is a real number.
  \item<1-> $h(t)=0$ is the zero vector.
  \item<1-> For any function
    \begin{eqnarray*}
      f(t) + (-1)f(t) & = & (1 + (-1))f(t), \\
      & = & 0 f(t), \\
      & = & 0.
    \end{eqnarray*}
  \end{itemize}

\end{frame}


\begin{frame}
  \frametitle{Example - Quadratic Functions}

  The set of polynomials of at most degree 2 (the quadratic functions)
  is a vector space:

  \uncover<1->{%
    \begin{eqnarray*}
      h(x) & = & a x^2 + bx + c, \\
      g(x) & = & dx^2 + ex + f,
    \end{eqnarray*}
    where $a$, $b$, $c$, $d$, $e$, and $f$ are real numbers.
  }

  \only<2>{%
    \color{red}{Addition:}{~}
    \begin{eqnarray*}
      h(x)+g(x) & = & (a+d) x^2 + (b+e) x + (c+f)
    \end{eqnarray*}
    is a polynomial of at most degree 2.
  }

  \only<2>{%
    \color{red}{Scalar multiplication}, if $r$ is a real number:
    \begin{eqnarray*}
      r h(x) & = & ra x^2 + rb x + rc
    \end{eqnarray*}
    is a polynomial of at most degree 2.
  }

  \only<2>{%
    The function $h(x) = 0$ is a polynomial of at most degree 2 and is
    the \color{red}{``zero vector.''}  
  }


\end{frame}


\begin{frame}
  \frametitle{Counter Example}

  The set of straight lines through $(0,1)$ is not a vector space.

  \uncover<2->{%
    Any polynomial, $h(x)$ or $g(x)$, can be expressed as
    \begin{eqnarray*}
      h(x) & = & ax + 1 \\
      g(x) & = & bx + 1
    \end{eqnarray*}
  }

  \uncover<2->{%
    It fails the addition test:
    \begin{eqnarray*}
      h(x) + g(x) & = & (a+b)x + 2
    \end{eqnarray*}
    is a straight line but does not go through $(0,1)$.
  }

\end{frame}





% LocalWords:  Clarkson pausesection hideothersubsections
