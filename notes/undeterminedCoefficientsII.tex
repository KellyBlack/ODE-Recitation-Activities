\part{More-Undetermined-Coefficients}
\lecture{More Undetermined Coefficients}{More-Undetermined-Coefficients}
\section{More Undetermined Coefficients}

\title{Ordinary Differential Equations}
\subtitle{Math 232 - Revenge of the Method of Undetermined Coefficients}
\date{17 Oct 2012}

\begin{frame}
  \titlepage
\end{frame}

\begin{frame}
  \frametitle{Outline}
  \tableofcontents[ currentsection ]
\end{frame}


\subsection{Non-homogeneous Equations (continued)}


\iftoggle{clicker}{%
\begin{frame}
  \frametitle{Clicker Quiz}

      \ifnum\value{clickerQuiz}=1{%

        \vfill

        Determine the roots to the characteristic equation for
        \begin{eqnarray*}
          y'' + 3y' + 2y & = & 0.
        \end{eqnarray*}


        \vfill

        \begin{tabular}{ll}
          A: & $r_1 = -3/2+\sqrt{3}/2 i$, $-3/2-\sqrt{3}/2i$ \\ [12pt]
          B: & $r_1 = -3/2+\sqrt{3}/2$, $-3/2-\sqrt{3}/2$ \\
          C: & $r_1=-2$, $r_2=-1$ \\ [12pt]
          D: & $r_1=2$, $r_2=1$ \\ [12pt]
        \end{tabular}

        \vfill

      }\fi

      \ifnum\value{clickerQuiz}=2{%

        \vfill

        Determine the roots to the characteristic equation for
        \begin{eqnarray*}
          y'' + 3y' + 2y & = & 0.
        \end{eqnarray*}


        \vfill

        \begin{tabular}{ll}
          A: & $r_1=-2$, $r_2=-1$ \\ [12pt]
          B: & $r_1=2$, $r_2=1$ \\ [12pt]
          C: & $r_1 = -3/2+\sqrt{3}/2 i$, $-3/2-\sqrt{3}/2i$ \\ [12pt]
          D: & $r_1 = -3/2+\sqrt{3}/2$, $-3/2-\sqrt{3}/2$ \\
        \end{tabular}

        \vfill

     }\fi
   
     \ifnum\value{clickerQuiz}=3{%
         \vfill

        Determine the roots to the characteristic equation for
        \begin{eqnarray*}
          y'' + 3y' + 2y & = & 0.
        \end{eqnarray*}


        \vfill

        \begin{tabular}{ll}
          A: & $r_1=-2$, $r_2=-1$ \\ [12pt]
          B: & $r_1=2$, $r_2=1$ \\ [12pt]
          C: & $r_1 = -3/2+\sqrt{3}/2 i$, $-3/2-\sqrt{3}/2i$ \\ [12pt]
          D: & $r_1 = -3/2+\sqrt{3}/2$, $-3/2-\sqrt{3}/2$ \\
        \end{tabular}

        \vfill

    }\fi
  

\end{frame}
}


\begin{frame}
  \frametitle{Example}

  \begin{eqnarray*}
    y'' + 3y' + 2y & = & 4 e^{-t}.
  \end{eqnarray*}

  \uncover<2->
  {
    Homogeneous solution:
    \begin{eqnarray*}
      y_h'' + 3y_h' + 2y_h & = & 0, \\
      r^2 + 3r + 2 & = & 0, \\
      (r+2)(r+1) & = & 0, \\
      \blueText{y_h} & \blueText{=} & \blueText{C_1 e^{-2t} + C_2 e^{-t}}.
    \end{eqnarray*}
  }

\end{frame}

\begin{frame}
  \frametitle{Particular Solution}


  \begin{eqnarray*}
    y_p'' + 3y_p' + 2y_p & = & 4 e^{-t}.
  \end{eqnarray*}

  \uncover<2->{%
    \begin{eqnarray*}
      y_p & = & A t e^{-t} \\
      y_p' & = & \redText{ A e^{-t} - Ate^{-t}} \\
      y_p'' & = & \blueText{-2A e^{-t} + At e^{-t}}
    \end{eqnarray*}
  }

  \uncover<3->{%
    \begin{eqnarray*}
      \blueText{-2A e^{-t} + At e^{-t}} + 3 \redText{\lp A e^{-t} - Ate^{-t} \rp}
      + 2 A t e^{-t} & = & 4 e^{-t}, \\
      A e^{-t} & = & 4 e^{-t} \\
      A & = & 4.
    \end{eqnarray*}
  }

\end{frame}

\begin{frame}
  \frametitle{The Solution}

  \begin{eqnarray*}
    y(t) & = & \blueText{y_h(t)} + \redText{y_p(t)}, \\
         & = & \blueText{C_1 e^{-2t} + C_2 e^{-t}} + \redText{4te^{-t}}.
  \end{eqnarray*}

  \textit{Apply boundary conditions here if they are given in the original problem statement.}

\end{frame}

\begin{frame}
  \frametitle{Example}

  \begin{eqnarray*}
    y'' + 4y' + 4y & = & 5 e^{-2t}.
  \end{eqnarray*}

  \uncover<2->
  {
    Homogeneous solution:
    \begin{eqnarray*}
      r^2 + 4r + 4 & = & 0, \\
      (r+2)^2 & = & 0, \\
      \redText{y_h} & \redText{=} & \redText{C_1 e^{-2t} + C_2 t e^{-2t}}.
    \end{eqnarray*}
  }

\end{frame}

\begin{frame}
  \frametitle{Particular Solution}
  
  \begin{eqnarray*}
    y_p & = & A t^2 e^{-2t}, \\
    y_p' & = & -2 A t^2 e^{-2t} + A 2t e^{-2t}, \\
    y_p'' & = & 4 A t^2 e^{-2t} - 8A t e^{-2t} + 2 A e^{-2t}.
  \end{eqnarray*}

  \uncover<2->{%
    \begin{eqnarray*}
      \redText{y_p''} + 4\blueText{y_p'} + 4y_p & = & 5 e^{-2t}, \\
      \redText{4 A t^2 e^{-2t} - 8A t e^{-2t} + 2 A e^{-2t}} & & \\
      + 4 \blueText{\lp -2 A t^2 e^{-2t} + A 2t e^{-2t} \rp} & & \\
      + 4 A t^2 e^{-2t} & = & 5 e^{-2t} \\
      2 A e^{-2t} & = & 5 e^{-2t} \\
      A & = & \frac{5}{2}, \\
      \blueText{y_p(t)} & \blueText{=} & \blueText{\frac{5}{2} t^2 e^{-2t}}.
    \end{eqnarray*}
  }


\end{frame}


\begin{frame}
  \frametitle{The General Solution}
  Solution:
  \begin{eqnarray*}
    y(t) & = & \redText{y_h(t)} + \blueText{y_p(t)}, \\
         & = & \redText{C_1 e^{-2t} + C_2 t e^{-2 t}} + \blueText{\frac{5}{2} t^2 e^{-2t}}.
  \end{eqnarray*}

  \textit{Apply boundary conditions here if they are given in the original problem statement.}

\end{frame}


\subsection{Cauchy-Euler Equations}

\iftoggle{clicker}{%
\begin{frame}
  \frametitle{Clicker Quiz}

      \ifnum\value{clickerQuiz}=1{%

        \vfill

        Is the equation
        \begin{eqnarray*}
          t^2 y'' + 3t y' + 2y & = & 0
        \end{eqnarray*}
        linear or nonlinear?

        \vfill

        \begin{tabular}{ll}
          A: & Linear \\ [12pt]
          B: & Not Linear \\ [12pt]
        \end{tabular}

        \vfill

      }\fi

      \ifnum\value{clickerQuiz}=2{%

        \vfill

        Is the equation
        \begin{eqnarray*}
          t^2 y'' + 3t y' + 2y & = & 0
        \end{eqnarray*}
        linear or nonlinear?

        \vfill

        \begin{tabular}{ll}
          A: & Linear \\ [12pt]
          B: & Not Linear \\ [12pt]
        \end{tabular}

        \vfill

     }\fi
   
     \ifnum\value{clickerQuiz}=3{%
       \vfill

        Is the equation
        \begin{eqnarray*}
          t^2 y'' + 3t y' + 2y & = & 0
        \end{eqnarray*}
        linear or nonlinear?

        \vfill

        \begin{tabular}{ll}
          A: & Linear \\ [12pt]
          B: & Not Linear \\ [12pt]
        \end{tabular}

        \vfill

    }\fi
  

\end{frame}
}



\begin{frame}
  \frametitle{Cauchy-Euler Equations}

  \begin{eqnarray*}
    a t^2 y'' + b t y' + c y & = & 0.
  \end{eqnarray*}

  \uncover<2->{%
    Assume that the solution is in the form of
    \begin{eqnarray*}
      y & = & A t^r \\
      y' & = & A r t^{r-1} \\
      y'' & = & A r (r-1) t^{r-2}
    \end{eqnarray*}
  }


\end{frame}


\begin{frame}
  \frametitle{Cauchy-Euler Equations}

  \begin{eqnarray*}
    a t^2 y'' + b t y' + c y & = & 0.
  \end{eqnarray*}

  \uncover<2->{%
    Substitute back into the equation to get
    \begin{eqnarray*}
      a t^2 A r (r-1) t^{r-2} + b t A r t^{r-1} + c A t^r & = & 0, \\
      a A r (r-1) t^{r} + b A r t^{r} + c A t^r & = & 0,  \\
      A t^r \lp a r (r-1) + br + c \rp & = & 0, \\
      a r (r-1) + br + c & = & 0.
    \end{eqnarray*}
  }

  \uncover<3->{%
    The solution is of the form
    \begin{eqnarray*}
      y & = & C_1 t^{r_1} + C_2 t^{r_2}.
    \end{eqnarray*}
  }
    

\end{frame}


\begin{frame}
  \frametitle{Example}

  \begin{eqnarray*}
    t^2 y'' + t y' - 3y & = & 0.
  \end{eqnarray*}

  \uncover<2->{%
    Assume that the solution is in the form of $y=At^r$:
    \begin{eqnarray*}
      r(r-1) + r - 3 & = & 0, \\
      r^2 - 3 & = & 0, \\
      r & = & \pm \sqrt{3}.
    \end{eqnarray*}
  }

  \uncover<3->{%
    The solution is
    \begin{eqnarray*}
      y & = & C_1 t^{\sqrt{3}} + C_2 t^{-\sqrt{3}}.
    \end{eqnarray*}
  }

\end{frame}


\begin{frame}
  \frametitle{Example}


  \begin{eqnarray*}
    t y'' - 4 y' + 3 \frac{y}{t} & = & 0, \\
    \uncover<2->
    {
      t^2 y'' - 4t y' + 3y & = & 0
    }
  \end{eqnarray*}

  \uncover<3->
  {
    Assume that the solution is in the form of $y=At^r$:
    \begin{eqnarray*}
      r(r-1) - 4r + 3 & = & 0, \\
      r^2 - 5r + 3 & = & 0, \\
      r & = & \frac{5 \pm \sqrt{13}}{2}
    \end{eqnarray*}
  }

  \uncover<4->
  {
    The solution is
    \begin{eqnarray*}
      y & = & C_1 t^{\frac{5 + \sqrt{13}}{2}} + C_2 t^{\frac{5 - \sqrt{13}}{2}}.
    \end{eqnarray*}
  }


\end{frame}


\subsection{Reduction of Order}



\begin{frame}
  \frametitle{Example}

  \begin{eqnarray*}
    y'' + 2 y' + y & = & 2 e^{-t}.
  \end{eqnarray*}

  The characteristic equation for the homogeneous solution is
  \begin{eqnarray*}
    r^2 + 2r + 1 & = & 0, \\
    (r+1)^2 & = & 0, \\
    r & = & -1.
  \end{eqnarray*}

  The homogeneous solution is
  \begin{eqnarray*}
    \redText{y_h} & \redText{=} & \redText{C_1 e^{-t} + C_2 t e^{-t}}.
  \end{eqnarray*}

\end{frame}


\begin{frame}
  \frametitle{Reduction of Order}

  Go back to the original ODE,
  \begin{eqnarray*}
    y'' + 2 y' + y & = & 2 e^{-t}.
  \end{eqnarray*}

  Let
  \begin{eqnarray*}
    y & = & u(t) e^{-t}.
  \end{eqnarray*}

  \uncover<2->{%
    Take the derivatives
    \begin{eqnarray*}
      y' & = & \redText{u'(t) e^{-t} - u(t) e^{-t}}. \\
      y'' & = & \blueText{u''(t) e^{-t} - 2 u'(t) e^{-t} + u(t) e^{-t}}.
    \end{eqnarray*}
  }

\end{frame}

\begin{frame}

  Go back to the original ODE,
  \begin{eqnarray*}
    y'' + 2 y' + y & = & 2 e^{-t}.
  \end{eqnarray*}
  Let $y=\redText{u(t)}e^{-t}$ and substitute into the equation.

  \uncover<2->{%
    Substitute back into the original equation:
    \begin{eqnarray*}
      \blueText{u'' e^{-t} - 2 u' e^{-t} + u e^{-t}} + 2 \lp \redText{u' e^{-t} - ue^{-t}}\rp + u e^{-t} & = & 2 e^{-t} \\
      u'' e^{-t} & = & 2 e^{-t} \\
      \fuchsiaText{u''} & \fuchsiaText{=} & \fuchsiaText{2}.
    \end{eqnarray*}
  }

  \uncover<3->{%
    Solving this differential equation we get
    \begin{eqnarray*}
      u & = & \fuchsiaText{t^2 + C_1 t + C_2}, \\
      y & = & \lp \fuchsiaText{t^2 + C_1 t + C_2} \rp e^{-t}
    \end{eqnarray*}
  }
  
\end{frame}


\begin{frame}
  \frametitle{Reduction of Order}

  Suppose that $y_1(t)$ is \textbf{a} solution to the homogeneous part
  of
  \begin{eqnarray*}
    a y'' + by' + cy & = & f(t).
  \end{eqnarray*}

  \uncover<2->
  {
    Let
    \begin{eqnarray*}
      y_p & = & u(t) y_1(t), \\
      \uncover<3->{y_p' & = & \redText{u'(t) y_1(t) + u(t) y_1'(t)}} \\
      \uncover<4->{y_p'' & = & \blueText{u''(t) y_1(t) + 2 u'(t) y_1'(t) + u(t) y_1''(t)}}
    \end{eqnarray*}
  }

\end{frame}

\begin{frame}
  \frametitle{Substitute this back into the original equation}

  \begin{eqnarray*}
    a \lp \redText{u''(t) y_1(t) + 2 u'(t) y_1'(t) + u(t) y_1''(t)} \rp & & \\
    + b \lp \blueText{u'(t) y_1(t) + u(t) y_1'(t)} \rp & & \\
    c u(t) y_1(t) & = & f(t)
  \end{eqnarray*}

  \begin{eqnarray*}
    \begin{array}{r|r|cl}
      \cline{2-2}
      \redText{ a u''(t) y_1(t) + 2 a u'(t) y_1'(t) +} & \redText{a u(t) y_1''(t)} & & \\
      \blueText{+ b u'(t) y_1(t) + } & \blueText{b u(t) y_1'(t)} & & \\
      & c u(t) y_1(t) & = & f(t) \\ \cline{2-2}
    \end{array}
  \end{eqnarray*}

\end{frame}


\begin{frame}
  \frametitle{Reduction of Order}

  The equation reduces to
  \begin{eqnarray*}
    a \redText{u''(t)} \blueText{y_1(t)} + 2 a \redText{u'(t)} \blueText{y_1'(t)} + b \redText{u'(t)} \blueText{y_1(t)} & = & f(t).
  \end{eqnarray*}

  Let $v=u'$ so the equation is now
  \begin{eqnarray*}
    a \redText{v'} y_1 + 2 a \redText{v} y_1' + b \redText{v} y_1 & = & f(t).
  \end{eqnarray*}
  This is a first order, linear equation! We can solve this!


\end{frame}

% LocalWords: Clarkson pausesection hideothersubsections
