\part{Solutions-to-Second-Order-DEs}
\lecture{Solutions to Second Order DEs}{Solutions-to-Second-Order-DEs}
\section{Solutions to Second Order DEs}

\title{Ordinary Differential Equations}
\subtitle{Math 232 - Solutions to Second Order, Linear, Constant Coef. Equations}
\date{25 September 2013}

\begin{frame}
  \titlepage
\end{frame}

\begin{frame}
  \frametitle{Outline}
  \tableofcontents[ currentsection ]
\end{frame}


\subsection{Solutions to Second Order DEs}


\begin{frame}
  \frametitle{What is a solution to a general, \redText{linear,
      constant coefficient, homogeneous, second order} DE?}

  Solutions to
  \begin{eqnarray*}
    a y'' + by' + cy & = & 0.
  \end{eqnarray*}

  \uncover<2->
  {
    Assume
    \begin{eqnarray*}
      \only<2->{ {\color{green} y} & {\color{green}=} & {\color{green}A e^{rt},} \\}
      \uncover<3->{%
        {\color{red}y'} & {\color{red}=} & {\color{red}r A e^{rt},} \\
        {\color{red}y''} & {\color{red}=} & {\color{red}r^2 A e^{rt}.}
      }
    \end{eqnarray*}
    (If it works I am done!)
  }

\end{frame}


\begin{frame}
  \frametitle{Substitute back into the equation}

  The original de:
  \begin{eqnarray*}
    a y'' + by' + cy & = & 0, \\
    a r^2 A e^{rt} + b r A e^{rt} + c A e^{rt} & = & 0 \\
    \uncover<2->
    {
      A e^{rt} \left[ {\color{red} a r^2 + br + c } \right] & = & 0.
    }
  \end{eqnarray*}

  \uncover<3->
  {
    Assume $A\neq 0$ and $e^{rt}$ cannot be zero so
    \begin{eqnarray*}
      {\color{red}a r^2 + br + c} & {\color{red}=} & {\color{red}0.}
    \end{eqnarray*}
  }

  \uncover<4->
  {
    This is true if
    \begin{eqnarray*}
      r & = & \frac{-b \pm \sqrt{b^2-4ac}}{2a}.
    \end{eqnarray*}
  }

  \uncover<5->
  {
    Three cases: distinct real, repeated roots, complex roots.
  }

\end{frame}

\subsection{Case 1: Real, Distinct Roots}

\begin{frame}
  \frametitle{Case 1: Distinct, Real Roots}

  \begin{eqnarray*}
    y & = & C_1 e^{r_1 t} + C_2 e^{r_2 t}.
  \end{eqnarray*}

\end{frame}


\begin{frame}
  \frametitle{Example}

  Find the general solution to
  \begin{eqnarray*}
    y'' - 5y' + 6y & = & 0.
  \end{eqnarray*}

  \only<2>
  {
    Assume that $y=Ae^{rt}$. Then we have
    \begin{eqnarray*}
      r^2Ae^{rt} - 5r Ae^{rt} + 6 Ae^{rt} & = & 0, \\
      Ae^{rt} \lp r^2 - 5r + 6 \rp & = & 0, \\
      r^2 - 5r + 6 & = & 0.
    \end{eqnarray*}
  }

  \only<3>
  {
    The characteristic equation is
    \begin{eqnarray*}
      r^2 - 5r + 6 & = & 0, \\
      (r-3)(r-2) & = & 0.
    \end{eqnarray*}

    The general solution is
    \begin{eqnarray*}
      {\color{red}y} & {\color{red}=} & {\color{red}C_1 e^{3t} + C_2 e^{2t}}.
    \end{eqnarray*}

  }


  \only<4>
  {
    The general solution is
    \begin{eqnarray*}
      y & = & {\color{red}C_1 e^{3t}} + C_2 e^{2t}.
    \end{eqnarray*}
    The dominant term is $e^{3t}$. Exponential growth.
  }

\end{frame}


\iftoggle{clicker}{%
\begin{frame}
  \frametitle{Clicker Quiz}

      \ifnum\value{clickerQuiz}=1{%

        \vfill

        Which equation below is the characteristic equation for the
        differential equation
        \begin{eqnarray*}
          y'' - 2y' - 8y & = & 0.
        \end{eqnarray*}

        \vfill

        \begin{tabular}{ll}
          A: & $8r^2-2r+1=0$ \\
          B: & $r^2-2r-8=0$ \\
          C: & $r^2-5r+6=0$
        \end{tabular}


        \vfill

      }\fi

      \ifnum\value{clickerQuiz}=2{%

        \vfill

        Which equation below is the characteristic equation for the
        differential equation
        \begin{eqnarray*}
          y'' - 2y' - 8y & = & 0.
        \end{eqnarray*}

        \vfill

        \begin{tabular}{ll}
          A: & $r^2-5r+6=0$ \\
          B: & $8r^2-2r+1=0$ \\
          C: & $r^2-2r-8=0$
        \end{tabular}

        \vfill

     }\fi

     \ifnum\value{clickerQuiz}=3{%
     Which equation below is the characteristic equation for the
        differential equation
        \begin{eqnarray*}
          y'' - 2y' - 8y & = & 0.
        \end{eqnarray*}

        \vfill

        \begin{tabular}{ll}
          A: & $r^2-5r+6=0$ \\
          B: & $8r^2-2r+1=0$ \\
          C: & $r^2-2r-8=0$
        \end{tabular}

   
     \vfill

    }\fi


\end{frame}
}



\begin{frame}
  \frametitle{Example}

  Find the general solution to
  \begin{eqnarray*}
    y'' - 2y' - 8y & = & 0.
  \end{eqnarray*}

  \uncover<2->
  {
    \begin{eqnarray*}
      r^2 - 2r - 8 & = & 0, \\
      (r-4)(r+2) & = & 0.
    \end{eqnarray*}
  }

  \only<2>
  {
    The general solution is
    \begin{eqnarray*}
      y & = & C_1 e^{4t} + C_2 e^{-2t}
    \end{eqnarray*}

  }

  \only<3->
  {
    The general solution is
    \begin{eqnarray*}
      y & = & C_1 {\color{red}e^{4t}} + C_2 e^{-2t}
    \end{eqnarray*}
    The $e^{4t}$ term is the dominant term.

  }


\end{frame}


\subsection{Case 2: Repeated Roots}

\begin{frame}
  \frametitle{Case 2: Repeated Roots}

  \begin{eqnarray*}
    \only<1>{
      y & = & C_1 e^{\blueText{r}t} + C_2 t e^{\blueText{r}t} \\
    }
    \only<2>{
      y & = & C_1 e^{\blueText{r}t} + C_2 \redText{t} e^{\blueText{r}t} \\
    }
    \only<3>{
      y & = & C_1 e^{\blueText{r}t} + \redText{C_2 t e^{\blueText{r}t}} \\
    }
  \end{eqnarray*}


\end{frame}


\begin{frame}
  \frametitle{Example}

  \begin{eqnarray*}
    y'' + 6y' + 9y & = & 0.
  \end{eqnarray*}

  \uncover<2->
  {
    \begin{eqnarray*}
      r^2 + 6r + 9 & = & 0, \\
      (r+3)^2 & = & 0.
    \end{eqnarray*}

    The general solution is
    \begin{eqnarray*}
      y & = & C_1 e^{-3t} + C_2 t e^{-3t}
    \end{eqnarray*}

  }

\end{frame}


\begin{frame}
  \frametitle{Nomenclature}

  The governing equation for spring/mass systems (or RCL circuits) is
  \begin{eqnarray*}
    m x'' + b x' + k x & = & 0.
  \end{eqnarray*}

  We characterize the system in terms of its behavior which is
  dictated by the roots of the characteristic equation:
  \begin{itemize}
  \item Distinct real, negative roots - \textit{\redText{Over damped}}
  \item One repeated root - \textit{\redText{Critically damped}}
  \item Complex roots - \textit{\redText{Under damped}}
  \end{itemize}

\end{frame}

\begin{frame}
  \frametitle{Example}

  \begin{eqnarray*}
    y'' + 4y' + 4y & = & 0, \\
    y(0) & = & 0, \\
    y'(0) & = & 3.
  \end{eqnarray*}

  \uncover<2->
  {
    \begin{eqnarray*}
      r^2 + 4r + 4 & = & 0, \\
      (r+2)^2 & = & 0.
    \end{eqnarray*}

    The general solution is
    \begin{eqnarray*}
      y & = & C_1 e^{-2t} + C_2 t e^{-2t}
    \end{eqnarray*}

  }

\end{frame}



\begin{frame}
  \frametitle{Solve for the constants}

  \begin{eqnarray*}
    y(0) & = & 0 \\
    & = & C_1 \\
    \Rightarrow C_1 & = & 0.
  \end{eqnarray*}

  \begin{eqnarray*}
    y(t) & = & C_2 t e^{-2t} \\
    y'(t) & = & C_2 e^{-2t} - 2 C_2 t e^{-2t}, \\
    y'(0) & = & C_2 \\
    \Rightarrow C_2 & = & 3.
  \end{eqnarray*}

  The solution is
  \begin{eqnarray*}
    y & = & 3 t e^{-2t}.
  \end{eqnarray*}

\end{frame}

\subsection{More Examples}


\iftoggle{clicker}{%
\begin{frame}
  \frametitle{Clicker Quiz}

      \ifnum\value{clickerQuiz}=1{%

        \vfill

        Which equation below is the characteristic equation for the
        differential equation
        \begin{eqnarray*}
          y'' + 10 y' + 24y & = & 0.
        \end{eqnarray*}

        \vfill

        \begin{tabular}{ll}
          A: & $24r^2+10r+1=0$ \\
          B: & $r^2+10r+24=0$ \\
          C: & $r^2-5r+6=0$
        \end{tabular}


        \vfill

      }\fi

      \ifnum\value{clickerQuiz}=2{%

        \vfill

        Which equation below is the characteristic equation for the
        differential equation
        \begin{eqnarray*}
          y'' + 10 y' + 24y & = & 0.
        \end{eqnarray*}

        \vfill

        \begin{tabular}{ll}
          A: & $r^2-5r+6=0$ \\
          B: & $24r^2+10r+1=0$ \\
          C: & $r^2+10r+24=0$
        \end{tabular}

        \vfill

     }\fi

     \ifnum\value{clickerQuiz}=3{%
                Which equation below is the characteristic equation for the
        differential equation
        \begin{eqnarray*}
          y'' + 10 y' + 24y & = & 0.
        \end{eqnarray*}

        \vfill

        \begin{tabular}{ll}
          A: & $r^2-5r+6=0$ \\
          B: & $24r^2+10r+1=0$ \\
          C: & $r^2+10r+24=0$
        \end{tabular}

        \vfill

        \vfill


    }\fi


\end{frame}
}


\begin{frame}
  \frametitle{Example}

  Find the general solution to
  \begin{eqnarray*}
    y'' + 10 y' + 24y & = & 0.
  \end{eqnarray*}

  \uncover<2->
  {
    \begin{eqnarray*}
      r^2 + 10r + 24 & = & 0, \\
      r & = & \frac{-10\pm\sqrt{100-4(24)}}{2} \\
      & = & -4,~ -6.
    \end{eqnarray*}
  }

  \only<2>
  {
    The general solution is
    \begin{eqnarray*}
      y & = & C_1 e^{-4t} + C_2 e^{-6t}
    \end{eqnarray*}
  }

  \only<3->
  {
    The general solution is
    \begin{eqnarray*}
      y & = & C_1 {\color{red}e^{-4t}} + C_2 e^{-6t}
    \end{eqnarray*}
    The $e^{-4t}$ term is the dominant term.
  }


\end{frame}

%\begin{frame}
%  \frametitle{Linear Independence}
%
%  Question: Are the functions linear independent?
%
%  \uncover<2->
%  {
%
%    If we have $n$-derivatives we need $n$ linearly independent
%    functions:
%    \begin{eqnarray*}
%      \mathrm{det}\arrayTwo{e^{-4t}}{e^{-6t}}{-4e^{-4t}}{-6e^{-6t}}
%      & = & -2 e^{-10t}.
%    \end{eqnarray*}
%
%  }
%
%\end{frame}


%\begin{frame}
%  \frametitle{Example}
%
%  Find the general solution to
%  \begin{eqnarray*}
%    y'' + y' - 2y & = & 0, \\
%    y(0) & = & 1, \\
%    y'(0) & = & 2.
%  \end{eqnarray*}
%
%  \uncover<2->
%  {
%    \begin{eqnarray*}
%      r^2 + r - 2 & = & 0, \\
%      (r-1)(r+2) & = & 0.
%    \end{eqnarray*}
%
%    The general solution is
%    \begin{eqnarray*}
%      y & = & C_1 e^{-2t} + C_2 e^{t}
%    \end{eqnarray*}
%
%  }
%\end{frame}

%\begin{frame}
%  \frametitle{Find The Specific Solution}
%
%  Apply the initial condition:
%  \begin{eqnarray*}
%    y(0) & = & C_1 + C_2 \\
%    y'(0) & = & -2C_1 + C_2
%  \end{eqnarray*}
%
%  \begin{eqnarray*}
%    \startRowOpsTwo
%    \oneRowOpsTwo{1}{1}{1}{~}
%    \oneRowOpsTwo{-2}{1}{2}{~}
%    \stopRowOps \\
%    \Rightarrow
%    \startRowOpsTwo
%    \oneRowOpsTwo{1}{0}{-\frac{1}{3}}{~}
%    \oneRowOpsTwo{0}{1}{\frac{4}{3}}{~}
%    \stopRowOps
%  \end{eqnarray*}
%
%  So the solution is
%  \begin{eqnarray*}
%    y & = & -\frac{1}{3} e^{-2t} + \frac{4}{3} e^{t}.
%  \end{eqnarray*}
%
%\end{frame}


\begin{frame}
  \frametitle{Higher Order Derivatives}

  No reason to stop at two derivatives! Find the general solution to
  \begin{eqnarray*}
    y''' + 2 y'' - y' - 2y & = & 0.
  \end{eqnarray*}
  Assume that $y=Ae^{rt}$:
  \begin{eqnarray*}
    r^3 + 2 r^2 - r + 2 & = & 0, \\
    \uncover<2->
    {
      (r+2)(r-1)(r+1) & = & 0.
    }
  \end{eqnarray*}

  \only<3>
  {
    \begin{eqnarray*}
      y & = & C_1 e^{-2t} + C_2 e^{t} + C_3 e^{-t}.
    \end{eqnarray*}
  }

  \only<4->
  {
    \begin{eqnarray*}
      y & = & C_1 e^{-2t} + C_2 \redText{e^{t}} + C_3 e^{-t}.
    \end{eqnarray*}
    The dominant term is \redText{$e^t$}.
  }

\end{frame}


\begin{frame}
  \frametitle{Higher Order Derivatives}

  Find the general solution to
  \begin{eqnarray*}
    y''' + 11 y'' + 38 y' + 40y & = & 0.
  \end{eqnarray*}
  Assume that $y=Ae^{rt}$:
  \begin{eqnarray*}
    r^3 + 11 r^2 + 38 r + 40 & = & 0, \\
    \uncover<2->
    {
      (r+2)(r+4)(r+5) & = & 0.
    }
  \end{eqnarray*}

  \only<3>
  {
    \begin{eqnarray*}
      y & = & C_1 e^{-2t} + C_2 e^{-4t} + C_3 e^{-5t}.
    \end{eqnarray*}
  }

  \only<4->
  {
    \begin{eqnarray*}
      y & = & C_1 \redText{e^{-2t}} + C_2 e^{-4t} + C_3 e^{-5t}.
    \end{eqnarray*}
    The dominant term is $\redText{e^{-2t}}$.
  }


\end{frame}


\begin{frame}
  \frametitle{Higher Order Derivatives}

  Find the general solution to
  \begin{eqnarray*}
    y''' + 3 y'' - 4y & = & 0.
  \end{eqnarray*}
  Assume that $y=Ae^{rt}$:
  \begin{eqnarray*}
    r^3 + 3 r^2 - 4 & = & 0, \\
    \uncover<2->
    {
      (r+2)^2(r-1) & = & 0.
    }
  \end{eqnarray*}

  \only<3>
  {
    \begin{eqnarray*}
      y & = & C_1 e^{-2t} + C_2 t e^{-2t} + C_3 e^{t}.
    \end{eqnarray*}
  }

  \only<4->
  {
    \begin{eqnarray*}
      y & = & C_1 e^{-2t} + C_2 t e^{-2t} + C_3 \redText{e^{t}}.
    \end{eqnarray*}
    The dominant terms is $\redText{e^{t}}$.
  }


\end{frame}


\begin{frame}
  \frametitle{Example}

  Find the general solution to
  \begin{eqnarray*}
    y'' + 7y' + 10y & = & 0, \\
    y(0) & = & 2, \\
    y'(0) & = & 3.
  \end{eqnarray*}

  \uncover<2->
  {
    \begin{eqnarray*}
      r^2 + 7r + 10 & = & 0, \\
      (r+2)(r+5) & = & 0.
    \end{eqnarray*}

    \only<1>{
      The general solution is
      \begin{eqnarray*}
        y & = & C_1 e^{-2t} + C_2 e^{-5t}
      \end{eqnarray*}
    }

    \only<2>{
      The general solution is
      \begin{eqnarray*}
        y & = & C_1 \redText{e^{-2t}} + C_2 e^{-5t}
      \end{eqnarray*}
      The dominant terms is $\redText{e^{-2t}}$.
    }

  }


\end{frame}


\begin{frame}
  \frametitle{Find The Specific Solution}

  Apply the initial condition:
  \begin{eqnarray*}
    y(0) & = & C_1 + C_2 \\
    y'(0) & = & -2C_1 - 5 C_2
  \end{eqnarray*}

%  \begin{eqnarray*}
%    \startRowOpsTwo
%    \oneRowOpsTwo{1}{1}{2}{~}
%    \oneRowOpsTwo{-2}{-5}{3}{~}
%    \stopRowOps \\
%    \Rightarrow
%    \startRowOpsTwo
%    \oneRowOpsTwo{1}{0}{\frac{13}{3}}{~}
%    \oneRowOpsTwo{0}{1}{-\frac{7}{3}}{~}
%    \stopRowOps
%  \end{eqnarray*}

  \only<1>{
    \begin{eqnarray*}
      C_1 + C_2 & = & 2, \\
      -2C_1 - 5C_2 & = & 3.
    \end{eqnarray*}
  }

  \only<2>{
    \begin{eqnarray*}
      2C_1 + 2C_2 & = & 4, \\
      -2C_1 - 5C_2 & = & 3.
    \end{eqnarray*}
  }

  \only<3>{
    Multiply the first equation by 2:
    \begin{eqnarray*}
      \redText{2C_1 + 2C_2} & = & \redText{4}, \\
      -2C_1 - 5C_2 & = & 3. \\ \hline
      -3C_2 & = & 7.
    \end{eqnarray*}
    So $C_2=\frac{-7}{3}$.
  }


  \only<4->{
    So the solution is
    \begin{eqnarray*}
      y & = & \frac{13}{3} e^{-2t} - \frac{7}{3} e^{-5t}.
    \end{eqnarray*}
  }

\end{frame}


\begin{frame}
  \frametitle{Example}

  Find the general solution to
  \begin{eqnarray*}
    2y'' + 7y' + 3y & = & 0, \\
    y(0) & = & 1, \\
    y'(0) & = & 4.
  \end{eqnarray*}

  \uncover<2->
  {
    \begin{eqnarray*}
      2r^2 + 7r + 3 & = & 0, \\
      r & = & \frac{-7\pm\sqrt{49-24}}{4}, \\
      r & = & -\half,~-3.
    \end{eqnarray*}

    The general solution is
    \only<2>{
      \begin{eqnarray*}
        y & = & C_1 e^{-t/2} + C_2 e^{-3t}
      \end{eqnarray*}
    }
    \only<3>{
      \begin{eqnarray*}
        y & = & C_1 \redText{e^{-t/2}} + C_2 e^{-3t}
      \end{eqnarray*}
      The dominant term is $\redText{e^{-t/2}}$.
    }

  }


\end{frame}


\begin{frame}
  \frametitle{Find The Specific Solution}

  Apply the initial condition:
  \begin{eqnarray*}
    y(0) & = & C_1 + C_2 \\
    y'(0) & = & -\half C_1 - 3 C_2
  \end{eqnarray*}

%  \begin{eqnarray*}
%    \startRowOpsTwo
%    \oneRowOpsTwo{1}{1}{1}{~}
%    \oneRowOpsTwo{-\half}{-3}{4}{~}
%    \stopRowOps \\
%    \Rightarrow
%    \startRowOpsTwo
%    \oneRowOpsTwo{1}{0}{\frac{14}{5}}{~}
%    \oneRowOpsTwo{0}{1}{-\frac{9}{5}}{~}
%    \stopRowOps
%  \end{eqnarray*}


  \only<1>{
    Apply the initial condition:
    \begin{eqnarray*}
      C_1 + C_2 & = & 1 \\
      -\half C_1 - 3 C_2 & = & 4.
    \end{eqnarray*}
  }

  \only<2>{
    multiply the first equation by $\half$:
    \begin{eqnarray*}
      \half C_1 + \half C_2 & = & \half \\
      -\half C_1 - 3 C_2 & = & 4.
    \end{eqnarray*}
  }

  \only<3>{
    Add the two equations:
    \begin{eqnarray*}
      \half C_1 + \half C_2 & = & \half \\
      -\half C_1 - 3 C_2 & = & 4, \\ \hline 
      \frac{-5}{2} C_2 & = & \frac{9}{2}, \\
      C_2 & = & \frac{-9}{5}.
    \end{eqnarray*}
  }


  \only<4>{
    So the solution is
    \begin{eqnarray*}
      y & = & \frac{14}{5} e^{-t/2} - \frac{9}{5} e^{-3t}.
    \end{eqnarray*}
  }


\end{frame}



% LocalWords: Clarkson pausesection hideothersubsections DEs Coef RCL
