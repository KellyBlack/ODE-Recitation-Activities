\part{First-Order-Linear-Equations}
\lecture{First Order Linear Equations}{First-Order-Linear-Equations}
\section{First Order Linear Equations}

\title{Ordinary Differential Equations}
\subtitle{Math 232 - Week 2, Day 2}
\date{5 Sep 2012}

\begin{frame}
  \titlepage
\end{frame}

\begin{frame}
  \frametitle{Outline}
  \tableofcontents[pausesection,hideothersubsections]
\end{frame}


\subsection{First Order Linear Equations}

\iftoggle{clicker}{%

  \begin{frame}
    \frametitle{Clicker Quiz}

      \ifnum\value{clickerQuiz}=1{%

        Which differential equation is a linear equation?

       \vfill

       \begin{tabular}{l@{\hspace{3em}}l}
         A: & $y'=y^2$. \\
         B: & $y'' + 3y'-2y=0$. \\
         C: & $y'+\sqrt{y}=\sin(t)$. \\
         D: & $y' + \cos(t)y  =  \ln(t)$. \\
         E: & A and B. \\
         F: & B and D.
       \end{tabular}

     }\fi

     \ifnum\value{clickerQuiz}=2{%

        Which differential equation is a linear equation?

       \vfill

       \begin{tabular}{l@{\hspace{3em}}l}
         A: & $y'=3y$. \\
         B: & $y'' + 3\sin(t)y'-2y=0$. \\
         C: & $y''+ y^2 =\sin(t)$. \\
         D: & $y' + y y'  =  \ln(t)$. \\
         E: & A and B. \\
         F: & A and D.
       \end{tabular}


     }\fi
    

      \ifnum\value{clickerQuiz}=3{%
       Which differential equation is a linear equation?
      \vfill

       \begin{tabular}{l@{\hspace{3em}}l}
         A: & $y'-y^2 = 0$. \\
\vspace{0.3cm}
         B: & $y'' + 3y'-2y=t^2$. \\
\vspace{0.3cm}
         C: & $y'+\cos{y}=\sqrt(t)$. \\
\vspace{0.3cm}
         D: & $y' + \ln(t)y  =  \cos(t)$. \\
\vspace{0.3cm}
         E: & A and B. \\
\vspace{0.3cm}
         F: & B and D.
\vspace{0.3cm}
       \end{tabular}

      }\fi

    \vfill
    \vfill
    \vfill

  \end{frame}

}


\begin{frame}
  \frametitle{Another ``Special Form''}

  \begin{eqnarray*}
    y' + p(t) y & = & f(t)
  \end{eqnarray*}

  read pages 63-66 

  I will only cover integrating factors in class.

\end{frame}

\subsection{Product Rule}

\begin{frame}
  \frametitle{The Product Rule}

  \begin{eqnarray*}
    \frac{d}{dt} \lp u v \rp & = & u' v + u v'
  \end{eqnarray*}

\end{frame}


\begin{frame}
  \frametitle{Example}

  \begin{eqnarray*}
    y ' & = & - y + t \\
    \uncover<2->{ y' + y & = & t} \\
    \uncover<3->{y' e^t + y e^ t & = & t e^t } \\
    \uncover<4->{\deriv{~}{t} \lp y e^t \rp & = & t e^t } \\
    \uncover<5->{y e^t  & = & \int t e^t ~ dt } \\
    \uncover<6->{y e^t  & = & t e^t - e^t + C } \\
    \uncover<7->{ y & = & t -1 + Ce^{-t}}
  \end{eqnarray*}

  \uncover<8->{How did I know how to do this?}


\end{frame}


\subsection{Integrating Factors}

\begin{frame}
  \frametitle{Let's Backup}

  \begin{eqnarray*}
    y' + p(t) y & = & f(t) \\
    e^{G(t)} y' + p(t) e^{G(t)} y & = & e^{G(t)} f(t)
  \end{eqnarray*}


\end{frame}



\begin{frame}
  \frametitle{Product Rule Redux}

  \begin{eqnarray*}
    \frac{d}{dt} \lp e^{G(t)} \rp & = & G'(t) e^{G(t)} 
  \end{eqnarray*}

  Let
  \begin{eqnarray*}
    G'(t) & = & p(t) \\
    \Rightarrow G(t) & = & \int p(t) ~ dt.
  \end{eqnarray*}


\end{frame}


\begin{frame}
  \frametitle{Product Rule Redux}

  \begin{eqnarray*}
    \frac{d}{dt} \lp e^{G(t)} y \rp & = & e^{G(t)} y' + G'(t) e^{G(t)} y \\
    & = & e^{G(t)} y' + p(t) e^{G(t)} y \\
    & = & e^{G(t)} \lp y' + p(t) y \rp \\
    & = & f(t) e^{G(t)}
  \end{eqnarray*}

  or
  \begin{eqnarray*}
    e^{G(t)} y & = & \int f(t) e^{G(t)} ~ dt
  \end{eqnarray*}
  \textit{Hope and pray!}

\end{frame}


\subsection{Examples}

\begin{frame}
  \frametitle{Example}

  \vspace*{-3em}
  \begin{eqnarray*}
    y' + 2y & = & t \\
    p(t) & = & 2, \\
    \int 2 ~ dt & = & 2t + K, \\
    e^{2t} & & 
  \end{eqnarray*}

  \begin{eqnarray*}
    e^{2t} y' + 2 e^{2t} y & = & t e^{2t}, \\
    \frac{d}{dt} \lp e^{2t} y \rp & = & t e^{2t} \\
    e^{2t} y & = & \int t e^{2t} ~ dt \\
    e^{2t} y & = & \half t e^{2t} - \frac{1}{4} e^{2t} + C \\
    y & = & \half t - \frac{1}{4} + C e^{-2t}
  \end{eqnarray*}

\end{frame}


\iftoggle{clicker}{%
\begin{frame}
  \frametitle{Clicker Quiz}
    
      \ifnum\value{clickerQuiz}=1{%

        What is the integrating factor for the equation
        \begin{eqnarray*}
          y' & = & ty + 1 ?
        \end{eqnarray*}

       \vfill

       \begin{tabular}{l@{\hspace{3em}}l}
         A: & $e^{t}$. \\
         B: & $e^{-t}$. \\
         C: & $e^{t^2/2}$. \\
         D: & $e^{-t^2/2}$. \\
       \end{tabular}

     }\fi

     \ifnum\value{clickerQuiz}=2{%

        What is the integrating factor for the equation
        \begin{eqnarray*}
          y' & = & ty + t^4 ?
        \end{eqnarray*}

       \vfill

       \begin{tabular}{l@{\hspace{3em}}l}
         A: & $e^{t}$. \\
         B: & $e^{-t}$. \\
         C: & $e^{t^2/2}$. \\
         D: & $e^{-t^2/2}$. \\
       \end{tabular}


     }\fi

      \ifnum\value{clickerQuiz}=3{%
            What is the integrating factor for the equation
        \begin{eqnarray*}
          3 y' + y & = & 3t ?
        \end{eqnarray*}

       \vfill

       \begin{tabular}{l@{\hspace{3em}}l}
         A: & $e^{t}$. \\
         B: & $e^{-t}$. \\
         C: & $e^{t/2}$. \\
         D: & $e^{t/3}$. \\
       \end{tabular}


     }\fi

    \vfill
    \vfill
    \vfill

\end{frame}

}


\begin{frame}
  \frametitle{Example}

  \vspace*{-3em}
  \begin{eqnarray*}
    y' & = & ty + t^3 \\
    y' - ty & = & t^3 \\
    p(t) & = & -t, \\
    \int -t ~ dt & = & -\half t^2 + K \\
    e^{-\half t^2} & & 
  \end{eqnarray*}

  \begin{eqnarray*}
    y' e^{-\half t^2} - t e^{-\half t^2} y & = & t^3 e^{-\half t^2} \\
    \frac{d}{dt} \lp y e^{-\half t^2} \rp & = & \int t^3 e^{-\half t^2} ~ dt \\
    y e^{-\half t^2} & = & -t^2 e^{-\half t^2} - 2 e^{-\half t^2} + C \\
    y & = & -t^2 - 2 + C e^{\half t^2}
  \end{eqnarray*}

\end{frame}


\iftoggle{clicker}{%
\begin{frame}
  \frametitle{Clicker Quiz}
    
      \ifnum\value{clickerQuiz}=1{%

        What is the integrating factor for the equation
        \begin{eqnarray*}
          y' & = & -\frac{4}{t} y + 4 ?
        \end{eqnarray*}

       \vfill

       \begin{tabular}{l@{\hspace{3em}}l}
         A: & $4t$. \\
         B: & $4\ln(t)$. \\
         C: & $t^4$. \\
         D: & $t^{-4}$. \\
       \end{tabular}

     }\fi

     \ifnum\value{clickerQuiz}=2{%

        What is the integrating factor for the equation
        \begin{eqnarray*}
          y' & = & -\frac{4}{t} y + t^3?
        \end{eqnarray*}

       \vfill

       \begin{tabular}{l@{\hspace{3em}}l}
         A: & $t^{-4}$. \\
         B: & $t^4$. \\
         C: & $4\ln(t)$. \\
         D: & $4t$. \\
       \end{tabular}


     }\fi

      \ifnum\value{clickerQuiz}=3{%
       What is the integrating factor for the equation
        \begin{eqnarray*}
          y' & = & \frac{3}{t} y + 7?
        \end{eqnarray*}

       \vfill

       \begin{tabular}{l@{\hspace{3em}}l}
         A: & $t^{-3}$. \\
         B: & $t^3$. \\
         C: & $3\ln(t)$. \\
         D: & $3t$. \\
       \end{tabular}

     }\fi

    \vfill
    \vfill
    \vfill

\end{frame}

}


\begin{frame}
  \frametitle{Example}

  \begin{eqnarray*}
    y' & = & -\frac{4}{t} y + 4 t \\
    y' + \frac{4}{t} y & = & 4 t \\
    p(t) & = & \frac{4}{t} \\
    \int \frac{4}{t} ~ dt & = & 4 \ln(t) + K \\
    e^{4 \ln(t) } & = & t^4
  \end{eqnarray*}

\end{frame}

\begin{frame}
  \frametitle{Example - continues}

  \begin{eqnarray*}
    t^4 y' + 4t^3& = & 4 t^5 \\
    \frac{d}{dt} \lp t^4 y \rp & = & 4 t^5 \\
    t^4 y & = & \int 4 t^5 ~ dt \\
    t^4 y & = & \frac{4}{6} t^6 + C \\
    y & = & \frac{2}{3} t^2 + \frac{C}{t^4}
  \end{eqnarray*}

\end{frame}



\begin{frame}
  \frametitle{Example}

  \vspace*{-3em}
  \begin{eqnarray*}
    y' & = & -\frac{1}{t} y + \frac{1}{t^2 -1} \\
    y' + \frac{1}{t} y & = & \frac{1}{t^2 -1} \\
    p(t) & = & \frac{1}{t}, ~~  \int \frac{1}{t} ~ dt  =  \ln(t) + k \\
    \mu ( t ) = e^{\ln(t)} & = & t
  \end{eqnarray*}

  \begin{eqnarray*}
    t y' + y & = & \frac{t}{t^2-1} \\
    \frac{d}{dt} \lp t y \rp & = & \frac{t}{t^2-1} \\
    t y & = & \int \frac{t}{t^2-1} ~ dt
%  \end{eqnarray*}
%\end{frame}
%\begin{frame}
%  \frametitle{Partial Fractions}
%  \frametitle{Example}
%
%  \vspace{-3em}
%  \begin{eqnarray*}
%    \frac{t}{(t-1)(t+1)} & = & \frac{A}{t-1} + \frac{B}{t+1} \\
%    t & = & A(t+1) + B(t-1)
%  \end{eqnarray*}
%
%  Let $t=-1$
%  \begin{eqnarray*}
%    -1 & = & -2B, \\
%    B & = & \half
%  \end{eqnarray*}
%
%  Let $t=1$
%  \begin{eqnarray*}
%    1 & = & 2A \\
%    A & = & \half
%  \end{eqnarray*}
%  \begin{eqnarray*}
%    t y & = & \int \frac{\half}{t-1} + \frac{\half}{t+1} ~ dt \\
%    t y & = & \half \ln(t-1) + \half \ln(t+1) \\
%    t y & 
%=  \half \ln(|t^2-1|) + C \\
%    y &  = &  (\half \ln(|t^2-1|) + C )t^{-1}
 \end{eqnarray*}

\end{frame}


\begin{frame}
  \frametitle{Solving the Integral}

  Let $u=t^2-1$, and $du=2t$:
  \begin{eqnarray*}
    ty & = & \int \half \frac{1}{u} ~ du, \\
    t y & = & \half \ln(u) + C, \\
    t y & = & \half \ln \lp t^2 - 1 \rp + C, \\
    y & = & \frac{\ln\lp t^2-1 \rp}{t} + \frac{C}{t}.
  \end{eqnarray*}

\end{frame}


\begin{frame}
  \frametitle{Partial Fractions}

  \vspace{-3em}
  \begin{eqnarray*}
    \frac{t}{(t-1)(t+1)} & = & \frac{A}{t-1} + \frac{B}{t+1} \\
    t & = & A(t+1) + B(t-1)
  \end{eqnarray*}

  Let $t=-1$
  \begin{eqnarray*}
    -1 & = & -2B, \\
    B & = & \half
  \end{eqnarray*}

  Let $t=1$
  \begin{eqnarray*}
    1 & = & 2A \\
    A & = & \half
  \end{eqnarray*}

  \begin{eqnarray*}
    t y & = & \int \frac{\half}{t-1} + \frac{\half}{t+1} ~ dt \\
    t y & = & \half \ln(t-1) + \half \ln(t+1) \\
    y &  = & \frac{\half \ln(t-1) + \half \ln(t+1)}{t}
  \end{eqnarray*}
>>>>>>> 278de6b032600a7f55e724dc12cedc9e5989eded


\end{frame}


\begin{frame}
  \frametitle{Example}

  \begin{eqnarray*}
    y' & = & \cos(t) y - \cos(t) \\
    \uncover<2->{ y' & = & \cos(t) \lp y-1 \rp,} \\
    \uncover<3->{\frac{y'}{y-1} & = & \cos(t) \\
      \ln(y-1) & = & \sin(t) + C \\
      y - 1 & = & k e^{\sin(t)} \\
      y & = & 1 + ke^{\sin(t)} }
  \end{eqnarray*}

\end{frame}


\begin{frame}
  \frametitle{Example}

  \begin{eqnarray*}
    y' & = & y^2 + 1, \\
    \uncover<2->{ \frac{y'}{y^2+1} & = & 1,} \\
    \uncover<3->{\int \frac{y'}{y^2+1} ~ dt  & = & \int 1 ~ dt, \\
      \arctan(y) & = & t + C \\
      y  & = & \tan(t+C) }
  \end{eqnarray*}

\end{frame}



% LocalWords:  Clarkson pausesection hideothersubsections
