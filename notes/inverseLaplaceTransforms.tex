\part{Inverse-Laplace-Transforms}
\lecture{Inverse Laplace Transforms}{Inverse-Laplace-Transforms}
\section{Inverse Laplace Transforms}


\title{Ordinary Differential Equations}
\subtitle{Inverse Laplace Transforms}
\date{19 November 2012}

\begin{frame}
  \titlepage
\end{frame}

\begin{frame}
  \frametitle{Outline}
  \tableofcontents[pausesection,hideothersubsections]
\end{frame}


\subsection{Laplace Transform Table}

\begin{frame}
  \frametitle{Backwards Going Laplace Transforms }

  Before we begin a quick recap:
  \begin{enumerate}
  \item $\laplace{c_1 f(t) + c_2 g(t)}=c_1 \laplace{f(t)} + c_2 \laplace{g(t)}$
  \item $\laplace{f}=F(s)$
  \end{enumerate}

  Also:
  \begin{columns}
    \column{.5\textwidth}
    \begin{eqnarray*}
      \laplace{t^n}    & = & \frac{n!}{s^{n+1}} \\ 
      \laplace{e^{at}} & = & \frac{1}{s-a} \\ 
      \laplace{\sin(\omega t)} & = & \frac{\omega}{s^2+\omega^2} \\
      \laplace{\cos(\omega t)} & = & \frac{s}{s^2+\omega^2} \\ 
    \end{eqnarray*}

    \column{.5\textwidth}
    \begin{eqnarray*}
      \laplace{f(t)}  & = & \int^\infty_0 f(t) e^{-st} ~ dt \\ 
      \laplace{t^nf(t)} & = & (-1)^n\frac{d^n}{ds^n}F(s) \\
      \laplace{e^{at}f(t)} & = & F(s-a) 
    \end{eqnarray*}

  \end{columns}


\end{frame}

\iftoggle{clicker}{%
\begin{frame}
  \frametitle{Clicker Quiz}

   \ifnum\value{clickerQuiz}=1{%
     What is the Laplace transform of $e^{-3t}$?

     \vspace{2em}
     \begin{tabular}{ll}
       A: &  $\frac{1}{s+3}$ \\ [12pt]
       B: &  $\frac{1}{s-3}$ 
     \end{tabular}

     \vfill
   }\fi

   \ifnum\value{clickerQuiz}=2{%

     What is the Laplace transform of $e^{-3t}$?

     \vspace{2em}
     \begin{tabular}{ll}
       A: &  $\frac{1}{s+3}$ \\ [12pt]
       B: &  $\frac{1}{s-3}$ 
     \end{tabular}

   \vfill
   }\fi

  \ifnum\value{clickerQuiz}=3{%


  \vfill
 }\fi
\end{frame}
}



\subsection{The Inverse Laplace Transform}


\begin{frame}
  \frametitle{The Inverse Laplace Transform}

  \begin{block}{Question:}
    If $\laplace{f} = \frac{1}{s+3}$ what is f(t)?
  \end{block}

  \uncover<2->
  {
    Well, $a=-3$ so
    \begin{eqnarray*}
      \laplace{e^{at}} & = & \frac{1}{s-a}, \\
      f(t) & = & e^{-3t}.
    \end{eqnarray*}
  }

  \uncover<3->
  {
    Notation:
    \begin{eqnarray*}
      \invlaplace{\frac{1}{s+3}} & = & e^{-3t}.
    \end{eqnarray*}
  }

\end{frame}


\begin{frame}
  \frametitle{Example}

  \begin{block}{Question:}
    If $\laplace{f} = \frac{4}{s+3}-\frac{2}{s-5}$ what is f(t)?
  \end{block}

  \uncover<2->
  {
    \begin{eqnarray*}
      f & = & 4 e^{-3t} - 2 e^{5t}.
    \end{eqnarray*}
  }

\end{frame}


\begin{frame}
  \frametitle{Same Example, Only Different}

  \begin{block}{Question:}
    If $\laplace{f} = \frac{2s-26}{s^2-2s-15}$ what is f(t)?
  \end{block}

  \uncover<2->
  {
    \begin{eqnarray*}
      \frac{2s-26}{s^2-2s-15} & = & \frac{2s-26}{(s+3)(s-5)}, \\
      & = & \frac{a}{s+3} + \frac{b}{s-5}, \\
      & = & \frac{4}{s+3} - \frac{2}{s-5}.
    \end{eqnarray*}
  }

  \uncover<3->
  {
    \begin{eqnarray*}
      f & = & 4 e^{-3t} - 2 e^{5t}.
    \end{eqnarray*}
  }


\end{frame}


\begin{frame}
  \frametitle{Example}

  \begin{block}{Question:}
    If $\laplace{f} = \frac{-2}{(s-2)^4}$ what is f(t)?
  \end{block}

  \uncover<2->
  {
    \begin{eqnarray*}
      \laplace{e^{at}g(t)} & = & F(s-a), \\
      \Rightarrow f(t) & = & e^{2t} g(t), \\
      \laplace{g} & = & \frac{-2}{s^4}
    \end{eqnarray*}
  }



\end{frame}


\begin{frame}

  
    \begin{eqnarray*}
      \laplace{t^n} & = & \frac{n!}{s^{n+1}}, \\ [10pt]
      \Rightarrow n & = & 3, \\
      \laplace{g} & = & \frac{-2}{s^{4}}, \\
      & = & \frac{-2}{3!}\cdot\frac{3!}{s^4}, \\ [10pt]
      g(t) & = & \frac{-2}{3!} t^3, \\ [10pt]
      f(t) & = & -\frac{1}{3} e^{2t} t^3.
    \end{eqnarray*}


\end{frame}


\iftoggle{clicker}{%
\begin{frame}
  \frametitle{Clicker Quiz}

   \ifnum\value{clickerQuiz}=1{%

     Complete the square:
     \begin{eqnarray*}
       x^2+6x+12 & = & ?
     \end{eqnarray*}

     \vspace{2em}
     \begin{tabular}{ll}
       A: &  $(x+3)^2+3$ \\ [12pt]
       B: &  $(x+3)^2-3$ \\ [12pt]
       C: &  $(x+3)^2+12$ \\ [12pt]
       D: &  $(x+3)^2$ \\ [12pt]
     \end{tabular}

     \vfill
   }\fi

   \ifnum\value{clickerQuiz}=2{%

     Complete the square:
     \begin{eqnarray*}
       x^2+6x+12 & = & ?
     \end{eqnarray*}

     \vspace{2em}
     \begin{tabular}{ll}
       A: &  $(x+3)^2+3$ \\ [12pt]
       B: &  $(x+3)^2-3$ \\ [12pt]
       C: &  $(x+3)^2+12$ \\ [12pt]
       D: &  $(x+3)^2$ \\ [12pt]
     \end{tabular}

   \vfill
   }\fi

  \ifnum\value{clickerQuiz}=3{%


  \vfill
 }\fi
\end{frame}
}


\begin{frame}
  \frametitle{Example}

  \begin{block}{Question:}
    If $\laplace{f} = \frac{s+1}{(s+1)^2+9}$ what is f(t)?
  \end{block}

  \uncover<2->
  {
    \begin{eqnarray*}
      \laplace{e^{-t}g(t)} & = & \frac{s+1}{(s+1)^2+9}, \\
      \Rightarrow f(t) & = & e^{-t} g(t), \\
      \laplace{g} & = & \frac{s}{s^2+9}
    \end{eqnarray*}
  }



\end{frame}


\begin{frame}

    \begin{eqnarray*}
      \laplace{\cos(at)} & = & \frac{s}{s^2+a^2}, \\ [10pt]
      \Rightarrow a & = & 3, \\
      \laplace{g} & = & \frac{s}{s^2+9}, \\
      g(t) & = & \cos(3t), \\ [10pt]
      f(t) & = & e^{-t}\cos(3t).
    \end{eqnarray*}

\end{frame}


\begin{frame}
  \frametitle{Example}

  \begin{block}{Question}
    if $\laplace{g}=\frac{s}{s^2+2s+17}$ what is $g$?
  \end{block}

    \uncover<2->
    {
      \begin{eqnarray*}
        \laplace{g} & = & \frac{s}{s^2+2s+17}, \\
        & = & \frac{s}{(s+1)^2+16}, \\
        & = & \frac{s+1}{(s+1)^2+16} - \frac{1}{4}\cdot\frac{4}{(s+1)^2+16}.
      \end{eqnarray*}
    }

    \uncover<3->
    {
      \begin{eqnarray*}
        g(t) & = & e^{-t}\lp \cos(4t) - \frac{1}{4}\sin(4t)\rp.
      \end{eqnarray*}
    }
  

\end{frame}


\begin{frame}
  \frametitle{Example}

  \begin{block}{Question}
    if $\laplace{g}=\frac{s^2-14s-58}{s^3+2s^2+21s-58}$ what is $g$?
  \end{block}

    \uncover<2->
    {
      \begin{eqnarray*}
        \laplace{g} & = & \frac{s^2-14s-58}{s^3+2s^2+21s-58}, \\
        & = & \frac{s^2-14s-58}{(s-2)(s^2+4s+29)}, \\
        & = & \frac{as+b}{s^2+4s+29}+\frac{c}{s-2}, \\
        & = & 3\frac{s+2}{(s+2)^2+25} - 3\frac{2}{(s+2)^2+25}-\frac{2}{s-2}.
      \end{eqnarray*}
    }

    \uncover<3->
    {
      \begin{eqnarray*}
        g(t) & = & 3e^{-2t}\cos(5t) - \frac{6}{5}e^{-2t}\sin(5t)-2e^{2t}.
      \end{eqnarray*}
    }
  

\end{frame}


\begin{frame}
  \frametitle{Example}

  \begin{block}{Question}
    if $\laplace{g}=\frac{2s+1}{s^2(s^2+9)}$ what is $g$?
  \end{block}

    \uncover<2->
    {
      \begin{eqnarray*}
        \laplace{g} & = & \frac{2s+1}{s^2(s^2+9)}, \\
        & = & \frac{a}{s}+\frac{b}{s^2} + \frac{cs+d}{s^2+9}, \\
        & = & \frac{2}{9}\cdot\frac{1}{s} + \frac{1}{9}\cdot\frac{1}{s^2}-\frac{2}{9}\cdot\frac{s}{s^2+9}-\frac{1}{27}\cdot\frac{3}{s^2+9}.
      \end{eqnarray*}
    }

    \uncover<3->
    {
      \begin{eqnarray*}
        g(t) & = & \frac{2}{9} + \frac{1}{9}t - \frac{2}{9}\cos(3t) - \frac{1}{27} \sin(3t).
      \end{eqnarray*}
    }
  

\end{frame}


\begin{frame}
  \frametitle{Example}

  \begin{block}{Question}
    if $\laplace{g}=\frac{8s}{(s^2+16)^2}$ what is $g$?
  \end{block}

    \uncover<2->
    {
      \begin{eqnarray*}
        \laplace{g} & = & \frac{8s}{(s^2+16)^2}, \\
        & = & -4 \frac{-2s}{(s^2+16)^2}, \\
        & = & - \frac{d}{ds} \frac{4}{s^2+16}.
      \end{eqnarray*}
    }

    \uncover<3->
    {
      \begin{eqnarray*}
        g(t) & = & t \sin(4t).
      \end{eqnarray*}
    }
  

\end{frame}

\begin{frame}
  \frametitle{Example}

  \begin{block}{Question}
    if $\laplace{g}=\frac{8s^3-5s^2-12}{s^4-16}$ what is $g$?
  \end{block}

    \uncover<2->
    {
      \begin{eqnarray*}
        \laplace{g} & = & \frac{8s^3-5s^2-12}{s^4-16}, \\
        & = & \frac{82^3-5s^2-12}{(s-2)(s+2)(s^2+4)}, \\
        & = & \frac{a}{s-2} + \frac{b}{s+2} + \frac{cs+d}{s^2+4}, \\
        & = & \frac{1}{s-2} + \frac{3}{s+2} + 4 \frac{s}{s^2+4} - \frac{1}{2}\cdot\frac{2}{s^2+4}.
      \end{eqnarray*}
    }

    \uncover<3->
    {
      \begin{eqnarray*}
        g(t) & = & e^{2t} + 3e^{-2t} + 4\cos(2t) - \half\sin(2t).
      \end{eqnarray*}
    }
  

\end{frame}




% LocalWords:  Clarkson pausesection hideothersubsections
