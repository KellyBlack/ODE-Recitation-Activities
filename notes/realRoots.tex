\part{Real-Valued-Roots}
\lecture{Real Valued Roots}{Real-Valued-Roots}
\section{Ordinary Differential Equations}

\title{Ordinary Differential Equations}
\subtitle{Math 232 - Week 7, Day 3}
\date{12 Oct 2012}

\begin{frame}
  \titlepage
\end{frame}

\begin{frame}
  \frametitle{Outline}
  \tableofcontents[pausesection,hideothersubsections]
\end{frame}


\subsection{Second Order, Homogeneous Differential Equations}


\begin{frame}
  \frametitle{Second Order, Homogeneous Differential Equations}

  \begin{eqnarray*}
    a y'' + by' + cy & = & 0, a, b, c \text{are constants}, 
  \end{eqnarray*}
  Assume that $y =  A e^{rt}$, 
  which implies that
  \begin{eqnarray*}
    a r^2 + b r + c & = & 0, \\
    r & = & \frac{-b\pm\sqrt{b^2-4ac}}{2a}.
  \end{eqnarray*}

  Three cases: two distinct real roots, one repeated real root, and
  two complex roots.

\end{frame}

\begin{frame}
  \frametitle{Second Order, Homogeneous Differential Equations}
Since 
  \begin{eqnarray*}
  e^{d+ih} & = & e^d\left(\cos(h)+i\sin(h)\right) \\ 
  e^{d-ih} & = & e^d\left(\cos(-h)+i\sin(-h)\right)  \\
           & = & e^d\left(\cos(h)-i\sin(h)\right).  
  \end{eqnarray*}
Then  
  \begin{eqnarray*}
   &   &  C_1e^{d+ih} + C_2e^{d-ih}\\
   & = & C_1e^d\left(\cos(h)+i\sin(h)\right) 
       + C_2 e^d\left(\cos(h)-i\sin(h)\right)\\
   & = & e^d\left((C_1+C_2)\cos(h)+i(C_1-C_2)\sin(h)\right)\\
   & = & e^d\left(A_1\cos(h)+A_2\sin(h)\right)  
  \end{eqnarray*}

\end{frame}

\subsection{Complex Roots}

\begin{frame}
  \frametitle{Complex Roots}

  \begin{eqnarray*}
    y'' + y' + y & = & 0, \\
    \uncover<2->
    {
      r^2 + r + 1 & = & 0 \\
      r & = & \frac{-1\pm\sqrt{1-4}}{2} \\
      & = & -\half \pm \frac{\sqrt{-3}}{2} \\
      & = & -\half \pm i \frac{\sqrt{3}}{2} 
    }
  \end{eqnarray*}

\end{frame}

\begin{frame}
  \begin{eqnarray*}
    y & = & C_1 e^{\lp-\half + i \frac{\sqrt{3}}{2}\rp t} + C_2 e^{\lp -\half - i \frac{\sqrt{3}}{2}\rp t} \\
    & = & C_1 e^{-\half t}e^{i \frac{\sqrt{3}}{2}t} + C_2 e^{-\half t}e^{ - i \frac{\sqrt{3}}{2}t} \\
    & = & e^{-\half t} \left[ C_1 e^{i \frac{\sqrt{3}}{2}t} + C_2 e^{ - i \frac{\sqrt{3}}{2}t} \right] \\
    & = & e^{-\half t} \left[ \right. \\
    &   & C_1 \lp \cos\lp\frac{\sqrt{3}}{2}t\rp + i \sin\lp \frac{\sqrt{3}}{2} t \rp \rp \\ 
    &   & \left. + C_2 \lp \cos\lp\frac{\sqrt{3}}{2}t\rp - i \sin\lp\frac{\sqrt{3}}{2}t\rp\rp  \right] \\
    & = & e^{-\half t} \left[ 
          (C_1+C_2) \cos\lp\frac{\sqrt{3}}{2}t\rp + i(C_1-C_2) \sin\lp \frac{\sqrt{3}}{2}t\rp \right] \\ 
    & = & e^{-\half t} \left[ 
          A_1 \cos\lp\frac{\sqrt{3}}{2} t \rp + A_2 \sin\lp \frac{\sqrt{3}}{2} t \rp \right] \\ 
  \end{eqnarray*}
\end{frame}

\begin{frame}
  \frametitle{In General}

  \begin{eqnarray*}
    a y'' + by' + cy & = & 0,
  \end{eqnarray*}
  where $a$, $b$, and $c$ are constants, then assume
  $ y  =  A e^{rt}$,which implies that
  \begin{eqnarray*}
    a r^2 + b r + c & = & 0, \\
    r & = & \frac{-b\pm\sqrt{b^2-4ac}}{2a}.
  \end{eqnarray*}

  If $b^2-4ac  < 0$, 
  then the system has oscillations and
  \begin{eqnarray*}
    r & = & \frac{-b}{2a} \pm i \frac{\sqrt{4ac-b^2}}{2a}.
  \end{eqnarray*}
   The general solution is 
  \begin{eqnarray*}
    y(t) & = & e^{\frac{-b}{2a}} \left(
           A_1 \cos\left(\frac{\sqrt{4ac-b^2}}{2a}\right)
           +A_2 \sin\left(\frac{\sqrt{4ac-b^2}}{2a}\right)
           \right) 
  \end{eqnarray*}
 
\end{frame}

\subsection{Examples}

\begin{frame}
  \frametitle{Example}

  \begin{eqnarray*}
    2y'' + y' + 3y & = & 0.
  \end{eqnarray*}
  
  \uncover<2->
  {
    \begin{eqnarray*}
      2r^2+r+3 & = & 0, \\
      r & = & \frac{-1\pm\sqrt{1-24}}{4}, \\
      & = & \frac{-1}{4} \pm i\frac{\sqrt{23}}{4}, \\
      y & = & e^{-t/4} 
      \lp A_1 \cos\lp \frac{\sqrt{23}}{4} t \rp + A_2 \sin\lp \frac{\sqrt{23}}{4} t \rp \rp.
    \end{eqnarray*}

    It decays like $e^{-t/4}$ and oscillates with a period of $\frac{8\pi}{\sqrt{23}}$.
  }

\end{frame}


\begin{frame}
  \frametitle{Example}

  \begin{eqnarray*}
    y'' + 4 y' + 8y & = & 0.
  \end{eqnarray*}

  \uncover<2->
  {
    \begin{eqnarray*}
      r^2 + 4r + 8 & = & 0, \\
      r & = & \frac{-4\pm\sqrt{16-32}}{2} \\
      & = & -2 \pm 2i \\
      y & = & e^{-2t} \lp A_1 \cos(2t) + A_2 \sin(2t) \rp.
    \end{eqnarray*}
  }

\end{frame}


\begin{frame}
  \frametitle{Example}

  \begin{eqnarray*}
    y^v - 4 y'& = & 0.
  \end{eqnarray*}

  \uncover<2->
  {
    \begin{eqnarray*}
      r^5 - 4r & = & 0, \\
      r & = & 0 \\
      \mathrm{or~} r^4 & = & 4 \\
      & = & \sqrt{2},~\sqrt{2}e^{i \pi/2}, ~ \sqrt{2}e^{i \pi},~ \sqrt{2}e^{i 3\pi/2} \\
      & = & \sqrt{2},~i\sqrt{2},~-\sqrt{2},~-i\sqrt{2} \\
      y & = & C_1 + C_2 e^{\sqrt{2}t} + C_3 e^{-\sqrt{2}t} + 
      A_1 \cos(\sqrt{2}t) + A_2 \sin(\sqrt{2}t).
    \end{eqnarray*}
  }


\end{frame}

\subsection{An LRC Circuit}

\begin{frame}
  \frametitle{An LRC Circuit}

  An LRC circuit has a 30,0000 ohm resistor, a $5\times 10^{-6}$ F
  capacitor, and a 2000 Henries Inductor. Find the general equation
  describing  the charge in the capacitor at any time.

  \uncover<2->
  {
    \begin{eqnarray*}
      2000 Q'' + 30,000 Q' + \frac{Q}{5\times 10^{-6}} & = & 0 \\
      Q'' + 15 Q' + 100 Q & = & 0 \\
      r^2 + 15 r + 100 & = & 0 
    \end{eqnarray*}
    \begin{eqnarray*}
      r & = & \frac{-15\pm\sqrt{15^2-400}}{2} \\
      r & = & \frac{-15}{2} \pm i\frac{5\sqrt{7}}{2} \\
      Q & = & e^{-15/2 t} 
      \lp A_1 \cos\lp \frac{5\sqrt{7}}{2} t\rp + A_2 \sin\lp \frac{5\sqrt{7}}{2} t \rp \rp.
    \end{eqnarray*}
    
  }

\end{frame}

\subsection{Design a Spring Mass System}

\begin{frame}
  \frametitle{Design a Spring Mass System}

  Design a spring mass system so that a spring with spring constant
  $k=0.2$ N/m will oscillate a mass through two oscillations every
  three seconds, and the amplitude should decay like $e^{-t/2}$.

  \uncover<2->
  {
    \begin{eqnarray*}
      m x'' + bx' + 0.2x & = & 0. \\
      m r^2 + br + 0.2 & = & 0, \\
      r & = & \frac{-b\pm\sqrt{b^2-.8m}}{2m}, \\
      r & = & \frac{-b}{2m} \pm i \frac{\sqrt{.8m - b^2}}{2m}.
    \end{eqnarray*}
    
  }

\end{frame}


\begin{frame}

  The solution satisfies the following equation
  \begin{eqnarray*}
    x & = & e^{-b/2m t} 
    \lp A_1 \cos\lp\frac{\sqrt{.8m - b^2}}{2m} t\rp + A_2 \sin\lp\frac{\sqrt{.8m - b^2}}{2m} t \rp \rp.
  \end{eqnarray*}

  First the exponential term should satisfy
  \begin{eqnarray*}
    -\frac{b}{2m} t & = & -\frac{t}{2} \\
    \Rightarrow b & = & m.
  \end{eqnarray*}

  Second, the frequency should satisfy
  \begin{eqnarray*}
    \frac{\sqrt{.8m - b^2}}{2m} \frac{3}{2} & = & 2 \pi.
  \end{eqnarray*}

\end{frame}


\begin{frame}

  If we let $b=m$ and solve for $m$ in the second equation we get
  \begin{eqnarray*}
    m & = & \frac{0.8}{1 + \frac{64 \pi^2}{9}}, \\
    b & = & \frac{0.8}{1 + \frac{64 \pi^2}{9}}.  
  \end{eqnarray*}

\end{frame}


% LocalWords:  Clarkson pausesection hideothersubsections
