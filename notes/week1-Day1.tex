\part{Introduction}
\lecture{Introduction}{Introduction}
\section{Introduction}

\title{Ordinary Differential Equations}
\subtitle{Introduction}
\date{27 Aug 2012}

\begin{frame}
  \titlepage
\end{frame}

\begin{frame}
  \frametitle{Outline}
  \tableofcontents[pausesection,hideothersubsections,sectionstyle=show/hide]
\end{frame}

\subsection{Syllabus}
\begin{frame}
  \frametitle{Syllabus}

  Read the syllabus! The syllabus will take precedence over any
  discrepancies here. It includes \textbf{all} of our policies. We
  will not cover them all here.

  Faculty: \\
  \begin{tabular}{l@{\hspace{3em}}l@{\hspace{3em}}l}
    Kelly Black                      & Guangming Yao    \\
    361B Science Center              & 363 Science Center   \\
    268-3831                         & 268- 6496\\
    kjblack@gmail.com                &  gyao@clarkson.edu\\ [10pt]
  \end{tabular}

\end{frame}


\begin{frame}
  \frametitle{Materials}

  \begin{itemize}
  \item {\em Differential Equations and Linear Algebra}, second
    edition
  \item Recitation manual \\
    The UPS Store, \\
    200 Market St, \\
    315.265.4565, \\
    store5986$@$theupsstore.com
  \item Turning Point Technologies ResponseCard RF LCD clicker \\
    \url{https://store.turningtechnologies.com} \\
    The Clarkson University promotional code is Z0k0.
  \end{itemize}
  
\end{frame}


\begin{frame}
  \frametitle{Test Dates}

   The tests will be given 18 September, 30 October,
  and 29 November. The exams will take place in the evenings from
  7:00pm to 8:20pm. You should bring your own pencils.  The professor
  will not have any spare materials. The location will vary depending
  on your recitation section, and your room assignments will be given
  prior to the exams.
  
\end{frame}


\begin{frame}
  \frametitle{Grading}

   \begin{tabular}[t]{rl}
    45\% & 3 Tests. \\
    20\% & Final Exam. \\
    15\% & Homework and Webwork. \\
    15\% & Quiz. \\
     5\% & Clickers
  \end{tabular}

  \begin{itemize}
  \item The right to miss a scheduled exam and take a make up exam can
    be awarded only by your professor.
  \item If for some reason you must miss an exam, you must apply in
    writing {\bf before} the exam.
  \item In case of emergency contact the professor as soon as possible
    and provide documentation to confirm why you cannot take part in
    the exam. 
  \item An unexcused absence will result in a grade of zero on the
    exam.
  \end{itemize}


  
\end{frame}


\begin{frame}
  \frametitle{Academic Accommodations}

  If you require any kind of special
  accommodation please see your professor.  Requests for academic
  accommodations must be made during the first three weeks of the
  semester, except for unusual circumstances.  Students must register
  with the Office of Accommodative Services, located in the Student
  Success Center, 110 ERC, to verify their eligibility for appropriate
  accommodations.
  
\end{frame}

\begin{frame}
  \frametitle{Clickers}

  We will start using the clickers this Friday. Each day one or more
  questions will be asked.

  You need to register your clicker. Go to our course page on moodle:
  \begin{enumerate}
  \item Choose the ``clicker ID'' option at the top of the page.
  \item Enter your clicker Device ID (found on the back of the
    clicker.)
  \item Click on the submit button.
  \item Click on the ``next'' button.
  \item Click on ``Submit all and finish.''
  \item Confirm your choice.
  \item Click on ``Finish Review'' at the bottom of the page.
  \end{enumerate}

  If you do not click on ``Finish Review'' your device ID may not be saved.
   
  
\end{frame}


\begin{frame}
  \frametitle{Recitations}

  \begin{itemize}
  \item Take place every Thursday.
  \item Will have a quiz almost every session.
  \item Some days the quiz will consist of the pre-class activity in
    your lab manual. (Random!)
  \item The recitation is where you \textbf{DO} things. It is where
    you will learn the most in this class.
  \end{itemize}
\end{frame}

\subsection{What is a Differential Equation}


\begin{frame}
  \frametitle{What is a DE?}

  Given
  \begin{eqnarray*}
    y'(x) & = & y(x),
  \end{eqnarray*}
  what is $y(x)=?$

  \begin{eqnarray*}
    \deriv{y(x)}{x} & = & y(x)
  \end{eqnarray*}

\end{frame}


\begin{frame}{Question:}
  Given
  \begin{eqnarray*}
    y'' + 3y' +2y & = & 0
  \end{eqnarray*}

  what is $y(x)$?

  \uncover<2->{
    Note: we leave often leave off the function notation.
    }

  \uncover<3->{
    Why? We are lazy!
    }

\end{frame}

\begin{frame}
  \frametitle{Notation}
  \begin{eqnarray*}
    \dot{y} & = & \deriv{y}{t} \\
    \only<2->{\ddot{y} & = & \derivTwo{y}{t} }\\
    \only<3->{y' & = & \mathrm{depends.... usually ~~} \deriv{y(x)}{x} }\\
    \only<4->{y'' & = & \derivTwo{y(x)}{x} }
  \end{eqnarray*}
\end{frame}

\begin{frame}
  \frametitle{Nomenclature}
  
  \vfill

  $\deriv{y}{x}$ - then ``ordinary differential equation.''

  \vfill

  $\frac{\partial y}{\partial x}$ - then ``partial differential
  equation.''

  \vfill

  Order is the highest number of derivatives:
  \begin{eqnarray*}
    y'' - 3 y' + 2y & = & 0, \mathrm{second~order} \\
    y'  & = & 4y, \mathrm{first~order} 
  \end{eqnarray*}

  \vfill


\end{frame}

\subsection{Modeling}


\begin{frame}
  \frametitle{Modeling}

  Why bother?

  Many ``mathematical models'' provide a relationship between rates.

  ex: Newton's Second Law, ``$\vec{F} = m \vec{a}$'' In 1 dimension:
  \begin{eqnarray*}
    m\mathrm{~(acceleration)} & = & \sum_i \lp F \rp_i, \\
    m \ddot{x} & = & \sum_i \lp F \rp_i
  \end{eqnarray*}

  
\end{frame}


\begin{frame}
  \frametitle{Circuit}
  
  The voltage across a resistor is proportional to the current flowing
  through it.

  \only<2->
  {
    There is some number, R, where
    \begin{eqnarray*}
      V & = & IR
    \end{eqnarray*}
  }
\end{frame}

\begin{frame}
  \frametitle{Proportionality}
  
  If $a$ is proportional to $b$ then there is a constant, $k$, where 
  \begin{eqnarray*}
    a  & = & k \cdot b
  \end{eqnarray*}

  if $a$ is inversely proportional to $b$ then there is a constant $c$
  where 
  \begin{eqnarray*}
    a & = & c \frac{1}{b}
  \end{eqnarray*}
\end{frame}

\begin{frame}
  \frametitle{Example - Newton's Law of Cooling}

  The rate of change of the temperature of an object is proportional
  to the difference between the object and its surroundings (the
  ambient temperature).

  \uncover<2->
  {

    Define:
    \begin{itemize}
    \item $T(t)$ is the temperature at a time $t$.
    \item $A$ is the ambient temperature (the temperature of the
      surroundings).
    \item The rate of change of the temperature is $\frac{dT(t)}{dt}$.
    \end{itemize}

  }

  \uncover<3->
  {

    There is some constant number, $k$, where 
    \begin{eqnarray*}
      \frac{dT(t)}{dt} & = & -k (T-A).
    \end{eqnarray*}

  }


\end{frame}


\begin{frame}
  \frametitle{Newton's Law of Cooling}

  The rate of change of the temperature of an object is proportional
  to the difference between the object and its surroundings (the
  ambient temperature).

    There is some constant number, $k$, where 
    \begin{eqnarray*}
      \frac{dT(t)}{dt} & = & -k (T-A).
    \end{eqnarray*}

    Question: What is the temperature at any given time?

\end{frame}


% LocalWords:  Clarkson pausesection hideothersubsections sectionstyle Yao
% LocalWords:  Guangming ResponseCard Webwork moodle
