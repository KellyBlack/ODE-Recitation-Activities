\part{Trace-Determinant-Plane}
\lecture{Trace-Determinant Plane}{Trace-Determinant-Plane}
\section{Trace-Determinant Plane}


\title{Ordinary Differential Equations}
\subtitle{Math 232 - Week 12, Day 1}
\date{14 November 2012}

\begin{frame}
  \titlepage
\end{frame}

\begin{frame}
  \frametitle{Outline}
  \tableofcontents[pausesection,hideothersubsections]
\end{frame}


\subsection{Behavior of Solutions to DEs}


\begin{frame}
  \frametitle{Behavior of Solutions to DEs}

  The eigen values determien the stability characteristics of the
  system of DEs.

  If the eigen values are real:
  \begin{itemize}
  \item $\lambda_2 < \lambda_1 < 0$ - Attracting
  \item $\lambda_2 < 0 < \lambda_1$ - Saddle
  \item $0 < \lambda_2 < \lambda_1$ - Repelling
  \end{itemize}

  If the eigen values are complex, $\lambda_{1,2}=a+ib$:
  \begin{itemize}
  \item $a<0$ - Attracting spiral
  \item $a>0$ - Repelling spiral
  \item $a=0$ - Center
  \end{itemize}

\end{frame}

\subsection{Special Case of $2\times 2$ Systems}

\begin{frame}
  \frametitle{Special Case of $2\times 2$ Systems}

  \begin{eqnarray*}
    \deriv{~}{t} \vec{x} & = & \arrayTwo{a}{b}{c}{d} \vec{x}, \\
    & = & A \vec{x}.
  \end{eqnarray*}

  \uncover<2->
  {
    Find the eigen values:
    \begin{eqnarray*}
      \det\lp\arrayTwo{a-\lambda}{b}{c}{d-\lambda}\rp & = & 
      \lambda^2 - (a+d) \lambda + (ad-bc), \\
      & = & \lambda^2 - T\lambda + D.
    \end{eqnarray*}
  }

  \uncover<3->
  {
    \begin{eqnarray*}
      \Rightarrow \lambda_{1,2} & = & \frac{T\pm\sqrt{T^2-4D}}{2}.
    \end{eqnarray*}
  }


\end{frame}


\begin{frame}
  \frametitle{Different Cases}

  If $T^2-4D>0$
  \begin{itemize}
  \item $T<0$ and $D>0$ - Attracting
  \item $T<0$ and $D<0$ - Saddle
  \item $T>0$ and $D<0$ - Saddle
  \item $T>0$ and $D>0$ - Repelling
  \end{itemize}

  If $T^2-4D<0$ - Complex case:
  \begin{itemize}
  \item $T<0$ - Attracting Spiral
  \item $T=0$ - Centered
  \item $T>0$ - Repelling Spiral
  \end{itemize}


\end{frame}

\subsection{Examples}

\begin{frame}
  \frametitle{Example}

  \begin{eqnarray*}
    \deriv{~}{t} \vec{x} & = & \arrayTwo{2}{4}{1}{3} \vec{x}.
  \end{eqnarray*}

\end{frame}


\begin{frame}
  \frametitle{Example}

  \begin{eqnarray*}
    \deriv{~}{t} \vec{x} & = & \arrayTwo{1}{4}{1}{3} \vec{x}.
  \end{eqnarray*}

\end{frame}


\begin{frame}
  \frametitle{Example}

  What is the behavior of
  \begin{eqnarray*}
    \deriv{~}{t} \vec{x} & = & \arrayTwo{1}{k}{1}{0} \vec{x}.
  \end{eqnarray*}
  for all values of $k$?
  

\end{frame}


\begin{frame}
  \frametitle{Example}

  What is the behavior of
  \begin{eqnarray*}
    \deriv{~}{t} \vec{x} & = & \arrayTwo{-2}{k}{k}{0} \vec{x}.
  \end{eqnarray*}
  for all values of $k$?


\end{frame}


\begin{frame}
  \frametitle{Example}

  What is the behavior of
  \begin{eqnarray*}
    \deriv{~}{t} \vec{x} & = & \arrayTwo{1}{k}{1}{-2} \vec{x}.
  \end{eqnarray*}
  for all values of $k$?


\end{frame}


\begin{frame}
  \frametitle{Example}

  What is the behavior of
  \begin{eqnarray*}
    \deriv{~}{t} \vec{x} & = & \arrayTwo{k}{2}{-1}{0} \vec{x}.
  \end{eqnarray*}
  for all values of $k$?


\end{frame}




% LocalWords:  Clarkson pausesection hideothersubsections
