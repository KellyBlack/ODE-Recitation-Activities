\part{Complex-Eigen-Values}
\lecture{Complex Eigenvalues}{Complex-Eigen-Values}
\section{Complex Eigenvalues}


\title{Complex Valued Eigenvalues}
\subtitle{Spiral Solutions}
\date{6 November 2013}

\begin{frame}
  \titlepage
\end{frame}

\begin{frame}
  \frametitle{Outline}
  \tableofcontents[currentsection]
\end{frame}


\subsection{Complex Valued Eigenvalues}


\begin{frame}
  \frametitle{Complex Valued Eigenvalues}

  Recall:
  \begin{eqnarray*}
    e^{a+bi} & = & e^a e^{bi}, \\
    & = & e^a \lp \cos(b) + i \sin(b) \rp.
  \end{eqnarray*}

  \vfill

  The complex conjugate:
  \begin{eqnarray*}
    \overline{a+ib} & = & a-i b.
  \end{eqnarray*}

  \vfill

  If A is a real valued matrix with complex eigenvalues, then
  the eigenvalues and eigenvectors come in \blueText{complex
    conjugates pairs}.

\end{frame}

\iftoggle{clicker}{%
\begin{frame}
  \frametitle{Clicker Quiz}

   \ifnum\value{clickerQuiz}=1{%
   Given $A=\arrayTwo{-8}{30}{-3}{10}$. Find the eigenvalues. \\ [12pt]

   \begin{tabular}{ll}
     A: & $\lambda_{1,2} = 1 \pm 3i.$ \\ [12pt]
     B: & $\lambda_{1} = 4, \lambda_2 = -2.$ \\ [12pt]
     C: & $\lambda_{1,2} = 1.$ 
   \end{tabular}

     \vfill
   }\fi

   \ifnum\value{clickerQuiz}=2{%
   Given $A=\arrayTwo{-8}{30}{-3}{10}$. Find the eigenvalues. \\ [12pt]

   \begin{tabular}{ll}
     A: & $\lambda_{1,2} = 1 \pm 3i.$ \\ [12pt]
     B: & $\lambda_{1} = 4, \lambda_2 = -2.$ \\ [12pt]
     C: & $\lambda_{1,2} = 1.$ 
   \end{tabular}


   \vfill
   }\fi

  \ifnum\value{clickerQuiz}=3{%
   Given $A=\arrayTwo{-8}{30}{-3}{10}$. Find the eigenvalues. \\ [12pt]

   \begin{tabular}{ll}
     A: & $\lambda_{1,2} = 1 \pm 3i.$ \\ [12pt]
     B: & $\lambda_{1} = 4, \lambda_2 = -2.$ \\[12pt]
     C: & $\lambda_{1,2} = 1.$ 
   \end{tabular}

  \vfill
 }\fi
\end{frame}
}


\begin{frame}
  \frametitle{Example}

  \begin{eqnarray*}
    \deriv{~}{t} \vec{x} & = & \arrayTwo{-8}{30}{-3}{10} \vec{x}.
  \end{eqnarray*}

  \uncover<2->
  {
    Find the eigenvalues:
    \begin{eqnarray*}
      \det\lp\arrayTwo{-8-\lambda}{30}{-3}{10-\lambda}\rp 
      & = & \lambda^2-2\lambda+10 \\
      \lambda_{1,2} & = & 1 \pm 3i.
    \end{eqnarray*}
  }

\end{frame}


\begin{frame}
  \frametitle{Find the Eigenvectors}

  $\lambda_1 = 1+3i$:
  \begin{eqnarray*}
    \arrayTwo{-9-3i}{30}{-3}{9-3i} \vecTwo{x}{y} & = & \vecTwo{0}{0} \\
    x & = & \lp 3 - i \rp y, \\
    \Rightarrow \vec{v}_1 & = & y \lp \vecTwo{3}{1} + i\vecTwo{-1}{0} \rp.
  \end{eqnarray*}

  \uncover<2->
  {
    $\lambda_2 = 1-3i$:
    \begin{eqnarray*}
      \arrayTwo{-9+3i}{30}{-3}{9+3i} \vecTwo{x}{y} & = & \vecTwo{0}{0} \\
      x & = & \lp 3+i \rp y, \\
      \Rightarrow \vec{v}_2 & = & y \lp \vecTwo{3}{1} -i \vecTwo{-1}{0} \rp.
    \end{eqnarray*}
  }

\end{frame}


\begin{frame}
  \frametitle{Determine the Solution}

  \begin{eqnarray*}
    \uncover<1->
    {
      \vec{x} & = & A e^{(1+3i)t} \lp \vecTwo{3}{1} + i\vecTwo{-1}{0} \rp
      + B e^{(1-3i)t} \lp \vecTwo{3}{1} - i\vecTwo{-1}{0} \rp \\
    }
    \uncover<2->
    {
      \vec{x} & = & A e^{(1+3i)t} \lp \underbrace{\vecTwo{3}{1}}_{\vec{p}} + 
      i\underbrace{\vecTwo{-1}{0}}_{i\vec{q}} \rp
      + B e^{(1-3i)t} \lp \underbrace{\vecTwo{3}{1}}_{\vec{p}} -
      i\underbrace{\vecTwo{-1}{0}}_{i\vec{q}} \rp \\
    }
    \uncover<3->
    {
      & = & 
        e^{t} \lp A \redText{e^{3it}} \lp\vec{p}+i\vec{q}\rp +  
                 B \blueText{e^{-3it}} \lp\vec{p}-i\vec{q}\rp \rp \\ [12pt]
     }
    \uncover<4->
    {
      & = & 
        e^{t} \lp A \redText{\lp\cos(3t)+i\sin(3t)\rp} \lp\vec{p}+i\vec{q}\rp +  \right. \\
      & & \left. B \blueText{\lp\cos(3t)-i\sin(3t)\rp} \lp\vec{p}-i\vec{q}\rp \rp \\
     }
  \end{eqnarray*}


\end{frame}

\begin{frame}{The Solution}

  \begin{eqnarray*}
    \vec{x} & = & e^{t} \lp A \redText{\lp\cos(3t)+i\sin(3t)\rp \lp\vec{p}+i\vec{q}\rp} +  \right. \\
            & & \left. B \blueText{\lp\cos(3t)-i\sin(3t)\rp \lp\vec{p}-i\vec{q}\rp} \rp \\ [12pt]
    \uncover<2->{%
      & = & e^{t} \lp \redText{A \cos(3t) \vec{p} + i A \sin(3t) \vec{p} + iA\cos(3t) \vec{q} - A \sin(3t) \vec{q}} \right. \\
      & & \left. + \blueText{B \cos(3t) \vec{p} - i B \sin(3t) \vec{p} - iB\cos(3t) \vec{q} - B \sin(3t) \vec{q}} \rp  \\ [12pt]
    }
    \only<3>
    {
      & = & 
      e^{t} \redText{(A+B)} \lp \cos(3t)\vec{p} - \sin(3t)\vec{q} \rp  \\
      &  & + e^{t} \blueText{(A-B)} \lp\cos(3t) \vec{q}+\sin(3t)\vec{p} \rp  \\ [12pt]
     }
    \uncover<4->
    {
      & = & 
         e^{t} \redText{C_1} \lp \cos(3t)\vec{p} - \sin(3t)\vec{q} \rp  \\
      &  & +  e^{t} \blueText{C_2} \lp\cos(3t) \vec{q}+\sin(3t)\vec{p} \rp  \\ [12pt]
     }
    \uncover<5->
    {
      & = & 
        C_1 e^{t} \lp \cos(3t)\vecTwo{3}{1} - \sin(3t)\vecTwo{-1}{0} \rp  \\
      &  & +  C_2 e^{t} \lp\cos(3t) \vecTwo{-1}{0}+\sin(3t)\vecTwo{3}{1}\rp 
     }
   \end{eqnarray*}
  
\end{frame}

\begin{frame}
  \frametitle{The solutions spiral out clockwise}
   %(The solutions spiral out clockwise.)
  
 \only<2>{\centerline{\includegraphics[width=9.5cm]{img/complexPhaseOne}}}
  \only<3>{\centerline{\includegraphics[width=9.5cm]{img/complexPhaseTwo}}}
  \only<4>{\centerline{\includegraphics[width=9.5cm]{img/complexPhaseThree}}}
\end{frame}

\begin{frame}
  \frametitle{Remark}

  \begin{eqnarray*}
    \vec{x} & = & Ae^{(a+bi)t} \lp \vec{p} + i \vec{q} \rp 
             + Be^{(a-bi)t}\lp \vec{p} - i \vec{q} \rp \\
    & = & Ae^{at} \lp \cos(bt) + i \sin(bt) \rp \lp \vec{p} + i \vec{q} \rp \\
    & & + Be^{at}\lp \cos(bt) - i \sin(bt) \rp \lp \vec{p} - i \vec{q} \rp \\
  & = & C_1 e^{at} \lp \cos(bt) \vec{p} - \sin(bt) \vec{q} \rp
  + C_2 e^{at} \lp \cos(bt) \vec{q} + \sin(bt) \vec{p} \rp.
  \end{eqnarray*}

\end{frame}



\subsection{Characterization of Solutions}

\begin{frame}
  \frametitle{Characterization of Solutions}

  \begin{eqnarray*}
    \deriv{~}{t} \vec{x} & = & A \vec{x}.
  \end{eqnarray*}

  If $\lambda$'s are complex, $\lambda = a \pm ib$:
  \begin{enumerate}
  \item If $a>0$ then it is a repelling spiral (spiral source).
  \item If $a<0$ then it is an attracting spiral (spiral sink).
  \item If $a=0$ then the solutions are neutrally stable.
  \end{enumerate}

\end{frame}


\subsection{Nullclines}

\begin{frame}
  \frametitle{Nullclines}

  Given a solution to a differential equation,
  \begin{eqnarray*}
    \vec{x} & = & \vecTwo{x(t)}{y(t)},
  \end{eqnarray*}
  if
  \begin{itemize}
  \item $x'(t)=0$ then the trajectory is ``vertical'' at time $t$,
  \item $y'(t)=0$ then the trajectory is ``horizontal'' at time $t$.
  \end{itemize}

\end{frame}


\begin{frame}
  \frametitle{Definition of the Nullclines}

  \begin{definition}[Zero Vertical Change]
    The set of points, $\vecTwo{x}{y}$ where $x'(t)=0$ is called the
    v-nullcline. 
  \end{definition}


  \begin{definition}[Zero Horizontal Change]
    The set of points, $\vecTwo{x}{y}$ where $y'(t)=0$ is called the
    h-nullcline.
  \end{definition}


\end{frame}


\begin{frame}
  \frametitle{Example}
  
  In the previous example we had
  \begin{eqnarray*}
    \deriv{~}{t} \vec{x} & = & \arrayTwo{-8}{30}{-3}{10} \vec{x}.
  \end{eqnarray*}

  \uncover<2->
  {
    Find the v-nullcline:
    \begin{eqnarray*}
      \deriv{x}{t} & = & -8x + 30y, \\
      0 & = & -8x + 30y, \\
      y & = & \frac{8}{30} x.
    \end{eqnarray*}
  }

  \uncover<3->
  {
    Find the h-nullcline:
    \begin{eqnarray*}
      \deriv{y}{t} & = & -3x + 10y, \\
      0 & = & -3x + 10y, \\
      y & = &  \frac{3}{10} x.
    \end{eqnarray*}
  }


\end{frame}


\begin{frame}
  \frametitle{Example - Graphical View}

  \centerline{\includegraphics[scale=0.15]{img/complexPhaseNullClines}}  

\end{frame}

\begin{frame}
  \frametitle{Extra Example}

  Determine the solution to the system of equations given by
  \begin{eqnarray*}
    \deriv{~}{t} \vec{x} & = & \arrayTwo{1}{1}{-2}{3} \vec{x}.
  \end{eqnarray*}

  \uncover<2->
  {
    Find the eigenvalues:
    \begin{eqnarray*}
      \det\lp\arrayTwo{1-\lambda}{1}{-2}{3-\lambda}\rp 
      & = & \lambda^2-4\lambda+5 \\
      \lambda_{1,2} & = & 2 \pm i.
    \end{eqnarray*}
  }

\end{frame}

\begin{frame}
  \frametitle{Find the Eigenvectors}

  $\lambda_1 = 2+i$:
  \begin{eqnarray*}
    \arrayTwo{-1-i}{1}{-2}{1-i} \vecTwo{x}{y} & = & \vecTwo{0}{0} \\
    y & = & \lp 1+i \rp x, \\
    \Rightarrow \vec{v}_1 & = & x \lp \vecTwo{1}{1} + i \vecTwo{0}{1} \rp.
  \end{eqnarray*}

  \uncover<2->
  {
    $\lambda_2 = 2-i$:
    \begin{eqnarray*}
      \arrayTwo{-1+i}{1}{-2}{1+i} \vecTwo{x}{y} & = & \vecTwo{0}{0} \\
      y & = & \lp 1-i \rp x, \\
      \Rightarrow \vec{v}_2 & = & x \lp \vecTwo{1}{1} - i \vecTwo{0}{1} \rp.
    \end{eqnarray*}

  }

\end{frame}


\begin{frame}
  \frametitle{Determine the Solution}

  \begin{eqnarray*}
    \uncover<1->
    {
      \vec{x} & = & A e^{(2+i)t} \lp \underbrace{\vecTwo{1}{1}}_{\vec{p}} + i
      \underbrace{\vecTwo{0}{1}}_{\vec{q}} \rp
      + B e^{(2-i)t} \lp \underbrace{\vecTwo{1}{1}}_{\vec{p}} -
      i \underbrace{\vecTwo{0}{1}}_{\vec{q}} \rp \\[12pt]
    }
    \uncover<2->
    {
      & = & e^{2t} \lp
        A \lp  \cos(t) + i \sin(t) \rp \lp \vec{p} + i \vec{q} \rp 
          + 
        B \lp \cos(t) - i \sin(t) \rp \lp \vec{p} - i \vec{q} \rp 
      \rp\\[12pt]
    }
    \uncover<3->
    {
  & = & C_1 e^{2t} \lp \cos(t) \vec{p} - \sin(t) \vec{q} \rp
  +  C_2 e^{2t} \lp \cos(t) \vec{q} + \sin(t) \vec{p} \rp.
   }
  \end{eqnarray*}

  (The solutions spiral out clockwise.)


\end{frame}



% LocalWords:  Clarkson pausesection hideothersubsections Nullclines nullcline
