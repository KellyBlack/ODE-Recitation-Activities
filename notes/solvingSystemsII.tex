\part{Solutions-to-Systems-of-DEs}
\lecture{Solutions to Systems of DEs}{Solutions-to-Systems-of-DEs}
\section{Solutions to Systems of DEs}


\title{Ordinary Differential Equations}
\subtitle{Math 232 - Week 11, Day 2}
\date{9 November 2012}

\begin{frame}
  \titlepage
\end{frame}

\begin{frame}
  \frametitle{Outline}
  \tableofcontents[pausesection,hideothersubsections]
\end{frame}


\subsection{Solutions to Systems of DEs}


\begin{frame}
  \frametitle{What is a solution to a System of DEs?}

  \begin{eqnarray*}
    \deriv{~}{t} \vec{x} & = & A \vec{x}.
  \end{eqnarray*}

  \uncover<2->
  {
    Assume that 
    \begin{eqnarray*}
      \vec{x} & = & c e^{\lambda t} \vec{v}.
    \end{eqnarray*}
  }


\end{frame}


\begin{frame}
  \frametitle{Substitute it into the equation}

  \begin{eqnarray*}
    \deriv{~}{t} \vec{x} & = & c \lambda e^{\lambda t} \vec{v} \\
    \Rightarrow c  e^{\lambda t} A \vec{v} & = & c \lambda e^{\lambda t} \vec{v}, \\
    A \vec{v} & = & \lambda \vec{v}.
  \end{eqnarray*}

  This is an eigen vector/eigen value problem!

\end{frame}


\begin{frame}
  \frametitle{Homogeneous Equations}

  Given
  \begin{eqnarray*}
    \deriv{~}{t} \vec{x} & = & A \vec{x},
  \end{eqnarray*}
  where $A$ is an $n\times n$ matrix, find the eigen vectors and eigen
  values. \textbf{If} there are $n$ linearly independent eigen vectors
  then 
  \begin{eqnarray*}
    \vec{x} & = & c_1 e^{\lambda_1 t} \vec{v}_1 + c_2 e^{\lambda_2 t} \vec{v}_2 +
    \cdot + c_n e^{\lambda_n t} \vec{v}_n.
  \end{eqnarray*}

\end{frame}

\subsection{Examples}

\begin{frame}
  \frametitle{Example}

  \begin{eqnarray*}
    \deriv{~}{t} \vec{x} & = & \arrayTwo{5}{24}{-2}{-9} \vec{x}.
  \end{eqnarray*}

  \uncover<2->
  {
    First find the eigen values,
    \begin{eqnarray*}
      0 & = & \det\lp\arrayTwo{5-\lambda}{24}{-2}{-9-\lambda}\rp, \\
      & = & (\lambda+2)(\lambda+1).
    \end{eqnarray*}
    The eigen values are $\lambda_1=-1$ and $\lambda_2=-3$.
  }

\end{frame}


\begin{frame}
  \frametitle{Find the eigen vectors}

  $\lambda_1 = -1$:
  \begin{eqnarray*}
    \arrayTwo{6}{24}{-2}{-8} \vecTwo{x}{y} & = & \vecTwo{0}{0}, \\
    6x + 24y & = & 0, \\
    -2x - 8y & = & 0, \\
    \uncover<2->
    {
      \Rightarrow \vec{v}_1 & = & \vecTwo{-4}{1}.
    }
  \end{eqnarray*}

  \uncover<3->
  {
    $\lambda_2 = -3$:
    \begin{eqnarray*}
      \arrayTwo{8}{24}{-2}{-6} \vecTwo{x}{y} & = & \vecTwo{0}{0}, \\
      8x + 24y & = & 0, \\
      -2x - 6y & = & 0, \\
      \uncover<4->
      {
        \Rightarrow \vec{v}_2 & = & \vecTwo{-3}{1}.
      }
    \end{eqnarray*}
  }

\end{frame}


\begin{frame}
  \frametitle{Assemble the Solution}

  \begin{eqnarray*}
    \vec{x}(t) & = & c_1 e^{-t} \vecTwo{-4}{1} + c_2 e^{-3t} \vecTwo{-3}{1}.
  \end{eqnarray*}

\end{frame}

\begin{frame}
  \frametitle{Example}

  \begin{eqnarray*}
    x'' + 5x' + 6x & = & 0, \\
    \Rightarrow x_h & = & C_1 e^{-3t} + C_2 e^{-2t}.
  \end{eqnarray*}

  Let $x'=y$,
  \begin{eqnarray*}
    x' & = & y, \\
    y' & = & -6x - 5y,
  \end{eqnarray*}
  which leads to the following system
  \begin{eqnarray*}
    \deriv{~}{t} \vec{x} & = & \arrayTwo{0}{1}{-6}{-5} \vec{x}.
  \end{eqnarray*}
\end{frame}

\begin{frame}
    \frametitle{First find the eigen values}
    \begin{eqnarray*}
      0 & = & \det\lp\arrayTwo{-\lambda}{1}{-6}{-5-\lambda}\rp, \\
      & = & (\lambda+2)(\lambda+3).
    \end{eqnarray*}
    The eigen values are $\lambda_1=-2$ and $\lambda_2=-3$.

\end{frame}


\begin{frame}
  \frametitle{Find the eigen vectors}

  $\lambda_1 = -2$:
  \begin{eqnarray*}
    \arrayTwo{2}{1}{-6}{-3} \vecTwo{x}{y} & = & \vecTwo{0}{0}, \\
    2x + y & = & 0, \\
    -6x - 3y & = & 0, \\
    \uncover<2->
    {
      \Rightarrow \vec{v}_1 & = & \vecTwo{1}{-2}.
    }
  \end{eqnarray*}

  \uncover<3->
  {
    $\lambda_2 = -3$:
    \begin{eqnarray*}
      \arrayTwo{3}{1}{-6}{-2} \vecTwo{x}{y} & = & \vecTwo{0}{0}, \\
      3x + y & = & 0, \\
      -6x - 2y & = & 0, \\
      \uncover<4->
      {
        \Rightarrow \vec{v}_2 & = & \vecTwo{1}{-3}.
      }
    \end{eqnarray*}
  }

\end{frame}


\begin{frame}
  \frametitle{Assemble the Solution}

  \begin{eqnarray*}
    \vec{x}(t) & = & c_1 e^{-2t} \vecTwo{1}{-2} + c_2 e^{-3t} \vecTwo{1}{-3}.
  \end{eqnarray*}

\end{frame}


\subsection{The Phase Plane}

\begin{frame}
  \frametitle{Phase Plane Solutions}


\end{frame}

\subsection{Repeated Eigen Values}

\begin{frame}
  \frametitle{Example}

  \begin{eqnarray*}
    \deriv{~}{t} \vec{x} & = & \arrayTwo{0}{-1}{4}{-4} \vec{x}.
  \end{eqnarray*}

  \uncover<2->
  {
    Find the eigen values:
    \begin{eqnarray*}
      0 & = & \det\lp\arrayTwo{-\lambda}{-1}{4}{-4-\lambda}\rp, \\
      & = & \lp \lambda + 2 \rp^2,
    \end{eqnarray*}
    The value of $\lambda$ is -2.
  }

\end{frame}


\begin{frame}
  \frametitle{Find the eigen vectors}

  \begin{eqnarray*}
    \arrayTwo{2}{1}{4}{2} \vec{x}{y} & = & \vecTwo{0}{0}, \\
    2x + y & = & 0, \\
    4x + 2y & = & 0.
  \end{eqnarray*}

  There is only one eigen vector,
  \begin{eqnarray*}
    \vec{v}_1 & = & \vecTwo{1}{-2}.
  \end{eqnarray*}

\end{frame}

\begin{frame}
  \frametitle{What to do?}

  Last time this happened we tried to multiply by $t$, and it
  worked. We will try that again...
  Assume that
  \begin{eqnarray*}
    \vec{x} & = & t e^{\lambda t} \vec{v} + e^{\lambda t} \vec{u}.
  \end{eqnarray*}

  \uncover<2->
  {
    Substitute back into the equation:
    \begin{eqnarray*}
      \deriv{~}{t} \vec{x} & = & \lambda t e^{\lambda t} \vec{v} + e^{\lambda t} \vec{v} 
         + \lambda e^{\lambda t} \vec{u}, \\
         & = & 
      \lambda t e^{\lambda t} \vec{v} + e^{\lambda t} \vec{v} + \lambda e^{\lambda t} \vec{u}, \\
      A\vec{x} & = & t e^{\lambda t} A \vec{v} + e^{\lambda t} A \vec{u}.
    \end{eqnarray*}
  }
  
\end{frame}


\begin{frame}
  \frametitle{Set The Two Expressions Equal}

    \begin{eqnarray*}
      \lambda t e^{\lambda t} \vec{v} + e^{\lambda t} \vec{v} + \lambda e^{\lambda t} \vec{u}
      & = & 
      t e^{\lambda t} A \vec{v} + e^{\lambda t} A \vec{u}.
    \end{eqnarray*}
  
    This has to be true for all $t$ so we have the following equations:
    \begin{eqnarray*}
      \lambda t e^{\lambda t} \vec{v} & = & t e^{\lambda t} A \vec{v}, \\
      \lambda \vec{v} & = & A \vec{v}.
    \end{eqnarray*}
    (This is the eigen value and eigen vector!)

    We also have 
    \begin{eqnarray*}
      e^{\lambda t} \vec{v} + \lambda e^{\lambda t} \vec{u} & = & e^{\lambda t} A \vec{u}, \\
      \vec{v} + \lambda \vec{u} & = &  A \vec{u}, \\
      \Rightarrow \lp A - \lambda I \rp \vec{u} & = & \vec{v}.
    \end{eqnarray*}

\end{frame}

\begin{frame}
  \frametitle{Back to the Example}

  In out example we have
  \begin{eqnarray*}
    \vec{x} & = & t e^{\lambda t} c_1 \vecTwo{1}{2} + e^{\lambda t} \vec{u}, \\
    \arrayTwo{2}{-1}{4}{-2} \vecTwo{x}{y} & = & \vecTwo{c_1}{2c_1}, \\
    2x-y & = & c_1 \\
    4x-2y & = & 2 c_1.
  \end{eqnarray*}

  The vector, $\vec{u}$, is the following:
  \begin{eqnarray*}
    \vec{u} & = & \vecTwo{c_2}{2c_2-c_1}.
  \end{eqnarray*}
  
\end{frame}

\begin{frame}
  \frametitle{Form the Solution}
  
  The solution to the differential equation is
  \begin{eqnarray*}
    \vec{x}(t) & = & t e^{-2t} c_1 \vecTwo{1}{2} + e^{-2t} \vecTwo{c_2}{2c_2-c_1}, \\
    & = & e^{-2t} c_1 \vecTwo{t}{2t-1} + e^{-2t} c_2 \vecTwo{1}{2}.
  \end{eqnarray*}
\end{frame}

\begin{frame}
  \frametitle{Example}

  \begin{eqnarray*}
    \deriv{~}{t} \arrayTwo{2}{0}{0}{2} \vec{x}.
  \end{eqnarray*}

  \uncover<2->
  {
    Find the eigen values,
    \begin{eqnarray*}
      0 & = & \det\lp\arrayTwo{2-\lambda}{0}{0}{2-\lambda}\rp, \\
      & = & (2-\lambda)^2.
    \end{eqnarray*}
    The eigen value is $\lambda=2$.
  }

\end{frame}

\begin{frame}
  \frametitle{Find the Eigen Vector}

  $\lambda_1=2$:
  \begin{eqnarray*}
    \arrayTwo{0}{0}{0}{0} \vecTwo{x}{y} & = & \vecTwo{0}{0}, \\
    \uncover<2->
    {
      \Rightarrow \vec{v} & = & \vecTwo{x}{y}, \\
      & = & x \vecTwo{1}{0} + y \vecTwo{0}{1}, \\
      \uncover<3->
      {
        \vec{v}_1 & = & \vecTwo{1}{0}, \\
        \vec{v}_2 & = & \vecTwo{0}{1}.
      }
    }
  \end{eqnarray*}
\end{frame}


\begin{frame}
  \frametitle{Form the Solution}
    
  The solution is 
  \begin{eqnarray*}
    \vec{x} & = & c_1 e^{2t} \vecTwo{1}{0} + c_2 e^{2t} \vecTwo{0}{1}.
  \end{eqnarray*}

\end{frame}


\subsection{Coupled Tank Problems}


\begin{frame}
  \frametitle{Coupled Tank}

  Two tanks are arranged as shown. Initially tank 1 has 100 liters of
  water and tank 2 has 50 liters of water. Initially each tank has 2
  kg of salt dissolved in the solution. Determine the amount of salt
  in each tank at any time.
  
\end{frame}


\begin{frame}
  \frametitle{System of Equations}

  $x(t)$ is the amount of salt in tank 1 at time $t$.

  $y(t)$ is the amount of salt in tank 2 at time $t$.

  \begin{eqnarray*}
    \deriv{x}{t} & = & 0\times 2 - 
    \frac{x}{100}\times 3 + \frac{y}{50}\times 2 \\
    \deriv{y}{t} & = & \frac{x}{100} \times 3 -
    \frac{y}{50} \times 1 - \frac{y}{50}\times 2.
  \end{eqnarray*}

  Written as a system we get
  \begin{eqnarray*}
    \deriv{~}{t} \vecTwo{x}{y} & = & 
    \arrayTwo{-3/100}{1/50}{3/100}{-3/50} \vecTwo{x}{y}.
  \end{eqnarray*}
  
\end{frame}


% LocalWords:  Clarkson pausesection hideothersubsections
