\part{Complex-Eigen-Values}
\lecture{Complex Eigen Values}{Complex-Eigen-Values}
\section{Complex Eigen Values}


\title{Ordinary Differential Equations}
\subtitle{Math 232 - Week 11, Day 3}
\date{11 November 2012}

\begin{frame}
  \titlepage
\end{frame}

\begin{frame}
  \frametitle{Outline}
  \tableofcontents[pausesection,hideothersubsections]
\end{frame}


\subsection{Complex Valued Eigen Values}


\begin{frame}
  \frametitle{Complex Valued Eigen Values}

  First, recall:
  \begin{eqnarray*}
    e^{a+bi} & = & e^a e^{bi}, \\
    & = & e^a \lp \cos(b) + i \sin(b) \rp.
  \end{eqnarray*}

  Also, the complex conjugate:
  \begin{eqnarray*}
    \overline{a+ib} & = & a-i b.
  \end{eqnarray*}

  Note: If A is a real valued matrix with complex valued eigen values,
  then the eigen values and eigen vectors come in complex conjugates
  pairs.

\end{frame}


\begin{frame}
  \frametitle{Example}

  \begin{eqnarray*}
    \deriv{~}{t} \vec{x} & = & \arrayTwo{5}{-10}{1}{-1} \vec{x}.
  \end{eqnarray*}

  \uncover<2->
  {
    Find the eigen values:
    \begin{eqnarray*}
      \det\lp\arrayTwo{5-\lambda}{-10}{1}{-1-\lambda}\rp & = & \lambda^2-4 \\
      \lambda_{1,2} & = & 2 \pm i.
    \end{eqnarray*}
  }

\end{frame}


\begin{frame}
  \frametitle{Find the Eigen Vectors}

  $\lambda_1 = 2+i$:
  \begin{eqnarray*}
    \arrayTwo{3-i}{-10}{1}{-3-i} \vecTwo{x}{y} & = & \vecTwo{0}{0} \\
    y & = & \lp \frac{3}{10} - \frac{i}{10} \rp x, \\
    \Rightarrow \vec{v}_1 & = & x \lp \vecTwo{1}{\frac{3}{10}} + \vecTwo{0}{\frac{-i}{10}} \rp.
  \end{eqnarray*}

  \uncover<2->
  {
    $\lambda_2 = 2-i$:
    \begin{eqnarray*}
      \arrayTwo{3+i}{-10}{1}{-3+i} \vecTwo{x}{y} & = & \vecTwo{0}{0} \\
      y & = & \lp \frac{3}{10} + \frac{i}{10} \rp x, \\
      \Rightarrow \vec{v}_2 & = & x \lp \vecTwo{1}{\frac{3}{10}} + \vecTwo{0}{\frac{i}{10}} \rp.
    \end{eqnarray*}
  }

\end{frame}


\begin{frame}
  \frametitle{Determine the Solution}

  \begin{eqnarray*}
    \uncover<1->
    {
      \vec{x} & = & A e^{(2+i)t} \lp \vecTwo{1}{\frac{3}{10}} + \vecTwo{0}{\frac{-i}{10}} \rp
      + B e^{(2-i)t} \lp \vecTwo{1}{\frac{3}{10}} - \vecTwo{0}{\frac{-i}{10}} \rp \\
    }
    \uncover<2->
    {
      \vec{x} & = & A e^{(2+i)t} \lp \underbrace{\vecTwo{1}{\frac{3}{10}}}_{\vec{p}} + 
      \underbrace{\vecTwo{0}{\frac{-i}{10}}}_{i\vec{q}} \rp
      + B e^{(2-i)t} \lp \underbrace{\vecTwo{1}{\frac{3}{10}}}_{\vec{p}} -
      \underbrace{\vecTwo{0}{\frac{-i}{10}}}_{i\vec{q}} \rp \\
    }
    \uncover<3->
    {
      & = & e^{2t} \lp
      A\lp \cos(t) + i \sin(t) \rp \lp \vec{p} + i \vec{q} \rp 
      + B \lp \cos(t) - i \sin(t) \rp \lp \vec{p} - i \vec{q} \rp 
      \rp
    }
  \end{eqnarray*}

\end{frame}


\begin{frame}
  \frametitle{More Algebra...}

  \begin{eqnarray*}
    \vec{x} & = & e^{2t} \lp
      A\lp \cos(t) + i \sin(t) \rp \lp \vec{p} + i \vec{q} \rp 
      + B \lp \cos(t) - i \sin(t) \rp \lp \vec{p} - i \vec{q} \rp 
      \rp \\
      & = & C_1 e^{2t} \lp \cos(t) \vec{p} - \sin(t) \vec{q} \rp
      + C_2 e^{2t} \lp \cos(t) \vec{q} + \sin(t) \vec{p} \rp.
  \end{eqnarray*}

  (The solutions spiral out.)

\end{frame}

\subsection{Characterization of Solutions}

\begin{frame}
  \frametitle{Characterization of Solutions}

  \begin{eqnarray*}
    \deriv{~}{t} \vec{x} & = & A \vec{x}.
  \end{eqnarray*}

  If $\lambda$'s are complex, $\lambda = a \pm ib$:
  \begin{enumerate}
  \item If $\alpha>0$ then it is a repelling spiral.
  \item If $\alpha<0$ then it is an attracting spiral.
  \item If $\alpha=0$ then the solutions are neutrally stable.
  \end{enumerate}

\end{frame}


\subsection{Nullclines}

\begin{frame}
  \frametitle{Nullclines}

  Given a solution to a differential equation,
  \begin{eqnarray*}
    \vec{x} & = & \vecTwo{x(t)}{y(t)},
  \end{eqnarray*}
  if
  \begin{itemize}
  \item $x'(t)=0$ then the trajectory is ``vertical'' at time $t$,
  \item $y'(t)=0$ then the trajectory is ``horizontal'' at time $t$.
  \end{itemize}

\end{frame}


\begin{frame}
  \frametitle{Definition of the Nullclines}

  The set of points, $\vecTwo{x}{y}$ where $x'(t)=0$ is called the v-nullcline. (Vertical slope.)

  The set of points, $\vecTwo{x}{y}$ where $y'(t)=0$ is called the h-nullcline. (Horizontal slope.)

\end{frame}


\begin{frame}
  \frametitle{Example}
  
  In the previous example we had
  \begin{eqnarray*}
    \deriv{~}{t} \vec{x} & = & \arrayTwo{5}{-10}{1}{-1} \vec{x}.
  \end{eqnarray*}

  \uncover<2->
  {
    Find the v-nullcline:
    \begin{eqnarray*}
      \deriv{x}{t} & = & 5x - 10y, \\
      0 & = & 5x - 10y, \\
      y & = & \frac{1}{2} x.
    \end{eqnarray*}
  }

  \uncover<3->
  {
    Find the h-nullcline:
    \begin{eqnarray*}
      \deriv{y}{t} & = & x - y, \\
      0 & = & x - y, \\
      y & = &  x.
    \end{eqnarray*}
  }


\end{frame}



% LocalWords:  Clarkson pausesection hideothersubsections
