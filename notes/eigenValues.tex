\part{Eigen-Vectors/Values}
\lecture{EigenVectors/Values}{Eigen-Vectors/Values}
\section{Eigenvalues and Eigenvectors}


\title{Ordinary Differential Equations}
\subtitle{Math 232 - Eigenvalues and eigenvectors}
\date{30 October 2013}

\begin{frame}
  \titlepage
\end{frame}

\begin{frame}
  \frametitle{Outline}
  \tableofcontents[currentsection]
\end{frame}


\subsection{Eigenvalues and Eigenvectors}


\begin{frame}
  \frametitle{Eigenvalues and Eigenvectors}

  Suppose that we have a system of differential equations,
  \begin{eqnarray*}
    \frac{d}{dt} \vec{x} & = & A \vec{x}.
  \end{eqnarray*}
  Assume that $A$ is a square matrix.

  What if we can find a vector, $\vec{v}$, that satisfies
  \begin{eqnarray*}
    A \vec{v} & = & \lambda \vec{v},
  \end{eqnarray*}
  where $\lambda$ is a scalar (number)?

\end{frame}


\begin{frame}
  If so. then....
  \begin{eqnarray*}
    \frac{d}{dt} \vec{v} & = & A \vec{v}, \\
    & = & \lambda \vec{v}.
  \end{eqnarray*}

  \only<1>%
  {
    But,
    \begin{eqnarray*}
      \vec{v} & = & 
      \left[
        \begin{array}{r}
          x_1 \\ x_2 \\ x_3 \\ \vdots \\ x_n
        \end{array}
      \right].
    \end{eqnarray*}
  }
  
  \only<2>%
  {
    This implies that
    \begin{eqnarray*}
      \deriv{x_1}{t} & = & \lambda x_1, \\
      \deriv{x_2}{t} & = & \lambda x_2, \\
      \deriv{x_3}{t} & = & \lambda x_3, \\
      \vdots         &   & \vdots \\
      \deriv{x_n}{t} & = & \lambda x_n
    \end{eqnarray*}

    We can solve each one of these equations separately!
  }

\end{frame}


\begin{frame}
  \frametitle{Eigenvalues and Eigenvectors}

  \begin{definition}
    Given a square matrix, $A$, an \blueText{eigenvector}, $\vec{v}$,
    and an associated \blueText{eigenvalue}, $\lambda$, satisfy
    \begin{eqnarray*}
      A \vec{v} & = & \lambda \vec{v}.
    \end{eqnarray*}
  \end{definition}



\end{frame}


\iftoggle{clicker}{%
\begin{frame}
  \frametitle{Clicker Quiz}

      \ifnum\value{clickerQuiz}=1{%
        We want to solve a system of linear equations
        \begin{eqnarray*}
          A \vec{x} & = & \vec{b}.
        \end{eqnarray*}
        We find the determinant of the matrix $A$, and the determinant is zero. 

        What does this imply about the solutions to the linear system.
        \vfill

        \begin{tabular}{ll}
          A: & The solution is unique. \\
          B: & There are infinite solutions. \\
          C: & There are no solutions. \\
          D: & Either B or C. \\
        \end{tabular}
        \vfill

      }\fi

      \ifnum\value{clickerQuiz}=2{%
        We want to solve a system of linear equations
        \begin{eqnarray*}
          A \vec{x} & = & \vec{b}.
        \end{eqnarray*}
        We find the determinant of the matrix $A$, and the determinant is zero. 

        What does this imply about the solutions to the linear system.
        \vfill

        \begin{tabular}{ll}
          A: & The solution is unique. \\
          B: & There are infinite solutions. \\
          C: & There are no solutions. \\
          D: & Either B or C. \\
        \end{tabular}

        \vfill

     }\fi
   
     \ifnum\value{clickerQuiz}=3{%
        \vfill
        We want to solve a system of linear equations
        \begin{eqnarray*}
          A \vec{x} & = & \vec{b}.
        \end{eqnarray*}
        We find the determinant of the matrix $A$, and the determinant is zero.

        What does this imply about the solutions to the linear system.
        \vfill

        \begin{tabular}{ll}
          A: & The solution is unique. \\
          B: & There are infinitely many solutions. \\
          C: & There are no solutions. \\
          D: & Either B or C. \\
        \end{tabular}

    }\fi
  

\end{frame}
}



\begin{frame}
  \frametitle{How to find}

  Given $A$ how can we find its eigenvalues and eigenvectors?
  \begin{eqnarray*}
    A \vec{v} & = & \lambda \vec{v}, \\
    A \vec{v} - \lambda \vec{v} & = & 0, \\
    A \vec{v} - \lambda \redText{I} \vec{v} & = & 0, \\
    \lp A - \lambda I \rp \vec{v} & = & 0.
  \end{eqnarray*}

  \only<1>{\textit{Remember $\vec{v}$ is a vector and $A$ is an array. We cannot
    add them!}}

  \uncover<2->{\redText{One solution is $\vec{v}=\vec{0}$.}} 
  \uncover<3->{\blueText{We do not like this solution.}}
  \uncover<4->{We must have a situation where there is not a unique solution. If there
    is not a unique solution then this implies that
  \begin{eqnarray*}
    \det\lp A - \lambda I \rp & = & 0.
  \end{eqnarray*}
  Find a value for $\lambda$ that makes this true. (\redText{Note that there
  will be infinitely many solutions for $\vec{v}$.})}

\end{frame}


\begin{frame}
  \frametitle{Finding Eigenvalues and Eigenvectors}

  Given a square matrix, $A$, to find the eigenvalues and eigenvectors do the following:
  \begin{enumerate}
  \item Determine the characteristic equation:
    \begin{eqnarray*}
      \det\lp A - \lambda I \rp & = & 0.
    \end{eqnarray*}
  \item Solve the equation for the \blueText{values} of $\lambda$,
    call them $\lambda_j$.
  \item Find the eigenvector associated with each value of $\lambda_j$,
    \begin{eqnarray*}
      \lp A - \lambda_j I \rp \vec{v}_j & = & \vec{0}.
    \end{eqnarray*}
    (Note that there will be infinitely many solutions for each
    $\vec{v}_j$.) Solving linear systems means dealing with
    RREF. Woohoo!!
  \end{enumerate}

\end{frame}

\subsection{Examples}

\begin{frame}
  \frametitle{Example}

  \begin{eqnarray*}
    A & = & \arrayTwo{12}{-10}{15}{-13}.
  \end{eqnarray*}

  \uncover<2->
  {
    Find the characteristic equation:
    \begin{eqnarray*}
      \det\lp\arrayTwo{12-\lambda}{-10}{15}{-13-\lambda}\rp & = & \lambda^2+\lambda-6.
    \end{eqnarray*}

    Find the values of $\lambda$:
    \begin{eqnarray*}
      \lambda^2+\lambda-6 & = & (\lambda-2)(\lambda+3), \\
      & = & 0.
    \end{eqnarray*}
    So \redText{$\lambda_1 = 2$} and \redText{$\lambda_2=-3$}.
  }

\end{frame}


\begin{frame}
  Find the vectors:

  $\lambda_1 = 2$:
  \begin{eqnarray*}
    \arrayTwo{10}{-10}{15}{-15} \vecTwo{x}{y} & = & \vecTwo{0}{0}, \\
    10x - 10y & = & 0, \\
    15x - 15y & = & 0, \\
    x & = & y.
  \end{eqnarray*}

  The first eigenvector is 
  \begin{eqnarray*}
    \vec{v}_1 & = & \vecTwo{y}{y}, \\
    & = & y \vecTwo{1}{1}.
  \end{eqnarray*}

  \redText{Let}
  \begin{eqnarray*}
    \redText{\vec{v}_1} & \redText{=} & \redText{\vecTwo{1}{1}}.
  \end{eqnarray*}

\end{frame}

\begin{frame}
  Find the vectors:

  $\lambda_2 = -3$:
  \begin{eqnarray*}
    \arrayTwo{15}{-10}{15}{-10} \vecTwo{x}{y} & = & \vecTwo{0}{0}, \\
    15x - 10y & = & 0, \\
    15x - 10y & = & 0, \\
    x & = & \frac{2}{3} y.
  \end{eqnarray*}

  The second eigenvector is 
  \begin{eqnarray*}
    \vec{v}_2 & = & \vecTwo{\frac{2}{3}y}{y}, \\
    & = & y \vecTwo{\frac{2}{3}}{1}.
  \end{eqnarray*}

  \redText{Let}
  \begin{eqnarray*}
    \redText{\vec{v}_2} & \redText{=} & \redText{\vecTwo{2}{3}}.
  \end{eqnarray*}

\end{frame}






\begin{frame}
  \frametitle{Example}

  \begin{eqnarray*}
    A & = & \arrayTwo{4}{18}{-3}{-11}.
  \end{eqnarray*}

  \uncover<2->
  {
    Find the characteristic equation:
    \begin{eqnarray*}
      \det\lp\arrayTwo{4-\lambda}{18}{-3}{-11-\lambda}\rp & = & \lambda^2+7\lambda+10.
    \end{eqnarray*}
    
    Find the values of $\lambda$:
    \begin{eqnarray*}
      \lambda^2+7\lambda+10 & = & (\lambda+2)(\lambda+5), \\
      & = & 0.
    \end{eqnarray*}
    So \redText{$\lambda_1 = -2$} and \redText{$\lambda_2=-5$}.
  }

\end{frame}


\begin{frame}
  Find the vectors:

  $\lambda_1 = -2$:
  \begin{eqnarray*}
    \arrayTwo{6}{18}{-3}{-9} \vecTwo{x}{y} & = & \vecTwo{0}{0}, \\
    6x + 18y & = & 0, \\
    -3x - 9y & = & 0, \\
    x & = & -3y.
  \end{eqnarray*}

  The first eigenvector is 
  \begin{eqnarray*}
    \vec{v}_1 & = & \vecTwo{-3y}{y}, \\
    & = & y \vecTwo{-3}{1}.
  \end{eqnarray*}

  \redText{Let}
  \begin{eqnarray*}
    \redText{\vec{v}_1} & \redText{=} & \redText{\vecTwo{-3}{1}}.
  \end{eqnarray*}

\end{frame}

\begin{frame}
  Find the vectors:

  $\lambda_2 = -5$:
  \begin{eqnarray*}
    \arrayTwo{9}{18}{-3}{-6} \vecTwo{x}{y} & = & \vecTwo{0}{0}, \\
    9x + 18y & = & 0, \\
    -3x - 6y & = & 0, \\
    x & = & -2 y.
  \end{eqnarray*}

  The second eigenvector is 
  \begin{eqnarray*}
    \vec{v}_2 & = & \vecTwo{-2y}{y}, \\
    & = & y \vecTwo{-2}{1}.
  \end{eqnarray*}

  \redText{Let}
  \begin{eqnarray*}
    \redText{\vec{v}_2} & \redText{=} & \redText{\vecTwo{-2}{1}}.
  \end{eqnarray*}

\end{frame}



\begin{frame}
  \frametitle{Theorem - Distinct Eigenvalues Makes Life Easier}

  \begin{theorem}
    If the eigenvalues of an $n\times n$ matrix include \blueText{\textbf{$n$}}
    \blueText{distinct} values then the eigenvectors are linearly
    independent.
  \end{theorem}


  \vfill

  \uncover<2->
  {
    So what? When we start solving DEs it will mean we are essentially
    done. 
  }

  \vfill

\end{frame}



\begin{frame}
  \frametitle{Example}

  See example 5 on page 318.
  \begin{eqnarray*}
    A & = & \arrayThree{-2}{1}{1}{1}{-2}{1}{1}{1}{-2}.
  \end{eqnarray*}

  \uncover<2->
  {
    Find the characteristic equation:
    \begin{eqnarray*}
      \det\lp\arrayThree{-2-\lambda}{1}{1}{1}{-2-\lambda}{1}{1}{1}{-2-\lambda}\rp
      & = & \lambda (\lambda+3)^2.
    \end{eqnarray*}

    Find the values of $\lambda$:
    \begin{eqnarray*}
      \lambda (\lambda+3)^2 & = & 0.
    \end{eqnarray*}
    So \redText{$\lambda_1 = 0$} and \redText{$\lambda_2=-3$}.
  }

\end{frame}



\begin{frame}
  Find the vectors:

  $\lambda_1 = 0$:
  \begin{eqnarray*}
    \arrayThree{-2}{1}{1}{1}{-2}{1}{1}{1}{-2}\vecThree{x}{y}{z} & = & \vecThree{0}{0}{0}, \\
  \end{eqnarray*}

  Express this system in an augmented matrix and put it in reduced row
  echelon form:
  \begin{eqnarray*}
    \mathrm{RREF}\startRowFour
    \oneRowFour{-2}{1}{1}{0} 
    \oneRowFour{1}{-2}{1}{0}
    \oneRowFour{1}{1}{-2}{0}
    \stopRowOps
    & = & 
    \startRowFour
    \oneRowFour{1}{0}{-1}{0} 
    \oneRowFour{0}{1}{-1}{0}
    \oneRowFour{0}{0}{0}{0}
    \stopRowOps
  \end{eqnarray*}


\end{frame}

\begin{frame}
  The first eigenvector is 
  \begin{eqnarray*}
    \vec{v}_1 & = & \vecThree{z}{z}{z}, \\
    & = & z \vecThree{1}{1}{1}.
  \end{eqnarray*}

  Let
  \begin{eqnarray*}
    \vec{v}_1 & = & \vecThree{1}{1}{1}.
  \end{eqnarray*}

\end{frame}

\begin{frame}
  Find the vectors:

  $\lambda_2 = -3$:
  \begin{eqnarray*}
    \arrayThree{1}{1}{1}{1}{1}{1}{1}{1}{1}
    \vecThree{x}{y}{z} & = & \vecThree{0}{0}{0}, \\
  \end{eqnarray*}

  Express this system in an augmented matrix and put it in reduced row
  echelon form:
  \begin{eqnarray*}
    \mathrm{RREF}\startRowFour
    \oneRowFour{1}{1}{1}{0} 
    \oneRowFour{1}{1}{1}{0}
    \oneRowFour{1}{1}{1}{0}
    \stopRowOps
    & = & 
    \startRowFour
    \oneRowFour{1}{1}{1}{0} 
    \oneRowFour{0}{0}{0}{0}
    \oneRowFour{0}{0}{0}{0}
    \stopRowOps
  \end{eqnarray*}

\end{frame}

\begin{frame}

  The second eigenvector is 
  \begin{eqnarray*}
    \vec{v}_2 & = & \vecThree{-y-z}{y}{z}, \\
    & = & y \vecThree{-1}{1}{0} + z \vecThree{-1}{0}{1}.
  \end{eqnarray*}

  Let
  \begin{eqnarray*}
    \vec{v}_2 & = & \vecThree{-1}{1}{0}, \\
    \vec{v}_3 & = & \vecThree{-1}{0}{1}.
  \end{eqnarray*}

\end{frame}

\begin{frame}
  \frametitle{Example}

  \begin{eqnarray*}
    A & = & \arrayThree{2}{1}{1}{0}{1}{1}{0}{0}{1}.
  \end{eqnarray*}

  \uncover<2->
  {
    Find the characteristic equation:
    \begin{eqnarray*}
      \det\lp\arrayThree{2-\lambda}{1}{1}{0}{1-\lambda}{1}{0}{0}{1-\lambda}\rp
      & = & (2-\lambda) (1-\lambda)^2.
    \end{eqnarray*}

    Find the values of $\lambda$:
    \begin{eqnarray*}
      (2-\lambda) (1-\lambda)^2 & = & 0.
    \end{eqnarray*}
    So \redText{$\lambda_1 = 2$} and \redText{$\lambda_2=1$}.
  }

\end{frame}


\begin{frame}
  Find the vectors:

  $\lambda_1 = 2$:
  \begin{eqnarray*}
    \arrayThree{0}{1}{1}{0}{-1}{1}{0}{0}{-1}\vecThree{x}{y}{z} & = & \vecThree{0}{0}{0}, \\
  \end{eqnarray*}

  Express this system in an augmented matrix and put it in reduced row
  echelon form:
  \begin{eqnarray*}
    \mathrm{RREF}\startRowFour
    \oneRowFour{0}{1}{1}{0} 
    \oneRowFour{0}{-1}{1}{0}
    \oneRowFour{0}{0}{-1}{0}
    \stopRowOps
    & = & 
    \startRowFour
    \oneRowFour{0}{1}{0}{0} 
    \oneRowFour{0}{0}{1}{0}
    \oneRowFour{0}{0}{0}{0}
    \stopRowOps
  \end{eqnarray*}


\end{frame}

\begin{frame}
  The first eigenvector is 
  \begin{eqnarray*}
    \vec{v}_1 & = & \vecThree{x}{0}{0}, \\
    & = & x \vecThree{1}{0}{0}.
  \end{eqnarray*}

  Let
  \begin{eqnarray*}
    \vec{v}_1 & = & \vecThree{1}{0}{0}.
  \end{eqnarray*}

\end{frame}

\begin{frame}
  Find the vectors:

  $\lambda_2 = 1$:
  \begin{eqnarray*}
    \arrayThree{1}{1}{1}{0}{0}{1}{0}{0}{0}
    \vecThree{x}{y}{z} & = & \vecThree{0}{0}{0}, \\
  \end{eqnarray*}

  Express this system in an augmented matrix and put it in reduced row
  echelon form:
  \begin{eqnarray*}
    \mathrm{RREF}\startRowFour
    \oneRowFour{1}{1}{1}{0} 
    \oneRowFour{0}{0}{1}{0}
    \oneRowFour{0}{0}{0}{0}
    \stopRowOps
    & = & 
    \startRowFour
    \oneRowFour{1}{1}{0}{0} 
    \oneRowFour{0}{0}{1}{0}
    \oneRowFour{0}{0}{0}{0}
    \stopRowOps
  \end{eqnarray*}

\end{frame}

\begin{frame}

  The second eigenvector is 
  \begin{eqnarray*}
    \vec{v}_2 & = & \vecThree{-y}{y}{0}, \\
    & = & y \vecThree{-1}{1}{0}.
  \end{eqnarray*}

  Let
  \begin{eqnarray*}
    \vec{v}_2 & = & \vecThree{-1}{1}{0}.
  \end{eqnarray*}

  There are only two eigenvectors! If we want to solve a system of
  differential equations we have a problem.

\end{frame}

\iftoggle{clicker}{%
\begin{frame}
  \frametitle{Clicker Quiz}
      \ifnum\value{clickerQuiz}=1{%
        We have two distinct eigenvalues for a $3 \times 3$
        matrix. What does this imply about the number of linearly
        independent eigenvectors?

        \vfill
  
        \begin{tabular}{ll}
          A: & There are  two. \\
          B: & There are three. \\
          C: & There could be two or three, but there is not enough information. \\
          D: & Wut? \\
        \end{tabular}

        \vfill
  
      }\fi

      \ifnum\value{clickerQuiz}=2{%
        We have two distinct eigenvalues for a $3 \times 3$
        matrix. What does this imply about the number of linearly
        independent eigenvectors?
  
        \vfill
  
        \begin{tabular}{ll}
          A: & There are  two. \\
          B: & There are three. \\
          C: & There could be two or three, but there is not enough information. \\
          D: & Wut? \\
        \end{tabular}

        \vfill

     }\fi
     \ifnum\value{clickerQuiz}=3{%
        \vfill
       We have two distinct eigenvalues for a $3 \times 3$
        matrix. What does this imply about the number of linearly
        independent eigenvectors?

        \vfill

        \begin{tabular}{ll}
          A: & There are  two. \\
          B: & There are three. \\
          C: & There could be two or three,
             but there is not enough information. \\
          D: & Have no idea.\\
       \end{tabular}

    }\fi
\end{frame}
}




% LocalWords:  Clarkson pausesection hideothersubsections Eigen eigen
%  LocalWords:  Eigenvectors eigenvector eigenvectors RREF
