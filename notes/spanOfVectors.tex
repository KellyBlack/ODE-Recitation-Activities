\part{Span-of-Vectors}
\lecture{Span of Vectors}{Span-of-Vectors}
\section{Span of Vectors}

\title{Ordinary Differential Equations}
\subtitle{Math 232 - Span of Vectors}
\date{5 Oct 2012}

\begin{frame}
  \titlepage
\end{frame}

\begin{frame}
  \frametitle{Outline}
  \tableofcontents[pausesection,hideothersubsections]
\end{frame}


\subsection{Vector Spaces}

\begin{frame}
  \frametitle{Vector Spaces}

  Definition of a vector space, V:
  \begin{itemize}
  \item Elements in V are called ``vectors.''
  \item If $\vec{x}$, $\vec{y}$ are members of V then
    $\vec{x}+\vec{y}$ is in V.
  \item If $c$ is a real number and $\vec{x}$ is in V then so is
    $c\vec{x}$.
  \item There is a ``zero vector'' where $\vec{x}+\vec{0}=\vec{x}$ for
    every member of V.
  \item For any $\vec{x}$ in V there is another $\vec{y}$ in V where
    $\vec{x}+\vec{y}=\vec{0}$. ($\vec{y}$ is called ``$-\vec{x}$.'')
  \end{itemize}

  See p. 168 for definition and properties.

\end{frame}

\begin{frame}
  \frametitle{Example}

  $\mathbb{R}^2$ is a vector space.
  \only<1->{Suppose that $x$,  $y$, $u$, $v$, and $c$ are real numbers.}
  \begin{eqnarray*}
    \only<1->{\vecTwo{x}{y} & \in & \mathbb{R}^2} \\
    \only<1->{\vecTwo{x}{y} + \vecTwo{u}{v} & = & \vecTwo{x+u}{y+v}} \\
    \only<1->{c \vecTwo{x}{y} & = & \vecTwo{cx}{cy}} \\
    \only<1->{\vec{0} & = & \vecTwo{0}{0}} \\
    \only<1->{\vecTwo{x}{y} + \vecTwo{-x}{-y} & = & \vec{0}}
  \end{eqnarray*}

\end{frame}


\begin{frame}
  \frametitle{Example - Quadratic Functions}

  The set of quadratic functions is a vector space:

  \uncover<1->{%
    \begin{eqnarray*}
      h(x) & = & a x^2 + bx + c, \\
      g(x) & = & dx^2 + ex + f,
    \end{eqnarray*}
    where $a$, $b$, $c$, $d$, $e$, and $f$ are real numbers.
  }

  \only<1>{%
    \color{red}{Addition:}{~}
    \begin{eqnarray*}
      h(x)+g(x) & = & (a+d) x^2 + (b+e) x + (c+f)
    \end{eqnarray*}
    is a quadratic function.
  }

  \only<1>{%
    \color{red}{Scalar multiplication}, if $r$ is a real number:
    \begin{eqnarray*}
      r h(x) & = & ra x^2 + rb x + rc
    \end{eqnarray*}
    is a quadratic.
  }

  \only<1>{%
    The function $h(x) =  0$ 
    is a quadratic and is the \color{red}{``zero vector.''}
  }


\end{frame}



\subsection{Span of a Set of Vectors}


\begin{frame}
  \frametitle{Span of a Set of Vectors}

  Any vector in $\mathbb{R}^2$ can be written as 
  \begin{eqnarray*}
    \vec{x} & = & \vecTwo{x}{y},  \\
    \uncover<2->
    {
      & = & \vecTwo{{\color{red}x}}{0} + \vecTwo{0}{{\color{blue}y}},
    } \\
    \uncover<2->
    {
      & = & {\color{red}x} \vecTwo{1}{0} + {\color{blue}y} \vecTwo{0}{1},
    }\\
    \uncover<2->
    {
     & = & {\color{red}x} \vec{i} + {\color{blue}y} \vec{j} ,
     }
  \end{eqnarray*}

  where $x$ and $y$ can be \textbf{any} constants,  
  $\vec{i} =  \vecTwo{1}{0}, \vec{j} =  \vecTwo{0}{1}$.

\end{frame}


\begin{frame}
  \frametitle{Span of a Set of Vectors}

  {\color{brown}The set} of all possible vectors 
  that can be written in the form
  \begin{eqnarray*}
    \vec{v} & = & x \vecTwo{1}{0} + y \vecTwo{0}{1}
  \end{eqnarray*}
  for any real numbers $x$ and $y$ is called  
  {the \color{red}\textit{\underline{span}} of the
  vectors $\vecTwo{1}{0}$ and $\vecTwo{0}{1}$}. 

  \vfill

  \uncover<1->
  {
    Any vector that I can write in the form
    \begin{eqnarray*}
      x \vecTwo{1}{0} + y \vecTwo{0}{1}
    \end{eqnarray*}
    is in this \textbf{\underline{set}} of vectors.
  }

\end{frame}


\begin{frame}
  \frametitle{Span of a Set of Vectors}

  In general, given a \textbf{\underline{set}} of vectors, $\vec{v_1}$, $\vec{v_2}$, $\vec{v_3}$,
  $\ldots$, $\vec{v_n}$ then the \textbf{\underline{set}} of vectors
  that can be written in the form
  \begin{eqnarray*}
    c_1 \vec{v_1} + c_2 \vec{v_2} + c_3 \vec{v_3} + \cdots + c_n \vec{v_n}
  \end{eqnarray*}
  is called {\color{red}the \textit{\underline{span}} of the vectors.}

  \uncover<1->
  {
    Notation:
    {\color{blue}
     \begin{eqnarray*}
      \mathrm{span}\{\vec{v_1}, \vec{v_2}, \vec{v_3},\ldots, \vec{v_n} \}
    = \{c_1 \vec{v_1} + c_2 \vec{v_2} + c_3 \vec{v_3} + \cdots + c_n \vec{v_n}\}
    \end{eqnarray*}
    }
  }

\end{frame}


\iftoggle{clicker}{%
\begin{frame}
  \frametitle{Clicker Quiz}

      \ifnum\value{clickerQuiz}=1{%

        \vfill

        Is $\mathrm{span}\{\vecTwo{2}{1},\vecTwo{4}{2}\}$ equal to $\mathbb{R}^2$?

        \vfill

        \begin{tabular}{ll}
          A: & Yes \\
          B: & No
        \end{tabular}


        \vfill

      }\fi

      \ifnum\value{clickerQuiz}=2{%

        \vfill

        Is $\mathrm{span}\left\{\vecTwo{1}{3},\vecTwo{2}{6}\right\}$ equal to $\mathbb{R}^2$?

        \vfill

        \begin{tabular}{ll}
          A: & Yes \\
          B: & No
        \end{tabular}

        \vfill

     }\fi
   
     \ifnum\value{clickerQuiz}=3{%
        
        Is $\mathrm{span}\left\{\vecTwo{3}{2},\vecTwo{6}{4}\right\}$ equal to $\mathbb{R}^2$?

        \vfill

        \begin{tabular}{ll}
          A: & Yes \\
          B: & No
        \end{tabular}

  
        \vfill

    }\fi
  

\end{frame}
}


\begin{frame}
  \frametitle{$\mathbb{R}^2$}

  Note: any vector in $\mathbb{R}^2$ can be written as 
  \begin{eqnarray*}
    \vec{v} & = & \vecTwo{x}{y}.
  \end{eqnarray*}

  We can break this out
  \begin{eqnarray*}
    \vec{v} & = & x \vecTwo{1}{0} + y \vecTwo{0}{1}.
  \end{eqnarray*}

  So... 
  \begin{eqnarray*}
    \mathbb{R}^2 & = & \mathrm{span}\left\{\vecTwo{1}{0},\vecTwo{0}{1}\right\}
  \end{eqnarray*}
 


\end{frame}


\subsection{Column Space of a Matrix}

\begin{frame}
  \frametitle{Column Space of a Matrix}

  Why should I care?

  \begin{eqnarray*}
    \arrayThree{1}{2}{4}{0}{1}{3}{2}{1}{4} \vecThree{x_1}{x_2}{x_3} 
    & = & 
    \only<2>{%
      \vecThree{x_1+2x_2+4x_3}{0x_1+1 x_2+3x_3}{2x_1+1 x_2+4 x_3} 
    }
    \uncover<3->{%
      \vecThree{{\color{red}x_1}+{\color{blue}2x_2}+{\color{green}4x_3}}{%
        {\color{red}0x_1}+{\color{blue}1 x_2}+{\color{green}3x_3}}{%
        {\color{red}2x_1}+{\color{blue}1 x_2}+{\color{green}4 x_3}} \\
    }
    \uncover<4->{%
      & = & x_1 {\color{red}\vecThree{1}{0}{2}} + 
            x_2 {\color{blue}\vecThree{2}{1}{1}} + 
            x_3 {\color{green}\vecThree{4}{3}{4}}
    }
  \end{eqnarray*}

  \uncover<4->{%
    The matrix vector multiplication is a linear combination of the
    columns of the matrix!
  }

\end{frame}


\begin{frame}
  \frametitle{Column Space of a Matrix}

  {\color{red}Definition:} 
  {\color{brown}The \textit{Column space} of a matrix is the span of its
  column vectors.}

  \uncover<1->
  {
    The column space of 
    \begin{eqnarray*}
      \arrayThree{1}{2}{4}{0}{1}{3}{2}{1}{4}
    \end{eqnarray*}
    is 
    \begin{eqnarray*}
      \mathrm{span}\left\{
        \vecThree{1}{0}{2},\vecThree{2}{1}{1},\vecThree{4}{3}{4}\right\}.
    \end{eqnarray*}
  }

\end{frame}

\subsection{Linear Systems}

\begin{frame}
  \frametitle{Solving Linear Systems}

  but... I still do not care.

  We want to solve
  \begin{eqnarray*}
    A \vec{x} & = & \vec{b}.
  \end{eqnarray*}

  \uncover<2->{%
    The matrix $A$ can be thought of as a bunch of column vectors
    \begin{eqnarray*}
      A & = & \left[ \vec{v}_1~\vec{v}_2~\vec{v}_3~\cdots~\vec{v}_n \right].
    \end{eqnarray*}

    We want to find constants $x_1, x_2, \cdots, x_n$ so that
    \begin{eqnarray*}
      x_1 \vec{v}_1 + x_2 \vec{v}_2 + x_3 \vec{v}_3 + \cdots + x_n \vec{v}_n & = & \vec{b}
    \end{eqnarray*}

  }


\end{frame}

\begin{frame}
  \frametitle{Think Different}

  We want to solve
  \begin{eqnarray*}
    A \vec{x} & = & \vec{b}.
  \end{eqnarray*}

    We want to find constants $x_1, x_2, \cdots, x_n$ so that
    \begin{eqnarray*}
      x_1 \vec{v}_1 + x_2 \vec{v}_2 + x_3 \vec{v}_3 + \cdots + x_n \vec{v}_n & = & \vec{b}
    \end{eqnarray*}
    Is the same thing as solving a system in the augmented matrix:
    \begin{eqnarray*}
      \left[ \vec{v}_1 ~ \vec{v}_2 ~ \vec{v}_3 ~ \cdots ~ \vec{v}_n \bigg| \vec{b} \right]
    \end{eqnarray*}

    Put this system into RREF and solve for the unknowns.


\end{frame}


\begin{frame}
  \frametitle{Another Way to Pose the Question}

  Given a vector, $\vec{b}$, can I find 
  \begin{eqnarray*}
    x_1,~x_2,~x_3,~\ldots~,x_n
  \end{eqnarray*}
  so that 
  \begin{eqnarray*}
    x_1 \vec{v}_1 + x_2 \vec{v}_2 + x_3 \vec{v}_3 + \cdots + x_n \vec{v}_n & = & \vec{b}?
  \end{eqnarray*}
  Is the same thing as finding constants $x_1, x_2, \cdots, x_n$, so that 
   $$\vec{b} \in  span
   \left\{\vec{v}_1 ~ \vec{v}_2 ~ \vec{v}_3 ~ \cdots ~ \vec{v}_n \right\}.
   $$

\end{frame}


%\begin{frame}
%  \frametitle{Relationship to Linear Systems}
%
%  Meh, I still do not care.
%%
%  But this is a big question!
%
%  Solving this:
%  \begin{eqnarray*}
%    x_1 \vec{v}_1 + x_2 \vec{v}_2 + x_3 \vec{v}_3 + \cdots + x_n \vec{v}_n & = & \vec{b}?
%  \end{eqnarray*}
%
%  Is the same thing as solving a system in the augmented matrix:
%  \begin{eqnarray*}
%    \left[ \vec{v}_1 ~ \vec{v}_2 ~ \vec{v}_3 ~ \cdots ~ \vec{v}_n \bigg| \vec{b} \right]
%  \end{eqnarray*}
%
%  Put this system into RREF and solve for the unknowns.
%  
%
%\end{frame}


\begin{frame}
  \frametitle{Example}

  Find the solution to the system of equations
  \begin{eqnarray*}
    \arrayTwo{1}{3}{2}{4} \vec{x} & = & \vec{b}.
  \end{eqnarray*}

  \uncover<1-> { Another way to state the problem: Given any vector,
    $\vec{b}$, can I find constants, $x_1$ and $x_2$, so that 
    \begin{eqnarray*}
      x_1 \vecTwo{1}{2} + x_2 \vecTwo{3}{4}  & = & \vec{b}?
    \end{eqnarray*}
  }

  
\end{frame}


\begin{frame}

  \begin{eqnarray*}
    \startRowOpsTwo
    \oneRowOpsTwo{1}{3}{b_1}{~}
    \oneRowOpsTwo{2}{4}{b_2}{~}
    \stopRowOps \\
    \uncover<1->
    {
      \stateTwo{~}{R_2-2R_1}
      \startRowOpsTwo
      \oneRowOpsTwo{1}{3}{b_1}{~}
      \oneRowOpsTwo{0}{-2}{b_2-2b_1}{~}
      \stopRowOps \\
    }
    \uncover<1->
    {
      \stateTwo{~}{-1/2 R_2}
      \startRowOpsTwo
      \oneRowOpsTwo{1}{3}{b_1}{~}
      \oneRowOpsTwo{0}{1}{\half \lp 2b_1-b_2 \rp}{~}
      \stopRowOps \\
    }
    \uncover<1->
    {
      \stateTwo{R_1-3R_2}{~}
      \startRowOpsTwo
      \oneRowOpsTwo{1}{0}{-2 b_1 + \frac{3}{2} b_2 }{~}
      \oneRowOpsTwo{0}{1}{\half \lp b_2-2b_1 \rp}{~}
      \stopRowOps \\
    }
  \end{eqnarray*}
  
  \uncover<1->{I can find the solution no matter what the value of $\vec{b}$ is!}
    
\end{frame}

\iftoggle{clicker}{%
\begin{frame}
  \frametitle{Clicker Quiz}

      \ifnum\value{clickerQuiz}=1{%

        \vfill

        Suppose that 
        \begin{eqnarray*}
          x_1 \vecTwo{1}{2} + x_2 \vecTwo{3}{4} & = & \vecTwo{0}{0}.
        \end{eqnarray*}
        What are the values of $x_1$ and $x_2$?

        \vfill

        \begin{tabular}{ll}
          A: & $x_1=0$, $x_2=0$ \\
          B: & The solution is not unique.
        \end{tabular}


        \vfill

      }\fi

      \ifnum\value{clickerQuiz}=2{%

        \vfill
        Suppose that 
        \begin{eqnarray*}
          x_1 \vecTwo{1}{2} + x_2 \vecTwo{3}{4} & = & \vecTwo{0}{0}.
        \end{eqnarray*}
        What are the values of $x_1$ and $x_2$?

        \vfill

        \begin{tabular}{ll}
          A: & $x_1=0$, $x_2=0$ \\
          B: & The solution is not unique.
        \end{tabular}

        \vfill

     }\fi
   
     \ifnum\value{clickerQuiz}=3{%

        \vfill
       Suppose that
        \begin{eqnarray*}
          x_1 \vecTwo{1}{2} + x_2 \vecTwo{3}{4} & = & \vecTwo{0}{0}.
        \end{eqnarray*}
        What are the values of $x_1$ and $x_2$?

        \vfill

        \begin{tabular}{ll}
          A: & $x_1=0$, $x_2=0$ \\
          B: & The solution is not unique.
        \end{tabular}


  %Find the solution to the system of equations
  %\begin{eqnarray*}
  %  \arrayTwo{1}{2}{2}{4} \vec{x} & = & \vec{b}.
  %\end{eqnarray*}

  %\uncover<1-> { Another way to state the problem: 
  %Given any vector,
  %  $\vec{b}$, can I find constants, $x_1$ and $x_2$, so that 
  %  \begin{eqnarray*}
  %    x_1 \vecTwo{1}{2} + x_2 \vecTwo{2}{4}  & = & \vec{b}?
  %  \end{eqnarray*}
  %}
  % 
  %  The system is 
  %
  %  \begin{tabular}{ll}
  %        A: & inconsistent \\
  %        B: & underdeterminant\\
  %        C: & I am not sure
  %      \end{tabular}

        \vfill


    }\fi
  

\end{frame}
}

%\begin{frame}
%  
%  \begin{eqnarray*}
%    \startRowOpsTwo
%    \oneRowOpsTwo{1}{2}{b_1}{~}
%    \oneRowOpsTwo{2}{4}{b_2}{~}
%    \stopRowOps \\
%    \uncover<1->
%    {
%      \stateTwo{~}{R_2-2R_1}
%      \startRowOpsTwo
%      \oneRowOpsTwo{1}{2}{b_1}{~}
%      \oneRowOpsTwo{0}{0}{b_2-2b_1}{~}
%      \stopRowOps \\
%    }
%  \end{eqnarray*}
%
%  \uncover<1->{I cannot find the solution except in certain circumstances!}
%
%\end{frame}

\begin{frame}
  \frametitle{Example}

  Given any vector, $\vec{b}$, can I find constants, $x_1$, $x_2$ and
  $x_3$, so that
  \begin{eqnarray*}
    x_1 \vecTwo{1}{2} + x_2 \vecTwo{1}{1}  + x_3 \vecTwo{4}{3} & = & \vec{b}?
  \end{eqnarray*}  

\end{frame}

\begin{frame}

  \begin{eqnarray*}
    \startRowOpsThree
    \oneRowOpsThree{1}{1}{4}{b_1}{}{}
    \oneRowOpsThree{2}{1}{3}{b_2}{}{}
    \stopRowOps \\
    \uncover<1->
    {
      \stateThree{~}{R_2-2R_1}{}
      \startRowOpsThree
      \oneRowOpsTwo{1}{1}{4}{\#}{}{}
      \oneRowOpsTwo{0}{-1}{-5}{\#}{}{}
      \stopRowOps \\
    }
    \uncover<1->
    {
      \stateThree{}{-R_2}{}
      \startRowOpsThree
      \oneRowOpsTwo{1}{1}{4}{\#}{}{}
      \oneRowOpsTwo{0}{1}{5}{\#}{}{}
      \stopRowOps \\
    }
    \uncover<1->
    {
      \stateThree{R_1-R_2}{}{}
      \startRowOpsThree
      \oneRowOpsTwo{1}{0}{-1}{\#}{}{}
      \oneRowOpsTwo{0}{1}{5}{\#}{}{}
      \stopRowOps \\
    }
  \end{eqnarray*}

  \uncover<1->{We have extra information! }
   
  
\end{frame}

\begin{frame}
  \frametitle{Linearly Dependent}


  Given any vector, $\vec{b}$, can I find constants, $x_1$, $x_2$ and
  $x_3$, so that
  \begin{eqnarray*}
    x_1 \vecTwo{1}{2} + x_2 \vecTwo{1}{1}  + x_3 \vecTwo{4}{3} & = & \vec{b}?
  \end{eqnarray*}  
  

  Note that
  \begin{eqnarray*}
    \vecTwo{4}{3} & = & -\vecTwo{1}{2} + 5 \vecTwo{1}{1}.
  \end{eqnarray*}

  So $\vecTwo{4}{3}$ \textit{\underline{depends}} on $\vecTwo{1}{2}$
  and $\vecTwo{1}{1}$. 

  {\color{red}We say that the vectors are ``linearly
  dependent.''}

\end{frame}

\subsection{The Wronksian}

\begin{frame}
  \frametitle{The Wronskian}

  Any quadratic can be written as 
  \begin{eqnarray*}
    f(t) & = & at^2 + bt + c.
  \end{eqnarray*}
  The span of $\{t^2,t,1\}$, is the set of all quadratic functions.

  What about $\{t^2-1,t^2+1,t^2-2\}$? Can I write any quadratic function with these?

\end{frame}

\begin{frame}
  \frametitle{The Wronskian}

  We define the Wronskian to be the following determinant:
  \begin{eqnarray*}
    W & = & 
    \mathrm{det}
    \arrayThree{t^2-1}{t^2+1}{t^2-2}{\frac{d}{dt}\lp t^2-1\rp}{\frac{d}{dt}\lp t^2+1\rp}{\frac{d}{dt}\lp t^2-2\rp}{\frac{d^2}{dt^2}\lp t^2-1\rp}{\frac{d^2}{dt^2}\lp t^2+1\rp}{\frac{d^2}{dt^2}\lp t^2-2\rp}  \\
    \uncover<1->
    {
    & = & 
    \mathrm{det}
    \arrayThree{t^2-1}{t^2+1}{t^2-2}{2t}{2t}{2t}{2}{2}{2}  \\
    }
    \uncover<1->
    {
      & = & 0
    }
  \end{eqnarray*}

  \uncover<1->
  {
    If the Wronskian is zero then the functions are linearly dependent.
  }

\end{frame}


\begin{frame}
  \frametitle{Definition of the Wronskian}

  The Wronskian is defined to be
  \begin{eqnarray*}
    W & = & \mathrm{det}
    \left[
      \begin{array}{rrcr}
        f_1 & f_2 & \cdot & f_n \\
        f_1' & f_2' & \cdot & f_n' \\
        f_1'' & f_2'' & \cdot & f_n'' \\
        \vdots & \vdots & & \vdots \\
        f_1^{(n-1)} & f_2^{(n-1)} & \cdots & f_n^{(n-1)}
      \end{array}
    \right]
  \end{eqnarray*}

\end{frame}

\iftoggle{clicker}{%
\begin{frame}
  \frametitle{Clicker Quiz}

       \ifnum\value{clickerQuiz}=1{%

        \vfill
       
        
       }\fi

       \ifnum\value{clickerQuiz}=2{%

       \vfill
        
         
       }\fi

      \ifnum\value{clickerQuiz}=3{%

        \vfill
  Are the functions $\{t^2,t,1\}$ linearly independent?

%  \begin{eqnarray*}
%    W & = & \mathrm{det}
%    \arrayThree{t^2}{t}{1}{2t}{1}{0}{2}{0}{0} \\
%    & = & -2 \\
%    & \neq & 0
%  \end{eqnarray*}
%  Yes. The functions $\{t^2,t,1\}$ are linearly independent.
       \vfill

        \begin{tabular}{ll}
          A: & Yes \\
          B: & No
        \end{tabular}

        \vfill

     }\fi

\end{frame}

}

% LocalWords:  Clarkson pausesection hideothersubsections Meh RREF Wronksian det
% LocalWords:  Wronskian
