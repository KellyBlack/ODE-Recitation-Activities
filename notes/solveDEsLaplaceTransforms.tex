\part{Solving-ODEs-With-Laplace-Transforms}
\lecture{Solving ODEs With Laplace Transforms}{Solving-DEs-With-Laplace-Transforms}
\section{Solving ODEs with Laplace Transforms}

\title{Ordinary Differential Equations}
\subtitle{Solving ODEs with Laplace Transforms}
\date{22 November 2013}

\begin{frame}
  \titlepage
\end{frame}

\begin{frame}
  \frametitle{Outline}
  \tableofcontents[ currentsection ]
\end{frame}


\subsection{Laplace Transforms of Derivatives}


\begin{frame}
  \frametitle{Laplace Transforms of Derivatives}

  Yet another identity:
  \begin{eqnarray*}
    \laplace{\redText{y'}} & = & \int^\infty_0 \redText{y'(t)} e^{-st} ~ dt, \\
    & = & \lim_{M\rightarrow\infty} \int^M_0 \redText{y'(t)} e^{-st} ~ dt, \\
    \end{eqnarray*}
    \only<2-3>{%
      We can integrate by parts:
      \begin{eqnarray*}
        u= e^{-st}, & & v=\only<3>{y(t)}, \\
        du= \only<3>{-se^{-st}\;dt}, & & dv=y'(t)dt, \\
      \end{eqnarray*}
    }

    \only<4->{%
      \begin{eqnarray*}
        \laplace{\redText{y'}}  & = & \lim_{M\rightarrow\infty} \redText{y(t)}e^{-st} \bigg|^M_0 + \int^M_0 s e^{-st} \redText{y(t)}~dt, \\
        & = & \lim_{M\rightarrow\infty} y(M)e^{-sM} - \redText{y(0)} + s \int^M_0 \redText{y(t)} e^{-st} ~dt, \\
        & = & -\redText{y(0)} + s \laplace{\redText{y}}.
      \end{eqnarray*}
    }

\end{frame}

\begin{frame}
  \begin{block}{Laplace Transform of the Derivative}
    \begin{eqnarray*}
      \laplace{y'} & = & -y(0) + s \laplace{y}.
    \end{eqnarray*}
  \end{block}
\end{frame}

\begin{frame}{Algebra, Algebra, Algebra }

  \vfill

  Why do we do this? There will not be any more integrals for the rest
  of the slides! Only algebra.

  \vfill
  
\end{frame}

\begin{frame}
  \frametitle{The Second Derivative}

  \begin{eqnarray*}
    \laplace{y'} & = & -y(0) + s \laplace{y}, \\
    \laplace{y''} & = & -y'(0) + s \laplace{\redText{y'}}, \\
    & = & -y'(0) + s \lp \redText{-y(0) + s \laplace{y}} \rp, \\
    & = & -y'(0) - s y(0) + s^2 \laplace{y}.
  \end{eqnarray*}

\end{frame}

\begin{frame}
  \begin{block}{Laplace Transform of the Derivative}
    \begin{eqnarray*}
      \laplace{y'} & = & -y(0) + s \laplace{y}, \\
      \laplace{y''} & = & -y'(0) - s y(0) + s^2 \laplace{y}.
    \end{eqnarray*}
  \end{block}
\end{frame}


\subsection{Solving ODEs}

\iftoggle{clicker}{%
\begin{frame}
  \frametitle{Clicker Quiz}

   \ifnum\value{clickerQuiz}=1{%
     Determine the Laplace transform of the function $f(t)=te^{4t}$.

     \vspace{2em}
     \begin{tabular}{ll}
       A: & $\frac{1}{(s-4)^2}$ \\ [12pt]
       B: & $\frac{-1}{(s-4)^2}$ \\ [12pt]
       C: & $\frac{1}{(s-4)}$ \\ [12pt]
       D: & $\frac{-1}{(s-4)}$ \\ [12pt]
     \end{tabular}

     \vfill
   }\fi

   \ifnum\value{clickerQuiz}=2{%
     Determine the Laplace transform of the function $f(t)=te^{2t}$.

     \vspace{2em}
     \begin{tabular}{ll}
       A: & $\frac{1}{(s-2)^2}$ \\ [12pt]
       B: & $\frac{-1}{(s-2)^2}$ \\ [12pt]
       C: & $\frac{1}{(s-2)}$ \\ [12pt]
       D: & $\frac{-1}{(s-2)}$ \\ [12pt]
     \end{tabular}

   \vfill
   }\fi

  \ifnum\value{clickerQuiz}=3{%
     %Determine the Laplace transform of the function $f(t)=t^2e^{3t}$.

     %\vspace{2em}
     %\begin{tabular}{ll}
     %  A: & $\frac{3!}{(s-2)^2}$ \\ [12pt]
     %  B: & $\frac{2!}{(s-3)^3}$ \\ [12pt]
     %  C: & $\frac{3!}{(s+2)^3}$ \\ [12pt]
     %  D: & $\frac{2!}{(s+3)^2}$ \\ [12pt]
     %\end{tabular}
     Determine the Laplace transform of the function $f(t)=te^{3t}$.

     \vspace{2em}
     \begin{tabular}{ll}
       A: & $\frac{1}{(s-3)^2}$ \\ [12pt]
       B: & $\frac{-1}{(s-3)^2}$ \\ [12pt]
       C: & $\frac{1}{(s-2)}$ \\ [12pt]
       D: & $\frac{-1}{(s-2)}$ \\ [12pt]
     \end{tabular}




  \vfill
 }\fi
\end{frame}
}



\begin{frame}
  \frametitle{So What?}

  \begin{eqnarray*}
    y' & = & t e^{3t}, \\
    y(0) & = & 4.
  \end{eqnarray*}

  \uncover<2->
  {
    \begin{eqnarray*}
      -y(0) + s \laplace{y} & = & \laplace{t e^{3t}}, \\
      \laplace{y} & = & \frac{1}{s} \lp 4 + \frac{1}{(s-3)^2} \rp, \\
      \laplace{y} & = & \frac{4}{s} + \frac{1}{s(s-3)^2}.
    \end{eqnarray*}
  }


\end{frame}


\begin{frame}
  \frametitle{Do the partial fractions thing}

  \begin{eqnarray*}
    \frac{1}{s(s-3)^2} & = & \frac{a}{s} + \frac{b}{s-3} + \frac{c}{(s-3)^2}, \\
    & = & \frac{1/9}{s} + \frac{-1/9}{s-3} + \frac{1/3}{(s-3)^2}.
  \end{eqnarray*}

  \uncover<2->
  {
    \begin{eqnarray*}
      \laplace{y} & = & 4 \frac{1}{s} + \frac{1}{9}\cdot\frac{1}{s}  - \frac{1}{9}\cdot\frac{1}{s-3} +
      \frac{1}{3}\cdot\frac{1}{(s-3)^2}, \\
      y & = & \frac{37}{9}  - \frac{1}{9}e^{3t} + \frac{1}{3} t e^{3t}.
    \end{eqnarray*}
  }

\end{frame}


\begin{frame}
  \frametitle{More Derivatives}

  \begin{eqnarray*}
    y'' & = & t e^{3t}, \\
    y'(0) & = & -2, \\
    y(0) & = & 4.
  \end{eqnarray*}

  \uncover<2->
  {
    \begin{eqnarray*}
      -y'(0) - s y(0) + s^2 \laplace{y} & = & \laplace{t e^{3t}}, \\
      \laplace{y} & = & \frac{1}{s^2} \lp -2 + 4s + \frac{1}{(s-3)^2} \rp, \\
      \laplace{y} & = & \frac{-2}{s^2} + \frac{4}{s} + \frac{1}{s^2(s-3)^2}.
    \end{eqnarray*}
  }


\end{frame}


\begin{frame}
  \frametitle{Do the partial fractions thing}

  \begin{eqnarray*}
    \frac{1}{s^2(s-3)^2} & = & \frac{a}{s} + \frac{b}{s^2} + \frac{c}{s-3} + \frac{d}{(s-3)^2}, \\
    & = & \frac{2/27}{s} + \frac{1/9}{s^2} + \frac{-2/27}{s-3} + \frac{1/9}{(s-3)^2}.
  \end{eqnarray*}

  \uncover<2->
  {
    \begin{eqnarray*}
      \laplace{y} & = & -2 \frac{1}{s^2} + 4 \frac{1}{s} +
      \frac{2}{27}\cdot\frac{1}{s} + \frac{1}{9}\cdot\frac{1}{s^2} - \frac{2}{27}\cdot\frac{1}{s-3} +
      \frac{1}{9}\cdot\frac{1}{(s-3)^2}, \\
      y & = & -2t + 4 + \frac{2}{27}  + \frac{1}{9} t - \frac{2}{27}e^{3t} + \frac{1}{9} t e^{3t}.
    \end{eqnarray*}
  }

\end{frame}


\begin{frame}
  \frametitle{Example}

  \begin{eqnarray*}
    y' + y & = & e^{-t}, \\
    y(0) & = & 1.
  \end{eqnarray*}

  \uncover<2->
  {
    \begin{eqnarray*}
      \laplace{y'} + \laplace{y} & = & \laplace{e^{-t}}, \\
        -y(0) + s \laplace{y} + \laplace{y} & = & \frac{1}{s+1}, \\
        (s+1) \laplace{y} & = & 1 + \frac{1}{s+1}, \\
        \laplace{y} & = & \frac{1}{s+1} + \frac{1}{(s+1)^2}, \\
        y & = & e^{-t} + t e^{-t}.
    \end{eqnarray*}
  }

\end{frame}


\iftoggle{clicker}{%
\begin{frame}
  \frametitle{Clicker Quiz}

   \ifnum\value{clickerQuiz}=1{%
     What is the Laplace transform of $y''$?

     \vspace{2em}
     \begin{tabular}{ll}
       A: & $-y(0)-sy'(0)-s^2\laplace{y}$ \\ [12pt]
       B: & $-y'(0)-sy(0)+s^2\laplace{y}$ \\ [12pt]
       C: & $y(0)+sy'(0)-s^2\laplace{y}$ \\ [12pt]
       D: & $y'(0)+sy(0)+s^2\laplace{y}$ \\ [12pt]
     \end{tabular}

     \vfill
   }\fi

   \ifnum\value{clickerQuiz}=2{%
     What is the Laplace transform of $y''$?

     \vspace{2em}
     \begin{tabular}{ll}
       A: & $-y(0)-sy'(0)-s^2\laplace{y}$ \\ [12pt]
       B: & $-y'(0)-sy(0)+s^2\laplace{y}$ \\ [12pt]
       C: & $y(0)+sy'(0)-s^2\laplace{y}$ \\ [12pt]
       D: & $y'(0)+sy(0)+s^2\laplace{y}$ \\ [12pt]
     \end{tabular}

   \vfill
   }\fi

  \ifnum\value{clickerQuiz}=3{%
     What is the Laplace transform of $y''$?

     \vspace{2em}
     \begin{tabular}{ll}
       A: & $-y(0)-sy'(0)-s^2\laplace{y}$ \\ [12pt]
       B: & $-y'(0)-sy(0)+s^2\laplace{y}$ \\ [12pt]
       C: & $y(0)+sy'(0)-s^2\laplace{y}$ \\ [12pt]
       D: & $y'(0)+sy(0)+s^2\laplace{y}$ \\ [12pt]
     \end{tabular}


  \vfill
 }\fi
\end{frame}
}



\begin{frame}
  \frametitle{Example}

  \begin{eqnarray*}
    y''  + 9y & = & 1, \\
    y'(0) & = & 0, \\
    y(0) & = & 1.
  \end{eqnarray*}

  \uncover<2->
  {
    \begin{eqnarray*}
      -y'(0) - s y(0) + s^2 \laplace{y} + 9 \laplace{y} & = & \laplace{1}, \\
      \laplace{y} & = & \frac{1}{s^2+9} \lp \frac{1}{s} + s \rp, \\
      \laplace{y} & = & \frac{1}{s(s^2+9)} + \frac{s}{s^2+9}.
    \end{eqnarray*}
  }


\end{frame}


\begin{frame}
  \frametitle{Do the partial fractions thing}

  \begin{eqnarray*}
    \frac{1}{s(s^2+9)} & = & \frac{a}{s} + \frac{bs+c}{s^2+9}, \\
    & = & \frac{1/9}{s} - \frac{s/9}{s^2+9}.
  \end{eqnarray*}

  \uncover<2->
  {
    \begin{eqnarray*}
      \laplace{y} & = & \frac{1}{9}\cdot\frac{1}{s} - \frac{1}{9}\cdot\frac{s}{s^2+9}+\frac{s}{s^2+9}, \\
      y & = & \frac{1}{9} + \frac{8}{9}\cos(3t).
    \end{eqnarray*}
  }

\end{frame}



\begin{frame}
  \frametitle{Example}

  \begin{eqnarray*}
    y''  + 2y' + y & = & 0, \\
    y'(0) & = & 2, \\
    y(0) & = & 0.
  \end{eqnarray*}

  \uncover<2->
  {
    \begin{eqnarray*}
      -y'(0) - s y(0) + s^2 \laplace{y} + 2 \lp -y(0) + s\laplace{y} \rp + \laplace{y} & = & 0, \\
      (s^2+2s+1) \laplace{y} & = & 2,
    \end{eqnarray*}
    \begin{eqnarray*}
      \laplace{y} & = & \frac{2}{s^2+2s+1}.
    \end{eqnarray*}
  }

  \uncover<3->
  {
    \begin{eqnarray*}
      \laplace{y} & = & \frac{2}{(s+1)^2}, \\
      y & = & 2 t e^{-t}.
    \end{eqnarray*}
  }

\end{frame}




\iftoggle{clicker}{%
\begin{frame}
  \frametitle{Clicker Quiz}

   \ifnum\value{clickerQuiz}=1{%
     What is the Laplace transform of $e^{2t}\sin(7t)$?

     \vspace{2em}
     \begin{tabular}{ll}
       A: & $\frac{7}{(s-2)^2+49}$ \\ [12pt]
       B: & $\frac{s}{(s-2)^2+49}$ \\ [12pt]
       C: & $\frac{2}{(s-7)^2+4}$ \\ [12pt]
       D: & $\frac{s}{(s-7)^2+4}$ \\ [12pt]
     \end{tabular}

     \vfill
   }\fi

   \ifnum\value{clickerQuiz}=2{%
     What is the Laplace transform of $e^{3t}\sin(4t)$?

     \vspace{2em}
     \begin{tabular}{ll}
       A: & $\frac{3}{(s-4)^2+9}$ \\ [12pt]
       B: & $\frac{s}{(s-4)^2+9}$ \\ [12pt]
       C: & $\frac{4}{(s-3)^2+16}$ \\ [12pt]
       D: & $\frac{s}{(s-3)^2+16}$ \\ [12pt]
     \end{tabular}

   \vfill
   }\fi

  \ifnum\value{clickerQuiz}=3{%
     What is the Laplace transform of $e^{-5t}\sin(3t)$?

     \vspace{2em}
     \begin{tabular}{ll}
       A: & $\frac{3}{(s+5)^2+9}$ \\ [12pt]
       B: & $\frac{s}{(s+5)^2+9}$ \\ [12pt]
       C: & $\frac{5}{(s-3)^2+25}$ \\ [12pt]
       D: & $\frac{s}{(s-3)^2+25}$ \\ [12pt]
     \end{tabular}


  \vfill
 }\fi
\end{frame}
}




\begin{frame}
  \frametitle{Example}

  \begin{eqnarray*}
    y''+y'+y & = & 0, \\
    y'(0) & = & 2, \\
    y(0) & = & 0.
  \end{eqnarray*}

  \uncover<2->
  {
    \begin{eqnarray*}
      -y'(0) - sy(0) + s^2 \laplace{y} - y(0) + s \laplace{y} + \laplace{y} & = & 0, \\
      \lp s^2+s+1 \rp \laplace{y} & = & 2, \\
      \laplace{y} & = & \frac{2}{s^2+s+1}.
    \end{eqnarray*}
        
  }

\end{frame}


\begin{frame}
  \frametitle{Algebra!}

  \begin{eqnarray*}
    \laplace{y} & = & \frac{2}{s^2+s+1}, \\
    & = & \frac{2}{s^2+s+\frac{1}{4} + \frac{3}{4}}, \\
    & = & \frac{2}{\lp s + \half \rp^2 + \frac{3}{4}}, \\
    & = & \frac{\sqrt{\frac{3}{4}}}{\lp s + \half \rp^2 + \frac{3}{4}} 2 \sqrt{\frac{4}{3}}.
  \end{eqnarray*}

  \uncover<2->
  {
    \begin{eqnarray*}
      y & = & \frac{4}{\sqrt{3}} e^{-t/2}\sin\lp\frac{\sqrt{3}}{2} t\rp.
    \end{eqnarray*}
  }

\end{frame}


\begin{frame}{Algebra, Algebra, Algebra }

  \vfill

  Again, the thing to note is that after the first identity was
  established we solved differential equations using algebra and no
  calculus.

  \vfill
  
\end{frame}



% LocalWords:  Clarkson pausesection hideothersubsections
